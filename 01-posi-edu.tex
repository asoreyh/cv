\ifeng
\section*{Current Positions}
\begin{itemize}
    \item \item \years{2021-current} Researcher (CNEA TNG 312 - Principal B) at the Medical Physics Department, Gerencia de Física (GF), and at Instituto de Tecnologías en Detección y Astropartículas (ITeDA), Gerencia de Área de Investigaciones y Aplicaciones No Nucleares (GAIYANN), Comisión Nacional de Energía Atómica (CNEA).
%    \item \years{2021-current} Researcher at the Unidad de Informática Científica, Centro de Investigaciones Energéticas, Medioambientales y Tecnológicas (CIEMAT)
    \item \years{2018-current} Associated Professor of the Física III B (Thermodynamics) and Física IV B (Introduction to Particle Physics, Astrophysics and Cosmology) courses of the Profesorado de Nivel Medio y Superior en Física of the Universidad Nacional de Río Negro (UNRN).
  % \item Associated Professor at the Double Doctorate in Astrophysics program, Universidad Nacional de San Martín (UNSAM).
  % \item Selected for incorporation to CONICET as Adjoint Researcher.
\end{itemize}
\else
\section*{Posiciones actuales}
\begin{itemize}
    \item \years{2021-presente} Investigador Principal B (CNEA TNG 312) en el Departamento Física Médica (DFM) y en el Instituto de Tecnologías en Detección y Astropartículas (ITeDA), Gerencia de Área de Investigaciones y Aplicaciones No Nucleares (GAIYANN), Comisión Nacional de Energía Atómica (CNEA).
%    \item \years{2021-presente} Investigador en la Unidad de Informática Científica del Centro de Investigaciones Energéticas, Medioambientales y Tecnológicas (CIEMAT)
	\item \years{2018-presente} Profesor Asociado con dedicación simple de los cursos de Física III B (Termodinámica) y Física IV B (Introducción a Física de Partículas, Astrofísica y Cosmología) del Profesorado de Nivel Medio y Superior en Física de la Universidad Nacional de Río Negro (UNRN).
\end{itemize}
\fi

\ifeng
\section*{Education}
\noindent
\years{2012}\textsc{Doctor in Physics (Ph.D.)}\\
{\emph{Institution}}: Particles and Fields Group, Centro Atómico Bariloche - Instituto Balseiro, CNEA-UNC. {\emph{Thesis}}: The Water Cherenkov Detectors of the Pierre Auger Observatory and their Application to the Study of Background Radiation. {\emph{Advisor}}: Dr. Ingomar Allekotte.
\years{2005}\textsc{Master in Science, Physics}\\
{\emph{Orientation}}: High Energy Physics. {\emph{Institution}}: Particles and Fields Group, Instituto Balseiro, Centro Atómico Bariloche (CNEA-UNC). {\emph{Thesis}}: Event Reconstruction with the Surface Detectors of the Pierre Auger Observatory. {\emph{Advisor}}: Dr. Ingomar Allekotte\\
\years{2004}\textsc{``Licenciado'' in Physics}\\
{\emph{Institution}}: Instituto Balseiro, Centro Atómico Bariloche (CNEA-UNC)\\
\else
\section*{Educación}
\noindent
\years{2012}\textsc{Doctor en Física}\\
{\emph{Institución}}: Grupo de Partículas y Campos, Instituto Balseiro, Centro Atómico Bariloche (CNEA-UNC). {\emph{Tesis}}: Los Detectores Cherenkov del Observatorio Pierre Auger y su Aplicación al Estudio de Fondos de Radiación. {\emph{Director}}: Dr. Ingomar Allekotte\\
\years{2005}\textsc{Magister en Ciencias Físicas}\\
{\emph{Orientación}}: Física de Partículas y Campos. {\emph{Institución}}: Grupo de Partículas y Campos, Instituto Balseiro, Centro Atómico Bariloche (CNEA-UNC). {\emph{Tesis}}: Reconstrucción de eventos con el Detector de Superficie del Observatorio Auger. {\emph{Director}}: Dr. Ingomar Allekotte\\
\years{2004}\textsc{Licenciado en Física}\\
{\emph{Institución}}: Instituto Balseiro, Centro Atómico Bariloche (CNEA-UNC)\\
\fi