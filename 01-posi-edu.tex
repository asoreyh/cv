\ifeng
\section*{Current Positions}
\begin{itemize}
	% \item Head of the Medical Physics Department, Gerencia de Física (GF), Gerencia de Área de Investigaciones y Aplicaciones No Nucleares (GAIYANN), Centro Atómico Bariloche (CAB), Comisión Nacional de Energía Atómica (CNEA), peer choice August 2017.
    \item Researcher (TNG 312 - Principal B) at the Medical Physics Department, Gerencia de Física (GF), and at Instituto de Tecnologías en Detección y Astropartículas (ITeDA), Gerencia de Área de Investigaciones y Aplicaciones No Nucleares (GAIYANN), Comisión Nacional de Energía Atómica (CNEA).
  % \item Jefe de Trabajos Prácticos at Insituto Balseiro, Science Department, Universidad Nacional de Cuyo (UNCuyo, licence).
  \item Associated Professor at Sede Andina, Universidad Nacional de Río Negro (UNRN, interinate).
  % \item Associated Professor at the Double Doctorate in Astrophysics program, Universidad Nacional de San Martín (UNSAM).
  % \item Selected for incorporation to CONICET as Adjoint Researcher.
\end{itemize}
\else
\section*{Posiciones actuales}
\begin{itemize}
	% \item Jefe del Departamento Física Médica (DFM), Gerencia de Área de Investigaciones y Aplicaciones No Nucleares (GAIYANN), Comisión Nacional de Energía Atómica (CNEA). Elección de pares, Agosto de 2017.
    \item Investigador Principal B (TNG 312) en el Departamento Física Médica (DFM) y en el Instituto de Tecnologías en Detección y Astropartículas (ITeDA), Gerencia de Área de Investigaciones y Aplicaciones No Nucleares (GAIYANN), Comisión Nacional de Energía Atómica (CNEA).
%	\item Investigador en el Departamento de Física Médica y en el Laboratorio Detección de Partículas y Radiación (LabDPR), Gerencia de Física (GF), Gerencia de Área de Investigaciones y Aplicaciones No Nucleares (GAIYANN), Centro Atómico Bariloche (CAB), Comisión Nacional de Energía Atómica (CNEA).
	% \item Investigador en el Instituto de Tecnologías en Detección y Astropartículas (ITeDA) Gerencia de Área de Investigaciones y Aplicaciones No Nucleares (GAIYANN), Comisión Nacional de Energía Atómica (CNEA).
%	\item Jefe de Trabajos Prácticos en el Insituto Balseiro, Área Ciencias, Universidad Nacional de Cuyo (UNCuyo, licencia).
	\item Profesor Asociado con dedicación simple en la Sede Andina, Universidad Nacional de Río Negro (UNRN).
%	\item Profesor Asociado con dedicación simple en la carrera del Doble Doctorado en Astrofísica, Universidad Nacional de San Martín (UNSAM).
%\item Seleccionado para ingreso al CONICET como Investigador Adjunto.
\end{itemize}
\fi

\ifeng
\section*{Education}
\noindent
\years{2012}\textsc{Doctor in Physics (Ph.D.)}\\
{\emph{Institution}}: Particles and Fields Group, Centro Atómico Bariloche - Instituto Balseiro, CNEA-UNC. {\emph{Thesis}}: The Water Cherenkov Detectors of the Pierre Auger Observatory and their Application to the Study of Background Radiation. {\emph{Advisor}}: Dr. Ingomar Allekotte.
\years{2005}\textsc{Master in Science, Physics}\\
{\emph{Orientation}}: High Energy Physics. {\emph{Institution}}: Particles and Fields Group, Instituto Balseiro, Centro Atómico Bariloche (CNEA-UNC). {\emph{Thesis}}: Event Reconstruction with the Surface Detectors of the Pierre Auger Observatory. {\emph{Advisor}}: Dr. Ingomar Allekotte\\
\years{2004}\textsc{``Licenciado'' in Physics}\\
{\emph{Institution}}: Instituto Balseiro, Centro Atómico Bariloche (CNEA-UNC)\\
\else
\section*{Educación}
\noindent
\years{2012}\textsc{Doctor en Física}\\
{\emph{Institución}}: Grupo de Partículas y Campos, Instituto Balseiro, Centro Atómico Bariloche (CNEA-UNC). {\emph{Tesis}}: Los Detectores Cherenkov del Observatorio Pierre Auger y su Aplicación al Estudio de Fondos de Radiación. {\emph{Director}}: Dr. Ingomar Allekotte\\
\years{2005}\textsc{Magister en Ciencias Físicas}\\
{\emph{Orientación}}: Física de Partículas y Campos. {\emph{Institución}}: Grupo de Partículas y Campos, Instituto Balseiro, Centro Atómico Bariloche (CNEA-UNC). {\emph{Tesis}}: Reconstrucción de eventos con el Detector de Superficie del Observatorio Auger. {\emph{Director}}: Dr. Ingomar Allekotte\\
\years{2004}\textsc{Licenciado en Física}\\
{\emph{Institución}}: Instituto Balseiro, Centro Atómico Bariloche (CNEA-UNC)\\
\fi