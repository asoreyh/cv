\ifeng
\section*{Current Positions}
\begin{itemize}
%    \item \item \years{2021-current} Researcher (CNEA TNG 312 - Principal B) at the Medical Physics Department, Gerencia de Física (GF), and at Instituto de Tecnologías en Detección y Astropartículas (ITeDA), Gerencia de Área de Investigaciones y Aplicaciones No Nucleares (GAIYANN), Comisión Nacional de Energía Atómica (CNEA).
%    \item \years{2021-current} Researcher at the Unidad de Informática Científica, Centro de Investigaciones Energéticas, Medioambientales y Tecnológicas (CIEMAT)
%   \item \years{2022-current} Piensas, founder and CEO: hard and soft, IoT, digital twins, project management
    \item \item \years{2016-2022} Researcher (CNEA TNG 312 - Principal B) at the Medical Physics Department, Gerencia de Física (GF), Comisión Nacional de Energía Atómica (CNEA).
    \item \years{2021-2022} Researcher at the Scientific Computing Unit, Centro de Investigaciones Energéticas, Medioambientales y Tecnológicas (CIEMAT)

%    \item \years{2018-current} Associated Professor of the Física III B (Thermodynamics) and Física IV B (Introduction to Particle Physics, Astrophysics and Cosmology) courses of the Profesorado de Nivel Medio y Superior en Física of the Universidad Nacional de Río Negro (UNRN).
  % \item Associated Professor at the Double Doctorate in Astrophysics program, Universidad Nacional de San Martín (UNSAM).
  % \item Selected for incorporation to CONICET as Adjoint Researcher.
\end{itemize}
\else
\section*{Posiciones actuales}
\begin{itemize}
%    \item \years{2022-current} Piensas, fundador y CEO: hard and soft, IoT, gemelos digitales, project management
    \item \years{2021-presente} Investigador Principal (CNEA TNG 312)\footnote{Categoría equivalente a Jefe de División. Última evaluación reportada periodo 2016--2018.} en el Departamento Física Médica (DFM), Gerencia de Área de Investigaciones y Aplicaciones No Nucleares (GAIYANN), Comisión Nacional de Energía Atómica (CNEA).
    \item \years{2021-presente} Estancia de investigación en la Unidad de Informática Científica del Centro de Investigaciones Energéticas, Medioambientales y Tecnológicas (CIEMAT)
    \item \years{2018-presente} Profesor Asociado\footnote{Las categorías docentes en Argentina se ordenan de la siguiente forma: Profesor Titular, Profesor Asociado, Profesor Adjunto, Jefe de Trabajos Prácticos, Auxiliar de Primera y Auxiliar de Segunda.} con dedicación simple de los cursos de Física III B (Termodinámica) y Física IV B (Introducción a Física de Partículas, Astrofísica y Cosmología) del Profesorado de Nivel Medio y Superior en Física de la Universidad Nacional de Río Negro (UNRN).
\end{itemize}
\fi

\ifeng
\section*{Key Performance Indicators (KPIs)}

\begin{itemize}
\item \years{Production} 136 publications in peer-reviewed indexed journals; 88 participations and presentations in Schools, Congresses, Conferences, or Symposia; 26 technical reports from CNEA and technical notes from the Pierre Auger Observatory.
\item \years{\href{http://www.scopus.com/authid/detail.url?authorId=35276880300}{Scopus}} h-index=47, 13,799 citations in 155 articles in peer-reviewed indexed journals, and 50 preprints.
\item \years{\href{https://scholar.google.com.co/citations?user=Vj7_fGsAAAAJ}{Scholar}} h${\text{tot}}$=58, h$5$=45 (since 2017), i${10}$=126 (i${10}$=103 since 2017), 327 articles indexed in Scholar and 23,521 (11,609 since 2017) in 155 peer-reviewed indexed journals. 88 contributions and presentations in symposia and congresses.
\item \years{Management} Principal Investigator or Co-Investigator in 12 national and international R\&D+i projects. Principal Investigator in an international collaboration (2013--2016). Head of the Medical Physics Department of CNEA (2017--2021). Project Manager in 3 international projects.
\item \years{Awards} Two awards for teaching performance.
\item \years{Patents and books} Author of a \href{https://editorial.unrn.edu.ar/index.php/catalogo/346/view_bl/62/lecturas-de-catedra/92/fisica-ia-de-las-galaxias-a-los-quarks}{introductory physics textbook}. A national and international patent for a neutron detector.
\item \years{Education} Advisor to 2 postdoctoral researchers, 5 doctoral, 5 master's, and 7 undergraduate physics students.
\end{itemize}

\else
\section*{Indicadores de Rendimiento 2005-2023}

\begin{itemize}
    \item \years{producción} 136 publicaciones en revistas indexadas con revisión de pares; 88 participaciones y presentaciones en Escuelas, Congresos, Conferencias o Simposios; 26 reportes técnicos de CNEA y notas técnicas del Observatorio Pierre Auger.
    \item \years{\href{http://www.scopus.com/authid/detail.url?authorId=35276880300}{Scopus}} h-index=47, 13.799 citas en 155 artículos en revistas indexadas con revisión por pares y 50 preprints.
    \item \years{\href{https://scholar.google.com.co/citations?user=Vj7_fGsAAAAJ}{Scholar}} h$_{\text{tot}}$=58, h$_5$=45 (desde 2017), i$_{10}$=126 (i$_{10}$=103 desde 2017), 327 artículos indexados en Scholar y 23521 (11609 desde 2017) en 155 revistas indexadas revisadas por pares. 88 contribuciones y presentaciones en simposios y congresos.
    \item \years{Gestión} Investigador Principal o Co-Investigador en 12 proyectos de I+D+i nacionales e internacionales. Investigador Principal en una colaboración internacional (2013--2016). Jefe del Departamento de Física Médica de la CNEA (2017--2021). Project Manager en 3 proyectos internacionales.
    \item \years{Premios} Dos premios por desempeño docente.
    \item \years{patentes y libros} Autor de un \href{https://editorial.unrn.edu.ar/index.php/catalogo/346/view_bl/62/lecturas-de-catedra/92/fisica-ia-de-las-galaxias-a-los-quarks}{libro de texto de física introductoria}. Una patente de ámbito nacional e internacional de un detector de neutrones.
    \item \years{formación} Asesor de 2 investigadores Postdoctorales, 5 de Doctorado, 5 de Maestría y 7 de Licenciatura en Física.
\end{itemize}
\fi

\ifeng
\section*{Education}
\noindent
\years{2012}\textsc{Doctor in Physics (Ph.D.)}\\
{\emph{Institution}}: Particles and Fields Group, Centro Atómico Bariloche (CNEA) -- Instituto Balseiro, Universidad Nacional de Cuyo (UNC). {\emph{Thesis}}: The Water Cherenkov Detectors of the Pierre Auger Observatory and their Application to the Study of Background Radiation. {\emph{Advisor}}: Dr. Ingomar Allekotte.
\years{2005}\textsc{Master in Science, Physics}\\
{\emph{Orientation}}: High Energy Physics. {\emph{Institution}}: Particles and Fields Group, Centro Atómico Bariloche (CNEA) -- Instituto Balseiro (UNC). {\emph{Thesis}}: Event Reconstruction with the Surface Detectors of the Pierre Auger Observatory. {\emph{Advisor}}: Dr. Ingomar Allekotte\\
\years{2004}\textsc{Licenciado in Physics}\\
{\emph{Institution}}: Instituto Balseiro, Centro Atómico Bariloche (CNEA-UNC)\\
\else
\section*{Educación}
\noindent
\years{2012}\textsc{Doctor en Física}\\
{\emph{Institución}}: Grupo de Partículas y Campos, Centro Atómico Bariloche (CNEA) -- Instituto Balseiro, Universidad Nacional de Cuyo (UNC), . {\emph{Tesis}}: Los Detectores Cherenkov del Observatorio Pierre Auger y su Aplicación al Estudio de Fondos de Radiación. {\emph{Director}}: Dr. Ingomar Allekotte\\
\years{2005}\textsc{Magíster en Ciencias Físicas}\\
{\emph{Orientación}}: Física de Partículas y Campos. {\emph{Institución}}: Grupo de Partículas y Campos, Centro Atómico Bariloche (CNEA) -- Instituto Balseiro (UNC). {\emph{Tesis}}: Reconstrucción de eventos con el Detector de Superficie del Observatorio Auger. {\emph{Director}}: Dr. Ingomar Allekotte\\
\years{2004}\textsc{Licenciado en Física}\\
{\emph{Institución}}: Centro Atómico Bariloche (CNEA) -- Instituto Balseiro (UNC)\\
\fi