\ifeng
\section*{Outreach \& Complementary Activities}
\else
\section*{Divulgación \& Actividades de Extensión}
\fi
\noindent
\years{2011}H. Asorey, A. Clúa, A. López Dávalos \href{http://www.clarin.com/sociedad/Cien-millones-toneladas-cenizas-solo_0_517148395.html}{Cien millones de toneladas en un sólo día}, 
\ifeng
Clarín (national circulation newspaper), 2011. Reproduced in hundreds of Argentinian and international newspapers and media.
\else
Clarín (diario de circulación nacional), 2011. Reproducido en cientos de medios argentinos e internacionales.
\fi

\years{2011}H. Asorey, {\emph{Viviendo con una estrella}}, 
\ifeng 
Solar physics and space weather phenomena talk, oriented to general public and high-school students of the Rio Negro Province. Begin: March-2011
\else
Charla para todo público sobre Física Solar y Cliamatología Espacial, orientada para estudiantes secundarios de la Provincia de Río Negro. Comienzo: Marzo-2011
\fi

\years{2010}{\emph{Distinguen trabajo de Investigadores del Centro Atómico Bariloche}} (H. Asorey, X. Bertou, M. Gómez Berisso), El Cordillerano, Bariloche 2000 y ANBariloche, 2010.

\years{2010}Laura García, {\emph{Red Latinoamericana de Detectores para Estudiar Radiación Gamma}} (H. Asorey, X. Bertou, M. Gómez Berisso), El Cordillerano, Bariloche 2000 y ANBariloche, 2010.

\years{2009}H. Asorey, 
\ifeng
{\emph{Astrophysics for everyone}}, bimonthly column in the ``Nature and technology'' local magazine. \else
{\emph{Astrofísica para todos}}, columna bimestral en la revista ``Naturaleza y Tecnología''
\fi

\years{2008}H. Asorey, 
\ifeng
{\emph{The Pierre Auger Observatory: a look to the Universe to the highest energies}}, invited talk oriented for general public, National University of Quilmes, Argentina, April 2008.
\else
{\emph{El Observatorio Pierre Auger: una mirada al Universo a las más altas energías}}, charla para todo público dada en la Universidad Nacional de Quilmes, Abril de 2008.
\fi
