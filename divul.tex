\ifeng
\section*{Outreach \& Complementary Activities}
\else
\section*{Divulgación \& Actividades de Extensión}
\fi
\noindent

\years{2017}H. Asorey, \href{}{Energía, Humanidad y Cambio Climático}, Ciclo de charlas en escuelas de educación media, San Carlos de Bariloche, Argentina.

\years{2015}H. Asorey, \href{http://www.semanadelaciencia.mincyt.gob.ar/}{Energía, Humanidad y Cambio Climático}, ``XIII Semana Nacional de la Ciencia y la Tecnología'', Sede Andina, Universidad Nacional de Río Negro, Bariloche, Argentina.

\years{2015}H. Asorey \& A. Cutsaimanis, \ifeng ``¿Qué onda con las ondas?'', Training course for Secondary School Teachers \else Curso de capacitación para docentes de enseñanza media ``¿Qué onda con las ondas?'',\fi Instituto Nacional de Formación Docente (INFOD), Ministerio de Educación, Viedma, Río Negro. \ifeng Role: professor and trainer.\else Rol: profesor y capacitador.\fi

\years{2009-2015}H. Asorey, \href{http://fisicareconocida.blogspot.com}{Física ReConocida} \ifeng Physics blog in spanish and facebook group. \else Blog de Física en español y grupo de facebook.\fi 

\years{2013-2014} H. Asorey \& L. Núñez, \href{http://halley.uis.edu.co/fisica_para_todos}{Física para todos}, \ifeng Introductory physics blog, School of Physics, \else blog de física general para la materia Introducción a la Física, Escuela de Física,\fi Universidad Industrial de Santander. 

\years{2014}H. Asorey, {\it{Energía, Humanidad y Cambio Climático}}, ``Café Científico'', La Casa del Libro Total, Bucaramanga, Colombia

\years{2011}H. Asorey \& A. López Dávalos, {\emph{Fermi Problem: Power
developed at the eruption of the Puyehue-Cordón Caulle volcanic system in June
2011}}, \href{http://arxiv.org/abs/1109.1165}{arXiv:1109.1165v1}[physics.ed-ph]. \ifeng
Selected as the best \href{http://arxiv.org}{arXiv} paper of September 2011 by the
\else
Seleccionado como el mejor trabajo enviado al \href{http://arxiv.org}{arXiv} durante Setiembre del 2011 por el blog
\fi
\href{http://www.technologyreview.com/blog/arxiv/27140/}{M.I.T. Technology
Review Physics arXiv Blog}, (2011)

\years{2011}H. Asorey, A. Clúa, A. López Dávalos \href{http://www.clarin.com/sociedad/Cien-millones-toneladas-cenizas-solo_0_517148395.html}{Cien millones de toneladas en un sólo día}, 
\ifeng
Clarín (national circulation newspaper), 2011. Reproduced in hundreds of Argentinian and international newspapers and media.
\else
Clarín (diario de circulación nacional), 2011. Reproducido en cientos de medios argentinos e internacionales.
\fi

\years{2011}H. Asorey, {\emph{Viviendo con una estrella}}, 
\ifeng 
Solar physics and space weather phenomena talk, oriented to general public and high-school students of the Rio Negro Province. Start: March-2011
\else
Charla para todo público sobre Física Solar y Climatología Espacial, orientada para estudiantes secundarios de la Provincia de Río Negro. Comienzo: Marzo-2011
\fi

\years{2010}{\emph{Distinguen trabajo de Investigadores del Centro Atómico Bariloche}} (H. Asorey, X. Bertou, M. Gómez Berisso), El Cordillerano, Bariloche 2000 y ANBariloche.

\years{2010}Laura García, {\emph{Red Latinoamericana de Detectores para Estudiar Radiación Gamma}} (H. Asorey, X. Bertou, M. Gómez Berisso), El Cordillerano, Bariloche 2000 y ANBariloche, 2010.

\years{2009}H. Asorey, 
\ifeng
{\emph{Astrophysics for everyone}}, bimonthly column in the ``Nature and technology'' local magazine. \else
{\emph{Astrofísica para todos}}, columna bimestral en la revista ``Naturaleza y Tecnología''
\fi

\years{2008}H. Asorey, 
\ifeng
{\emph{The Pierre Auger Observatory: a look to the Universe to the highest energies}}, invited general public talk, National University of Quilmes, Argentina, April 2008.
\else
{\emph{El Observatorio Pierre Auger: una mirada al Universo a las más altas energías}}, charla para todo público dada en la Universidad Nacional de Quilmes, Abril de 2008.
\fi
