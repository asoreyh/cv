\ifeng
\section*{Human Resources Training}
\else
\section*{Formación de Recursos Humanos}
\fi

% cuenta los dobles % 
% pre-end
% mag-end
% phd-end
% pdo-end
% pre-cur
% mag-cur
% phd-cur
% pdo-cur
\ifeng
	\subsection*{Underway}
\else
	\subsection*{Trabajos en ejecución}
\fi

\years{2021} %% pdo-cur
\ifeng
Posdoc fellowship advisor
\else
Director de beca posdoctoral
\fi
``Estudios de aplicación de técnicas de detección de radiación cósmica para la detección materiales con alto número atómico'', Dr. Christian Sarmiento-Cano \at \ifeng ITeDA\else ITeDA\fi, Argentina.

\years{2021} %% phd-cur
\ifeng
PhD thesis advisor
 \else
Director de tesis de Doctorado en Física
 \fi
``Estudios de aplicación de técnicas de detección de radiación cósmica para la detección de radiación gamma y materiales con alto número atómico'', Johan Serrano Contreras \at Instituto Sábato, Universidad Nacional de San Martín, Argentina.

\years{2020} %% pdo-cur
\ifeng
Posdoc fellowship advisor
\else
Director de beca posdoctoral
\fi
``Caracterización de Experimentos de Búsqueda de Materia Oscura y Física de Neutrinos con proyección al Laboratorio Subterráneo ANDES'', Dr. Álvaro Taboada \at \ifeng ITeDA\else ITeDA\fi, Argentina.

\years{2019}
\ifeng
PhD double doctoral thesis in Physics co-advisor
\else
Co-director de la tesis de Doble Doctorado en Física
\fi
``Performance of the Upgraded Surface Detector of the Pierre Auger Observatory'', Alexander Streich \at Universidad Nacional de San Martín, Argentina \ifeng and \else y \fi Karlsruher Instituts für Technologie (KIT), \ifeng Germany\else Alemania\fi.

\years{2018} %% phd-cur
\ifeng
PhD thesis advisor
 \else
Director de tesis de Doctorado en Física
 \fi
``Desarrollo de Técnicas de Muongrafía para Estudios Densitométricos de Objetos de Importancia Estratégica'', Rolando Calderón Ardila \at Instituto Sábato, Universidad Nacional de San Martín, Argentina.

\ifeng
\subsection*{Completed}
\else
\subsection*{Trabajos terminados}
\fi

\years{2021} %% phd-end
\ifeng
PhD thesis co-advisor
 \else
Co-director de tesis de Doctorado en Física
 \fi
``Diseño y calibración de un telescopio de muones híbrido para estudios vulcanológicos'', Jesús Peña Rodríguez \at \ifeng Universidad Industrial de Santander \else Universidad Industrial de Santander\fi (UIS), Bucaramanga, Colombia. \ifeng Qualification \else Nota obtenida: \fi 5/5 \ifeng Thesis awarded with a Honorific Mention at UIS.\else Tesis premiada con Mención de Honor en la UIS.\fi

\years{2020} %% pre-end
\ifeng
Physics thesis advisor
 \else
Director de tesis de Licenciatura en Física
 \fi
``Estimación del flujo de muones en el laboratorio subterráneo ANDES'', Lic. Carmina Perez Bertolli, \at Facultad de Ciencias Exactas y Naturales, Universidad Nacional de Buenos Aires (UBA), \ifeng Qualification \else Nota obtenida: \fi 10/10. \ifeng Winner of the 2020 Masperi Prize, awarded to the best Undergraduate Thesis in Physics presented at the 105th Annual Meeting of the Argentinian Physics Association, Córdoba, Argentina, 2020.\else Ganadora del Premio Masperi 2020 a la mejor tesis de física presentada en la 105º Reunión Anual de la Asociación de Física Argentina, Córdoba, Argentina, 2020. \fi 

\years{2019} %% phd-end
\ifeng
PhD thesis coadvisor
 \else
Codirector de tesis de Doctorado en Física
 \fi
``Variaciones del flujo de radiación cósmica en el suelo y en escenarios geofísicos'', Mauricio Suárez Durán \at \ifeng School of Physics\else Escuela de Física\fi, Universidad Industrial de Santander, Bucaramanga, Colombia. \ifeng Qualification \else Nota obtenida: \fi 5/5

\years{2017} %% mag-end
\ifeng
Master in Sciences thesis co-advisor
 \else
Codirector de tesis de Maestría en Ciencias Física
 \fi
``Eficiencia de un detector Cherenkov en agua para la detección de neutrones'', Nicolás Guarín \at \ifeng Instituto Balseiro\else Instituto Balseiro\fi, Universidad Nacional de Cuyo, Bariloche, Argentina. \ifeng Qualification \else Nota obtenida: \fi 10/10

\years{2015} %% mag-end
\ifeng
Master in Physis thesis advisor
 \else
Director de tesis de Maestría en Física
 \fi
 ``Aplicaciones en Meteorología Espacial de los Datos del Proyecto LAGO'', Yunior Perez \at \ifeng Physics Department\else Departamento de Física\fi, Universidad de los Andes, Mérida, Venezuela, \ifeng Qualification \else Nota obtenida: \fi 20/20, \ifeng Thesis Awarded with a Publication Mention (Honored Mention) at ULA \else Mención de Publicación (Mención Honorífica) en la ULA\fi.

\years{2015} %% mag-end
\ifeng
Master in Physis thesis advisor of
 \else
Director de tesis de Maestría en Física de
 \fi
``Búsqueda de Fuentes de Astropartículas en los Datos de la Colaboración LAGO'', Christian Sarmiento-Cano \at \ifeng School of Physics\else Escuela de Física\fi, Universidad Industrial de Santander, Bucaramanga, Colombia. \ifeng Qualification \else Nota obtenida: \fi 5/5, \ifeng Thesis Awarded with the Meritorious Mention\else Tesis premiada con una Mención de Mérito\fi.

\years{2015} %% mag-end
\ifeng
Master in Physis thesis advisor of
 \else
Director de tesis de Maestría en Física de
 \fi
``Modulación de Rayos Cósmicos Galácticos a nivel del suelo por cambios en el Campo Geomagnético y aplicaciones a Meteorología Espacial en el Proyecto LAGO'', Mauricio Suárez Durán \at \ifeng School of Physics\else Escuela de Física\fi, Universidad Industrial de Santander, Bucaramanga, Colombia. \ifeng Qualification \else Nota obtenida: \fi 5/5, \ifeng Thesis Awarded with a Meritorious Mention\else Tesis premiada con una Mención de Mérito\fi.

\years{2015} %% pre-end
\ifeng
Physics thesis advisor of
 \else
Director de tesis de Grado en Física de
 \fi
``Meteorología Espacial y la Navegación Aérea'', Sergio Pinilla \at \ifeng School of Physics\else Escuela de Física\fi, Universidad Industrial de Santander, Bucaramanga, Colombia.  \ifeng Qualification \else Nota obtenida: \fi 5/5, \ifeng Award-winning thesis\else Tesis Laureada\fi.

\years{2015} %% pre-end
\ifeng
Physics thesis advisor
 \else
Director de tesis de Licenciatura en Física
 \fi
``Sensibilidad del Proyecto LAGO a Señales Gamma provenientes del Centro de la Galaxia'', Arturo Núñez \at \ifeng Physics Department\else Departamento de Física\fi, Universidad de los Andes, Mérida, Venezuela, \ifeng Qualification \else Nota obtenida: \fi 20/20.

\years{2015} %% pre-end
\ifeng
Physics thesis advisor
 \else
Director de tesis de Grado en Física
 \fi
``Método de {\textit{Thinning}} y {\textit{Dethinning}} para Lluvias de Primarios de Alta Energía'', Alex Estupiñán \at \ifeng School of Physics\else Escuela de Física\fi, Universidad Industrial de Santander, Bucaramanga, Colombia, \ifeng Qualification \else Nota obtenida: \fi 5/5, \ifeng Award-winning thesis\else Tesis Laureada\fi.

\years{2015} %% pre-end
\ifeng
Physics thesis advisor
 \else
Director de tesis de Grado en Física
 \fi
``Simulación de los detectores Cherenkov en agua de la colaboración LAGO'', Rolando Calderón Ardila \at \ifeng School of Physics\else Escuela de Física\fi, Universidad Industrial de Santander, Bucaramanga, Colombia, \ifeng Qualification \else Nota obtenida: \fi 4.8/5. 

\years{2014} %% pre-end
\ifeng
System Engineering thesis advisor
 \else
Codirector de tesis de Ingeniería en Sistemas
 \fi
``Visualización de Cascadas de Rayos Cósmicos sobre GPUs'', Rafael Laverde \at \ifeng School of System Engineering\else Escuela de Ingeniería en Sistemas\fi, Universidad Industrial de Santander, Bucaramanga, Colombia, \ifeng Qualification \else Nota obtenida: \fi 4.8/5.

\years{2014} %% pre-end
\ifeng
Physics thesis advisor
 \else
Director de tesis de Licenciatura en Física
 \fi
``Estudios de la Respuesta del Arreglo de Detectores de Superficie del Observatorio Pierre Auger de Rayos Cósmicos'', Lic. Jonathan David Bossio Solá, \at Facultad de Ciencias Exactas y Naturales, Universidad Nacional de Buenos Aires (UBA), \ifeng Qualification \else Nota obtenida: \fi 10/10.