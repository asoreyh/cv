\vspace{-1cm}
\ifres
\else
\ifeng
\section*{Previous positions}
\noindent
\years{2017-2021} Head of the Medical Physics Department, Gerencia de Física (GF), Gerencia de Área de Investigaciones y Aplicaciones No Nucleares (GAIYANN), Centro Atómico Bariloche (CAB), Comisión Nacional de Energía Atómica (CNEA), peer choice.\\
\years{2015-2017} Researcher (TNG 422 - Principal C) at the Particle and Fields Division, Gerencia de Física (GF), Gerencia de Área de Investigaciones y Aplicaciones No Nucleares (GAIYANN), Comisión Nacional de Energía Atómica (CNEA).\\
\years{2015-2017} Associated Professor of the Física Moderna A (2015 y 2017), Física I A (2016), Física II B (Waves, 2015), Física III B (Thermodynamics, 2018-current), Física IV B (Introduction to Particle Physics, Astrophysics and Cosmology, 2016-current) of the Profesorado de Nivel Medio y Superior en Física of the Universidad Nacional de Río Negro (UNRN).\\
\years{2014-2015} Invited Professor at the School of Physics, Universidad Industrial de Santander, Bucaramanga, Colombia. Junior researcher at COLCIENCIAS.\\
\years{2013-2014} Post-doctoral researcher at Grupo de Investigación en Relatividad y Gravitación and Grupo Halley de Astronomía y Ciencias Aeroespaciales, Physics School, Universidad Industrial de Santander, Bucaramanga, Colombia.\\
\years{2013-2014} Assistant Professor at the School of Physics, Universidad Industrial de Santander, Bucaramanga, Colombia.\\
\years{2012} Senior Teaching Assistant (Jefe de Trabajos Prácticos) in charge of the Física I A and Física I B (Introduction to Physics) courses of the Profesorado de Nivel Medio y Superior en Física, Universidad Nacional de Río Negro (UNRN)\\
\years{2009-2011} Senior Teaching Assistant (Jefe de Trabajos Prácticos), courses Física I A and Física I B (Introduction to Physics) of the Profesorado de Nivel Medio y Superior en Física, Universidad Nacional de Río Negro (UNRN)\\
\years{2010-2012} Teaching Assistant at Science Department, Instituto Balseiro, Universidad Nacional de Cuyo (UNC)\\
\years{2006-2012} Ph.D. student, Instituto Balseiro (UNC).\\
\years{2004-2005} Master in Science, Instituto Balseiro (UNC).\\
\years{2002-2004} Physics undergraduate student, Instituto Balseiro (UNC).\\
\years{1992-1996} Industrial Engineering (first four of five years). University of Buenos Aires.\\
\years{1990-2001} AIM S.A., metal mechanical industry, R+D department in industrial projects, Buenos Aires, Argentina.\\
\else
\section*{Posiciones anteriores}
\noindent
\years{2017-2021} Jefe del Departamento Física Médica (DFM), Gerencia de Área de Investigaciones y Aplicaciones No Nucleares (GAIYANN), Comisión Nacional de Energía Atómica (CNEA). Elección de pares.\\
\years{2015-2017} Investigador Principal C (TNG 422) en la División Partículas y Campos, Gerencia de Física (GF), Gerencia de Área de Investigaciones y Aplicaciones No Nucleares (GAIYANN), Comisión Nacional de Energía Atómica (CNEA).\\
\years{2015-2017} Profesor Asociado con dedicación simple de los cursos de Física Moderna A (2015 y 2017), Física I A (2016), Física II B (Ondas, 2015), Física III B (Termodinámica, 2018-presente), Física IV B (Introducción a Física de Partículas, Astrofísica y Cosmología, 2016-presente) del Profesorado de Nivel Medio y Superior en Física de la Universidad Nacional de Río Negro (UNRN).\\
\years{2014-2015} Profesor Invitado en la Escuela de Física, Universidad Industrial de Santander, Bucaramanga, Colombia, s/Resolución Rectoría 1706/2014. Finalización de estadía: 31/Marzo/2015. Reconocido como Investigador categoría senior en la convocatoria COLCIENCIAS 2016.\\
\years{2013-2014} Investigador post-doctoral en el Grupo de Investigación en Relatividad y Gravitación y en el Grupo Halley de Astronomía y Ciencias Aeroespaciales, Escuela de Física, Universidad Industrial de Santander, Bucaramanga, Colombia. Reconocido como Investigador categoría junior en la convocatoria COLCIENCIAS 640/2013.\\
\years{2013-2014} Profesor Cátedra en la Escuela de Física, Universidad Industrial de Santander, Bucaramanga, Colombia.\\
\years{2012} Jefe de Trabajos Prácticos, a cargo del dictado de los cursos Física I A y Física I B (Introducción a la Física) del Profesorado de Nivel Medio y Superior en Física, Universidad Nacional de Río Negro (UNRN).\\
\years{2009-2011} Jefe de Trabajos Prácticos de la cátedra de Física I A y Física I B (Introducción a la Física) del Profesorado de Nivel Medio y Superior en Física, Universidad Nacional de Río Negro (UNRN).\\
\years{2010-2012} Auxiliar de primera, interino, del Área Ciencias, Instituto Balseiro, UNC.\\
\years{2006-2012} Doctorado en Física, Instituto Balseiro (UNC).\\
\years{2004-2005} Maestría en Ciencias Físicas, Instituto Balseiro (UNC).\\
\years{2002-2004} Licenciatura en Física, Instituto Balseiro (UNC).\\
\years{1992-1996} Ingeniería Industrial (primeros cuatro años). Universidad de Buenos Aires.\\
\years{1992-2001} AIM S.A., metalúrgica industrial, a cargo del diseño y ejecución de proyectos industriales, Bernal, Buenos Aires, Argentina.\\
\fi
\fi

\ifeng
\section*{Honours, Awards, Fellowships \& Grants}
\noindent
\years{2015} Universidad Industrial de Santander ``2013-2014 Best Professor of the Science Faculty Award'' for outstanding teaching skills at School of Physics\\
\years{2011} Balseiro Foundation ``Best Teacher Award'' for outstanding teaching skills at Instituto Balseiro.\\
\years{2022} ``Detectores de astropartículas y sus aplicaciones: muongrafía de grandes estructuras y meteorología espacial'', PICT2021-GRF-TII-00301, under evaluation\\
\years{2022} ``Astroparticle simulations and its applications'', European Grid Infrastructure - Advanced Computing for EOSC (EGI-ACE) Use Case, under evaluation\\
\years{2021} ``Detectores modulares para imágenes con Muones de fondo'', Fundación Hermanos Agustín y Enrique Rocca, running.\\
\years{2021} ``Utilización y desarrollo de ligandos específicos del microambiente tumoral acoplados a 177Lu para la detección y tratamiento de tumores primarios y metástasis'', Fundación Balseiro \& CNEA s/resol 306/21, running.\\
\years{2020} ``EOSC synergy – Building capacity, developing capability'', Horizon 2020 RI project 857647, Thematic Service Leadership, running.\\
\years{2020} ``Desarrollo de Técnicas de Muongrafía para Estudios Densitométricos de Objetos de Importancia Estratégica, II'' ASUTNBA0018565, running.\\
\years{2020} ``PlomBOX: un dispositivo de metrología de código abierto para combatir la contaminación por plomo en el agua potable mediante sensores biosintéticos'' GCRF Award R11178, running.\\
\years{2019} ``Desarrollo de Técnicas de Muongrafía para Estudios Densitométricos de Objetos de Importancia Estratégica'' ASUTNBA0005202, running.\\
\years{2019} ``Muongrafía de grandes estructuras'' SIIP2019-C035, approved.\\
\years{2018} ``Desarrollo de detectores de radiación'' PICT 2018-2886 (Argentina Innovadora 2020) Agencia, approved.\\
\years{2017} ``Desarrollo de detectores de neutrones basados en efecto Cherenkov en agua'', SECYT 06/C4863 (UNCuyo, Argentina), approved.\\
\years{2016} ``Astroparticle Detectors'', PICT 2015-2428 Grant (Agencia-MinCyT, Argentina), approved.\\
\years{2010-presente} Admission in the Researcher Professors Incentive Programs SPU/ME (Cat V, 2010 call; cat, III 2015 call, current).\\
\ifres
\else
\years{2015} Argentina-Colombia Cooperation Project Level II (PCB-II), ``Aplicación de Técnicas de Muongrafía para el Estudio de Estructuras Volcánicas de Riesgo'', MinCyT-CONICET-COLCIENCIAS: approved.\\
\years{2014} ``Nuclear Interactions Detections in CCDs for Dark Matter Search'', PICT 2013-2128 Grant (Agencia-MinCyT, Argentina): finished and approved.\\
\years{2014} ``Teaching-Research Articulation Project'' internal proposal for the Universidad Industrial de Santander 2014, with the proposal ``Introduction to XXI Century Physics: the best way to learn physics is doing physics'' (Director). Status: finished and approved.\\
\years{2014} ``GUANE3$^+$: Upgrade of the UIS GUANE Array of Water Cherenkov Astroparticle Detectors by the incorporation of plastic scintillators for Space Weather Studies'' internal research proposal for the Universidad Industrial de Santander (co-director). Status: finished and approved.\\
\years{2014} ``MuTe: Muon telescope for Volcanic Muongraphy'' proposal for the Colombian Council of Science COLCIENCIAS 660/2014 call (researcher). Status: approved (started in 2015).\\
\years{2014} ``Study of the Factibility of Volcanic Muongraphy techniques'' proposal for the Colombian Council of Science COLCIENCIAS 653/2014 call (researcher). Status: Selected.\\
\years{2013} ''Generate an Educative Experience under the Citizen Science paradigma as the base for a future MOOC'' proposal for FRIDA Foundation 2014 call (researcher). Status: approved.\\
\years{2013} ``The GUANE Array of Astroparticle Detectors for Space Weather Studies'' (co-director) internal proposal for the Universidad Industrial de Santander 2013 (co-director). Status: approved.\\
\years{2014-2015} Posdoctoral fellowship, Universidad Industrial de Santander, Bucaramanga, Colombia.
\years{2008-2010}Fellowship awarded by the National Council of Scientific and Technical Investigations (CONICET) to obtain a Ph.D. degree.\\
\years{2006-2008}Fellowship awarded by the Balseiro Foundation and the National Commission of Atomic Energy (FUNC-CNEA).\\
\years{2004-2005}Fellowship awarded by the National Commission of Atomic Energy (CNEA) to obtain a Master degree in Physics.\\
\years{2002-2004}Fellowship awarded by the National Commission of Atomic Energy (CNEA) to obtain a Master to study ``Licenciatura en Física'' at Instituto Balseiro.\\
\fi
\else
\section*{Premios, Reconocimientos, Becas y Subsidios}
\noindent
\years{2015} Premio ``Mejor Profesor Cátedra de la Facultad de Ciencias 2013-2014'' de la Universidad Industrial de Santander.\\
\years{2011} Premio ``Mejor Profesor del Instituto Balseiro 2011'' otorgado por la Fundación Balseiro.\\
\years{2022} Proyecto de Investigación ``Detectores de astropartículas y sus aplicaciones: muongrafía de grandes estructuras y meteorología espacial'', PICT2021-GRF-TII-00301, Estado: en evaluación.\\
\years{2022} Proyecto de Investigación ``Astroparticle simulations and its applications'', European Grid Infrastructure - Advanced Computing for EOSC (EGI-ACE) Use Case, Estado: en evaluación.\\
\years{2021} Proyecto de Investigación ``Detectores modulares para imágenes con Muones de fondo'', Fundación Hermanos Agustín y Enrique Rocca, Estado: en ejecución.\\
\years{2021} Proyecto de Investigación ``Utilización y desarrollo de ligandos específicos del microambiente tumoral acoplados a 177Lu para la detección y tratamiento de tumores primarios y metástasis'', Fundación Balseiro \& CNEA s/resol 306/21, Estado: en ejecución.\\
\years{2020} Proyecto de Investigación ``EOSC synergy – Building capacity, developing capability'', Horizon 2020 RI project 857647, Thematic Service Leadership, Estado: en ejecución.\\
\years{2020} Proyecto de Investigación ``Desarrollo de Técnicas de Muongrafía para Estudios Densitométricos de Objetos de Importancia Estratégica, II'' ASUTNBA0018565, Estado: en ejecución.\\
\years{2020} Proyecto de Investigación ``PlomBOX: un dispositivo de metrología de código abierto para combatir la contaminación por plomo en el agua potable mediante sensores biosintéticos'' GCRF Award R11178, Estado: en ejecución.\\
\years{2019} Proyecto de Investigación ``Desarrollo de Técnicas de Muongrafía para Estudios Densitométricos de Objetos de Importancia Estratégica'' ASUTNBA0005202, Estado: en ejecución\\
\years{2019} Proyecto de Investigación ``Muongrafía de grandes estructuras'' SIIP2019-C035, Estado: en ejecución.\\
\years{2018} Proyecto de Investigación ``Desarrollo de detectores de radiación'' PICT 2018-2886 (Argentina Innovadora 2020) Agencia, Estado: en ejecución.\\
\years{2017} Proyecto de Investigación ``Desarrollo de detectores de neutrones basados en efecto Cherenkov en agua'', SECYT 06/C4863 (UNCuyo, Argentina), Estado: aprobado.\\
\years{2016} Proyecto de Investigación ``Detectores de Astropartículas'', PICT 2015-2428 (Agencia-MinCyT, Argentina), Estado: aprobado.\\
\years{2010-presente} Docente categoría III (convocatoria 2015, previamente categoría V, convocatoria 2010) en el programa de incentivos a Docentes Investigadores SPU/ME.\\
\ifres
\else
\years{2015} Proyecto de Cooperation Project Nivel II (PCB-II) Argentina-Colombia, ``Aplicación de Técnicas de Muongrafía para el Estudio de Estructuras Volcánicas de Riesgo'', MinCyT-CONICET-COLCIENCIAS: aprobado.\\
\years{2014} Proyecto de investigación ``Detección de interacciones nucleares en CCD para la búsqueda de materia oscura'', PICT 2013-2128 (Agencia-MinCyT, Argentina), aprobado.\\
\years{2014} Proyecto de Articulación Docencia-Investigación-Extensión de la Universidad Industrial de Santander 2014, con la propuesta ``Introducción a la Física del Siglo XXI, la mejor manera de aprender Física es haciendo Física''. Rol: Director. Estado: aprobado.\\
\years{2014} Propuesta para proyecto de investigación de la Universidad Industrial de Santander 2014, con la propuesta ``GUANE3$^+$: Potenciación del Arreglo Guane de detectores de Astropartículas de la UIS mediante Técnicas de Detección por centelleo para estudios de Meteorología Espacial''. Rol: codirector. Estado: aprobado.\\
\years{2015} Proyecto de investigación aprobado en Convocatoria COLCIENCIAS 660/2014 ``MuTe: Telescopio de Muones para Muongrafía Volcánica''. Estado: aprobado.\\
\years{2014} Proyecto de Movilidad para el Apoyo a Proyectos con América Latina, convocatoria COLCIENCIAS 653/2014 para el Programa de intercambio Colombia-Argentina, con la propuesta: ``Factibilidad de Aplicación de Técnicas de Muongrafía para el Estudio de Erupciones Volcánicas''. Rol: Coinvestigador. Estado: Seleccionada Banco de Elegibles.\\
\years{2013} Proyecto de Investigación de la Fundación FRIDA con la propuesta: ''Generar una Experiencia Educativa bajo el Paradigma de la Ciencia que pueda ser Replicable para otras Organizaciones y sirva de base para un futuro MOOC''. Rol:Co-Investigador. Estado: aprobado.\\
\years{2013} Proyecto de Investigación de la Universidad Industrial de Santander 2013, con la propuesta ``El arreglo GUANE de detectores de astropartículas para estudios de Actividad Solar''. Rol: Co-director//Co-Investigador. Estado: aprobado.\\
\years{2014-2015} Beca posdoctoral otorgada por la Universidad Industrial de Santander, Bucaramanga, Colombia.
\years{2008-2010} Beca de posgrado tipo II (CONICET), para la Carrera de doctorado en Física en el Instituto Balseiro (UNC).\\
\years{2006-2008} Beca de posgrado tipo I (FUNC-CNEA), para la Carrera de doctorado en Física en el Instituto Balseiro (UNC).\\
\years{2004-2005} Beca de maestría (CNEA), para la carrera de Maestría en Ciencias Físicas en el Instituto Balseiro (UNC).\\
\years{2002-2004} Beca de grado (CNEA), para la carrera de Licenciatura en Física, en el Instituto Balseiro (UNC).\\
\fi
\fi