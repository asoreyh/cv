\ifeng
\section*{Research and Teaching Activities}

Since obtaining my Master's degree in 2005, I have participated in the following projects:

\subsection*{Medical Physics Department, Bariloche Atomic Center (CNEA)}
\years{2016--Present}
\begin{itemize}
\item Responsible investigator for CNEA and coordinator of the Simulations Group in the project `NEutrones Rápidos para la Explotación de Instalaciones con Dispositivos Atómicos (NEREIDA)` (2023--present).
\item Responsible investigator for CNEA and project manager of `PlomBOX, an open-source device for lead detection in water` (2019--2022).
\item Applications of astroparticle detection (I): development of simulations and detectors for spatial dose evaluation and reconstruction in clinical instances, high radiation exposure environments, and fast neutron production facilities.
\item Development of analysis techniques using artificial intelligence, curation, and anonymization in large volumes of data.
\item Head of the Medical Physics Department, under the Physics Management and the Non-Nuclear Research and Applications Management, Bariloche Atomic Center (CNEA).
Elected by peer researchers who make up the department (2017--2021).
The position includes responsibility for executing public funds as well as managing human capital.
During my tenure, the department created in 2016 was consolidated, by managing the incorporation or change of workplace for several researchers and students at all levels, while managing and executing funds for the purchase of equipment and supplies for an approximate total of more than 1.5 MUSD and national and international grants for more than 3 MUSD in total.
\end{itemize}

\subsection*{ITeDA, Constituyentes Atomic Center (CNEA)}
\years{2018--Present}

\begin{itemize}
\item Applications of astroparticle simulations: applications in muography, space weather, and design of new radiation detectors and shields.
\item Applications of astroparticle detection (II): muography of large artificial and natural structures of geophysical interest: volcanic risk assessment in Latin America, mining prospecting, and densitometry in dams and dykes.
\item Design, construction, and characterization of the directional muon flux measurement experiment for the ANDES underground laboratory. The constructed muon detector will be installed in an operating mine in the Province of San Juan, 330 m below sea level.
\end{itemize}

\subsection*{LAGO Project (Latin American Giant Observatory)}
{\small{\textit{See \href{http://lagoproject.net}{http://lagoproject.net}}}}
\years{2007--Present}
\begin{itemize}
\item Member of the LAGO Thematic Service in the Horizon-2020 EOSC-Synergy Project for the development and implementation of high-performance computing (HPC) and cloud environments for simulations, data analysis, and integration of the FAIR (\emph{Findable, Accessible, Interoperable and Reusable}) data paradigm.
\item Principal Investigator of the LAGO Project, period 2013--2016.
\item Design and implementation of the current organization of the LAGO Project.
\item Design and coordination of the space weather program of the LAGO Project.
\item Design, development, and implementation of the simulations and data analysis program for the detection of transient events (GRBs and Forbush events), background radiation, and atmospheric physics, using the \href{https://github.com/lagoproject/arti}{ARTI} package.
\item Design, development, and implementation of the \href{https://github.com/lagoproject/anna}{ANNA} data analysis package for the project.
\item Design, development, and implementation of the \href{https://github.com/lagoproject/acqua}{ACQUA} data acquisition package for the LAGO Project detectors.
\item Research, development, and construction of water Cherenkov detectors at the Universidad Industrial de Santander and the Centro Atómico Bariloche. One of them has been installed and is currently operating on the Antarctic Peninsula.
\item Design and development of the `Determination of the Muon Lifetime in Water` experiment for undergraduate and graduate students at universities where the LAGO Project operates.
\end{itemize}

\subsection*{ANDES Underground Laboratory}
{\small{\textit{See \href{http://www.andeslab.org}{www.andeslab.org}}}}
\years{2011--2022}
\begin{itemize}
\item Estimation of the expected background radiation at the ANDES Underground Laboratory due to natural radioactivity and high-energy atmospheric muon flux.
\item Design of the laboratory.
\item Design and construction of a detector for the directional measurement of the expected muon flux at ANDES. It will be installed in a mine operating 330 m below the surface.
\item Design of muon vetoes for the neutrino physics and dark matter search experiments to be installed in ANDES.
\end{itemize}

\subsection*{Pierre Auger Observatory}
{\small{\textit{See \href{http://www.auger.org/}{www.auger.org}}}}
\years{2006--2022}
\begin{itemize}
% \item Member of the Auger International Collaboration since 2006.
\item Cosmo-Geophysics Working Group Leader at the Pierre Auger Observatory (2014--2018).
\item Data analysis of the surface detector (SD) array of the Observatory.
\item Extended Atmospheric Showers Physics.
\item Development of the event reconstruction chain for events recorded by the SD detector.
\item Development and applications of low-energy modes (scaler` mode and histogram`mode) for the study of transient astrophysical events (GRBs and Forbush events) and on the short- and long-term modulation of the galactic cosmic ray flux due to solar activity.
\item Simulations of the detector and cosmic rays for determining the response of the water Cherenkov detectors in low-energy modes.
\item Data analysis of the atmospheric monitoring system of the Observatory.
\end{itemize}

\subsection*{Cherenkov Telescope Array (CTA) (2010--2014)}
{\small{\textit{See \href{http://www.cta-observatory.org}{www.cta-observatory.org}}}}
\begin{itemize}
\item Characterization of the proposed Argentine sites for the installation of the Observatory (San Antonio de los Cobres and Leoncito).
\item Research and development of an autonomous and remote station for the control and data acquisition of a meteorological station and a sky quality meter, installed in the town of San Antonio de los Cobres, Salta, Argentina.
\end{itemize}

\subsection*{Teaching (since 2009)}
\begin{itemize}
\ifres
\item Associate Professor\footnote{The teaching categories in Argentina are ordered as follows: Full Professor, Associate Professor, Assistant Professor, Senior Teaching Assistant, First and Second Assistant.} in undergraduate courses: Modern Physics A` (2015 and 2017), Physics IA`(2009--2012 and 2016), Physics IB`2009--2012), Physics IIB (Waves) `015), and currently Physics III B (Thermodynamics) (`nce 2018) and Physics IV B (Introduction to Particle Physics, Astrophysics, and Cosmology) (s`ce 2016); in the Physics Teaching program, Andean Campus, National University of Río Negro (UNRN); graduate courses Astroparticle Physics (20`--2021) and Particle and Radiation Detection Techniques (201`-2021) of the Double Doctorate in Astrophysics, National University of San Martín (UNSAM).
\else
\item \years{2015--present} Associate Professor, courses: Modern Physics A` (2015 and 2017) , Physics I A`(2016), Physics II B (Waves, 2015)`Physics III B (Thermodynamics, 2018 to present) `d Physics IV B (`troduction to Particle Physics, Astrophysics, and Cosmology, 2016 to present) in the Physics Teaching program, National University of Río Negro (UNRN) \item \years{2012--2020} Design and teaching of the courses The Physics of the LAGO Project,` Measurement of the Muon Lifetime,`and Astroparticle Simulations,`imed at advanced undergraduate and graduate students in Physics and Engineering, taught during the annual meetings of the LAGO Collaboration. These courses are still being taught by some of my former students in LAGO@.
\item \years{2018--2021} Associate Professor, courses: Astroparticle Physics` and Particle Detection Techniques`
in the Double Doctorate in Astrophysics, National University of San Martín (UNSAM)
\item \years{2015--2017} Senior Teaching Assistant in the subjects of Introduction to Particle Physics, Nuclear Physics, and Dosimetry` and Cosmic Ray Physics`(in charge) at the Balseiro Institute, National University of Cuyo (UNC).
\item \years{2014--2015} Visiting Professor in the courses Theoretical Mechanics` (graduate level) and Planetary Astronomy`at the School of Physics, Industrial University of Santander (UIS).
\item \years{2013--2014} Lecturer (equivalent to interim Assistant Professor) of the courses Introduction to Physics,` Introduction to Particle Physics,`and ``Theoretical Mechanics, for the Physics program
\item \years{2014--2015} Visiting Professor in the courses `Theoretical Mechanics` (graduate level) and `Planetary Astronomy` at the School of Physics, Industrial University of Santander (UIS).
\item \years{2013--2014} Lecturer Professor (equivalent to interim Associate Professor) of the courses `Introduction to Physics`, `Introduction to Particle Physics` and `Theoretical Mechanics` for the Physics major at the School of Physics, Industrial University of Santander (UIS).
\item \years{2014} Design and participation in the `Diploma in Astronomy, Astrophysics and Space Sciences` at the School of Physics, UIS (Starting in September 2014).
\item \years{2014} Design and teaching of the course `Astroclimate and the problem of Climate Change`, aimed at Teachers from Schools and High Schools, Industrial University of Santander, Bucaramanga, March 2014.
\item \years{2012} Head of Practical Works responsible for teaching Physics I A and Physics I B for the Teaching Program of Middle and Higher Education in Physics, National University of Rio Negro.
\item \years{2009--2011} Head of Practical Works for the courses Physics I A and Physics I B for the Teaching Program of Middle and Higher Education in Physics, National University of Rio Negro.
\item \years{2010--2012} First assistant in the course `Experimental III` at the Balseiro Institute, National University of Cuyo (UNC), responsible for the low-energy cosmic ray physics experiment and the muon lifetime measurement, designed by me and using the Nahuelito detector from the LAGO project.
\item \years{2010--2012} First assistant in the courses `Introduction to Nuclear Physics and Particle Physics` at the Balseiro Institute, National University of Cuyo (UNC).
\fi
\end{itemize}
\else

\section*{Actividades de Investigación y Docencia}

Desde que obtuve mi Maestría en 2005, he participado en los siguientes proyectos:

\subsection*{Departamento Física Médica, Centro Atómico Bariloche (CNEA)}
\years{2016--Presente}
\begin{itemize}
	\item Investigador responsable por CNEA y coordinador del Grupo de Simulaciones del proyecto \lq\lq{}NEutrones Rápidos para la Explotación de Instalaciones con Dispositivos Atómicos (NEREIDA)\rq\rq{} (2023--presente).
	\item Investigador responsable por CNEA y gerente del proyecto \lq\lq{}PlomBOX, un dispositivo de código abierto para la detección de plomo en agua\rq\rq{} (2019--2022).
	\item Aplicaciones de la detección de astropartículas (I): desarrollos de simulaciones y detectores para evaluación y reconstrucción espacial de dosis en instancias clínicas, en ambientes de alta exposición a la radiación y en instalaciones de producción de neutrones rápidos.
	\item Desarrollo de técnicas de análisis mediante inteligencia artificial, curaduría y anonimización en grandes volúmenes de datos.
	\item Jefe del Departamento Física Médica, dependiente de la Gerencia de Física, Gerencia de Investigación y Aplicaciones No Nucleares, Centro Atómico Bariloche (CNEA).
 	Elegido por pares investigadores que constituyen el departamento (2017--2021).
 	El cargo incluye la responsabilidad de ejecución de fondos públicos así como la gestión del capital humano.
 	Durante mi jefatura se consolidó el departamento creado en 2016, mediante mi gestión para la incorporación o cambio de lugar de trabajo de varios investigadores e investigadoras y estudiantes en todos los niveles, a la vez que se gestionaron y ejecutaron fondos para la compra de equipamientos e insumos por un total aproximado de más de 1.5 MUSD y subsidios nacionales e internacionales por más de 3 MUSD en total.
\end{itemize}

\subsection*{ITeDA, Centro Atómico Constituyentes (CNEA)}
\years{2018--Presente}
\begin{itemize}
	\item Aplicaciones de simulaciones de astropartículas: aplicaciones en muongrafía, meteorología del espacio y diseño de nuevos detectores y blindajes de radiación.
	\item Aplicaciones de la detección de astropartículas (II): muongrafía de grandes estructuras artificiales y naturales de interés geofísico: evaluación del riesgo volcánico en América Latina, prospección minera, y densitometría en represas y diques.
	\item Diseño, construcción y caracterización del experimento de medición de flujo direccional de muones para el laboratorio subterráneo ANDES. El detector de muones construido será instalado en una mina en operación en la Provincia de San Juan a 330 m bajo el nivel del mar.
\end{itemize}

\subsection*{Proyecto LAGO (Latin American Giant Observatory)}
{\small{\textit{Ver \href{http://lagoproject.net}{http://lagoproject.net}}}}
\years{2007--Presente}
\begin{itemize}
\item Integrante del Servicio Temático LAGO en el Proyecto Horizon-2020 EOSC-Synergy para el desarrollo e implementación en entornos de computación de alto rendimiento (HPC) y en la nube (cloud) de simulaciones, análisis de datos e integración del paradigma FAIR (\emph{Findable, Accessible, Interoperable and Reusable}) de datos.
\item Investigador Principal del Proyecto LAGO, período 2013--2016
\item Diseño y puesta en ejecución de la organización actual del Proyecto LAGO
\item Diseño y coordinación del programa de meteorología espacial del Proyecto LAGO
\item Diseño, desarrollo e implementación del programa de simulaciones y análisis de datos para la detección de eventos transitorios (GRBs y eventos Forbush), radiación de fondo y física de la atmósfera, mediante el paquete \href{https://github.com/lagoproject/arti}{ARTI}.
\item Diseño, desarrollo e implementación del paquete \href{https://github.com/lagoproject/anna}{ANNA} de análisis de datos del proyecto.
\item Diseño, desarrollo e implementación del paquete \href{https://github.com/lagoproject/acqua}{ACQUA} de adquisición de datos de los detectores del proyecto LAGO.
\item Investigación, desarrollo y construcción de detectores tipo Cherenkov en agua en el la Universidad Industrial de Santander y en el Centro Atómico Bariloche.
Uno de ellos ha sido instalado y actualmente está operando en la Península Antártica.
\item Diseño y desarrollo del experimento ``Determinación de la Vida Media del Muón en Agua para estudiantes de grado y posgrado de las universidades donde opera el proyecto LAGO\@.
\end{itemize}

\subsection*{Laboratorio Subterráneo ANDES}
{\small{\textit{Ver \href{http://www.andeslab.org}{www.andeslab.org}}}}
\years{2011--2022}
\begin{itemize}
\item Estimación del fondo de radiación esperado en el laboratorio subterráneo ANDES debido a la radiactividad natural y al flujo de muones atmosféricos de alta energía.
\item Diseño del laboratorio.
\item Diseño y construcción de un detector para la medición direccional del flujo de muones esperado en ANDES. Será instalado en una mina en operación a 330 m bajo la superficie.
\item Diseño de vetos de muones para los experimentos de física de neutrinos y búsqueda de materia oscura que serán instalados en ANDES\@.
\end{itemize}

\subsection*{Observatorio Pierre Auger}
{\small{\textit{Ver \href{http://www.auger.org/}{www.auger.org}}}}
\years{2006--2022}
\begin{itemize}
% \item Miembro de la Colaboración Internacional Auger desde el año 2006.
\item Líder de Grupo de Trabajo ``Cosmo-Geophysics del Observatorio Pierre Auger (2014--2018)
\item Análisis de datos del arreglo de detectores de superficie (SD) del Observatorio.
\item Física de Lluvias Atmosféricas Extendidas
\item Desarrollo de la cadena de reconstrucción de eventos registrados por el detector SD\@.
\item Desarrollo y aplicaciones de los modos de bajas energías (modo ``scaler y modo ``histograma) para el estudio de eventos astrofísicos transitorios (GRBs y eventos Forbush), y sobre la modulación a corto y largo plazo del flujo de rayos cósmicos galácticos debida a la actividad solar.
\item Simulaciones del detector y de rayos cósmicos para la determinación de la respuesta de los detectores water-Cherenkov en los modos de baja energía.
\item Análisis de datos del sistema de monitoreo atmosférico del Observatorio.
\end{itemize}
\subsection*{Cherenkov Telescope Array (CTA) (2010--2014)}
{\small{\textit{Ver \href{http://www.cta-observatory.org}{www.cta-observatory.org}}}}
\begin{itemize}
\item Caracterización de los sitios Argentinos propuestos para la instalación del Observatorio (San Antonio de los Cobres y Leoncito). %, propuesto por la colaboración Argentina como candidato para la instalación del proyecto CTA.
\item Investigación y desarrollo de una estación autónoma y remota para el control y la adquisición de datos de una estación meteorológica y un medidor de calidad del cielo, instalados en la localidad de San Antonio de los Cobres, Salta, Argentina.
\end{itemize}

\subsection*{Docencia (desde 2009)}
\begin{itemize}
\ifres
	\item Profesor categoría Asociado\footnote{Las categorías docentes en Argentina se ordenan de la siguiente forma: Profesor Titular, Profesor Asociado, Profesor Adjunto, Jefe de Trabajos Prácticos, Auxiliar de Primera y Auxiliar de Segunda.} en los cursos de grado: ``Física Moderna A (2015 y 2017), ``Física IA (2009--2012 y 2016), ``Física IB (2009--2012), ``Física IIB (Ondas) (2015), y actualmente ``Física III B (Termodinámica) (desde 2018) y ``Física IV B (Introducción a Física de Partículas, Astrofísica y Cosmología) (desde 2016); del Profesorado de Nivel Medio y Superior en Física, Sede Andina, Universidad Nacional de Río Negro (UNRN); cursos de posgrado ``Física de Astropartículas (2018--2021) y ``Técnicas en detección de partículas y radiación (2018--2021) de la Carrera del Doble Doctorado en Astrofísica, Universidad Nacional de San Martín (UNSAM).
\else
	\item \years{2015--presente} Profesor Asociado, cursos de: ``Física Moderna A (2015 y 2017) , ``Física I A (2016), ``Física II B (Ondas, 2015), ``Física III B (Termodinámica, 2018 hasta el presente) y ``Física IV B (Introducción a Física de Partículas, Astrofísica y Cosmología, 2016 hasta el presente) del Profesorado de Nivel Medio y Superior de Física, Universidad Nacional de Río Negro (UNRN)
	\item \years{2012--2020} Diseño y Dictado de los cursos ``La Física del Proyecto LAGO, ``Medición de la Vida Media del Muón y ``Simulaciones de Astropartículas, dirigidos a estudiantes avanzados de grado y posgrado en Física e Ingeniería, dictados durante los encuentros anuales de la Colaboración LAGO. Estos cursos aún continúan siendo dictados por algunos de mis anteriores estudiantes en LAGO\@.
	\item \years{2018--2021} Profesor Asociado, cursos de: ``Física de Astropartículas y ``Técnicas en detección de partículas;
	de la Carrera del Doble Doctorado en Astrofísica, Universidad Nacional de San Martín (UNSAM)
	\item \years{2015--2017} Jefe de Trabajos Prácticos en la cátedra de ``Introducción a Física de Partículas, Nuclear y Dosimetría y ``Física de Rayos Cósmicos (a cargo) del Instituto Balseiro, Universidad Nacional de Cuyo (UNC).
	\item \years{2014--2015} Profesor Invitado en los cursos ``Mecánica Teórica (posgrado) y ``Astronomía Planetaria de la Escuela de Física, Universidad Industrial de Santander (UIS).
	\item \years{2013--2014} Profesor Cátedra (equivalente Profesor Adjunto interino) de los cursos ``Introducción a la Física,``Introducción a Física de Partículas y ``Mecánica Teórica, para la Carrera de Física, Escuela de Física, Universidad Industrial de Santander (UIS).
    \item \years{2014} Diseño y participación en el ``Diplomado en Astronomía, Astrofísica y Ciencias Espaciales de la Escuela de Física de la UIS (Inicio Setiembre 2014).
	\item \years{2014} Diseño y dictado del curso ``Astroclima y la problemática del Cambio Climático, orientado a Profesores de Escuelas y Bachilleratos, Universidad Industrial de Santander, Bucaramanga, Marzo de 2014
	\item \years{2012} Jefe de Trabajos Prácticos a cargo del dictado de las materias Física I A y Física I B del Profesorado de Nivel Medio y Superior en Física, Universidad Nacional de Río Negro
	\item \years{2009--2011} Jefe de Trabajos Prácticos de las cátedras Física I A y Física I B del Profesorado de Nivel Medio y Superior en Física, Universidad Nacional de Río Negro
	\item \years{2010--2012} Auxiliar de primera en la cátedra de ``Experimental III del Instituto Balseiro, Universidad Nacional de Cuyo (UNC), a cargo del experimento de física de rayos cósmicos de baja energía y medición de la vida media del muón, diseñados por mí y utilizando el detector Nahuelito del proyecto LAGO\@.
	\item \years{2010--2012} Auxiliar de primera en las cátedras ``Introducción a Física Nuclear y Física de Partículas del Instituto Balseiro, Universidad Nacional de Cuyo (UNC).
\fi
\end{itemize}
\fi