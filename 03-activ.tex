\ifeng
\section*{Research and Teaching Activities}

Since I have earned my master degree in December 2005, I have been involved in the following projects:

\subsection*{Medical Physics Department, CAB,(2016-Present)}
\begin{itemize}
	\item Project manager of the PlomBOX project, an open device to measure lead contamination in tap water
	\item Astroparticle detection applications (I): development of
		simulations and detectors for the calculation and measurement
		of spatial dose distribution in clinical and high-level
		dose environments
 	\item Development of new artificial-intelligence-based big data analysis, big data curation, big data anonymization and medical imaging analysis and processing.
	\item Head of the Medical Physics Department (GF-GAIYANN-CNEA).
	Elected by the members of the Department (2017-2021).
\end{itemize}

\subsection*{ITeDA, CAC,(2018-Present)}
\begin{itemize}
	\item Astroparticle detection applications (III): muography of big artificial and geological buildings: applications
	to volcanic risk assessment, mining prospecting and dams densitometry
	\item Astroparticle simulations applications: application in muography, space weather and new radiation detectors
	and shielding designs.
\end{itemize}

\subsection*{Latin American Giant Observatory (LAGO) (2007-Present)}
{\small{\textit{See \href{http://lagoproject.net}{lagoproject.net}}}}
\begin{itemize}
\item Member of the LAGO Thematic Service at the Horizon 2020 EOSC-Synergy project.
\item Principal Investigator, 2013--2016
\item Design and execution of the project new organization
\item Design and coordination of the LAGO Space Weather program
\item Simulations and data analysis for the detection of transient events
(GRB and Forbush events), background radiation and atmospheric physics.
\item Research, development and building of water-Cherenkov detectors for the LAGO project at Universidad Industrial de Santander and Centro Atómico Bariloche.
	One of them is currently installed and is operating at the Antarctic Peninsula.
\item Design and coordination of the experiment ``Measurement of Muon Lifetime in Water'', done by undergraduate students at Instituto Balseiro.
\end{itemize}

\subsection*{ANDES Underground Laboratory (2010-2013, 2015--2016, 2018-present)}
{\small{\textit{See \href{http://www.andeslab.org}{www.andeslab.org}}}}
\begin{itemize}
\item Estimation and measurements of the expected backgrounds at the ANDES
underground lab due to natural radioactivity and high energy atmospheric muons.
\item Laboratory design.
\item Muon veto for the ANDES experiments design
\end{itemize}

\ifres
\else
\subsection*{Pierre Auger Observatory (2006-Present)}
{\small{\textit{See \href{http://www.auger.org/}{www.auger.org}}}}
\begin{itemize}
\item Task leader of the ``Cosmo-Geophysics'' task of the Pierre Auger Observatory, 2014--2018
\item Data analysis of the Surface Detector
\item Extensive Atmospheric Shower Physics
\item Development of the reconstruction event chain of the Surface Detector
\item Development and applications of the low energy modes (scaler and histogram
modes) of the surface detectors of the Pierre Auger Observatory, for the study
of transient events (Gamma Ray Bursts and Forbush events), and short and long
term modulation of the galactic cosmic rays flux due to solar activity
\item CORSIKA and detector simulations, oriented to determine the
water-Cherenkov response working in the low energy modes
\item Data analysis of the weather monitoring system of the Pierre Auger
Observatory
\end{itemize}
\subsection*{Cherenkov Telescope Array (CTA) (2010-2014)}
{\small{\textit{See \href{http://www.cta-observatory.org}{www.cta-observatory.org}}}}
\begin{itemize}
\item San Antonio de los Cobres site characterization
\item Research and development of the autonomous station for control and data
acquisition of the weather station and sky quality meter installed in San
Antonio de los Cobres, Argentina, one of the site candidates for the CTA
observatory.
\end{itemize}
\fi

\subsection*{Teaching (2009-Present)}
\begin{description}
	\item[2015-present] Associated Professor, Thermodynamics, Cosmology and Astrophysics, Modern Physics A and Wave Physics, Profesorado de Nivel Medio y Superior en Física, Sede Andina, Universidad Nacional de Río Negro (UNRN)
	\item [2012-2020] Lecturer of the ``La Física del Proyecto LAGO'', ``Medición de la Vida Media del Muón'' y ``Simulaciones de Astropartículas'' physics courses for graduate and postgraduate physics students.
	These courses were dictated during the annual meetings of the LAGO collaboration, and are still being dictated by some of my former students at LAGO\@.
	\item[2017-2021] Associated Professor, Astroparticle physics, Particle detection techniques, Double Doctorate in Astrophysics program, Universidad Nacional de San Martín (UNSAM)
	\item[2016-2020] Member of the Academic Committee of the Master in Medical Physics program of the Instituto Balseiro, Universidad Nacional de Cuyo (UNC).
	\item[2015-2017] Senior Teaching assistant (Jefe de Trabajos Prácticos), ``Introduction to nuclear, particle physics and dosimetry'' and ``Cosmic Rays Physics'' (lecturer) courses, Instituto Balseiro, Universidad Nacional de Cuyo (UNC)
	\item[2014-2015] Professor, Classical Mechanics (Graduate) and General Astronomy, School of Physics, UIS\@.
	\item[2013-2014] Professor, Introductory Physics course and Introductory Particle Physics course, UIS\@.
	\item[2014] Design and lecture of the course ``Astro-meteorology and Climate Change'', intended for High Schools teachers, UIS, March 2014.
	\item[2013] Professor, Advanced Mathematical Methods for Physics course, UIS\@.
	\item[2009-2012] Senior teaching assistant (Jefe de Trabajos Prácticos), Physics I A \& B (introductory physics) course, UNRN\@.
	\item[2010-2012] Teaching assistant, Experimental Physics III and Introduction to nuclear and particle physics courses, Instituto Balseiro, Universidad Nacional de Cuyo (UNC)
\end{description}
\else
\section*{Actividades de Investigación y Docencia}

Desde que obtuve mi Maestría en 2005, he participado en los siguientes proyectos:

\subsection*{Departamento Física Médica, CAB,(2016-Presente)}
\begin{itemize}
	\item Investigador responsable por CNEA y coordinador del Grupo de Simulaciones del proyecto \lq\lq{}NEutrones Rápidos para la Explotación de Instalaciones con Dispositivos Atómicos (NEREIDA)\rq\rq{} (2023-presente).
	\item Investigador responsable por CNEA y gerente del proyecto \lq\lq{}PlomBOX, un dispositivo de código abierto para la detección de plomo en agua\rq\rq{} (2019-2022).
	\item Aplicaciones de la detección de astropartículas (I): desarrollos de simulaciones y detectores para evaluación y reconstrucción espacial de dosis en instancias clínicas, en ambientes de alta exposición a la radiación y en instalaciones de producción de neutrones rápidos.
	\item Desarrollo de técnicas de análisis mediante inteligencia artificial, curaduría y anonimización en grandes volúmenes de datos.
 	\item Jefe del Departamento Física Médica, dependiente de la Gerencia de Física, Gerencia de Investigación y Aplicaciones No Nucleares, Centro Atómico Bariloche (CNEA).
 	Elegido por pares investigadores que constituyen el departamento (2017-2021).
 	El cargo incluye la responsabilidad de ejecución de fondos públicos así como la gestión del capital humano.
 	Durante mi jefatura se consolidó el departamento creado en 2016, mediante mi gestión para la incorporación o cambio de lugar de trabajo de varios investigadores e investigadoras y estudiantes en todos los niveles, a la vez que se gestionaron y ejecutaron fondos para la compra de equipamientos e insumos por un total aproximado de más de 1.5 MUSD y subsidios nacionales e internacionales por más de 3 MUSD en total.
\end{itemize}

\subsection*{ITeDA, CAC,(2018-Presente)}
\begin{itemize}
	\item Aplicaciones de simulaciones de astropartículas: aplicaciones en muongrafía, meteorología del espacio y diseño de nuevos detectores y blindajes de radiación.
	\item Aplicaciones de la detección de astropartículas (II): muongrafía de grandes estructuras artificiales y naturales de interés geofísico: evaluación del riesgo volcánico en América Latina, prospección minera, y densitometría en represas y diques.
	\item Diseño, construcción y caracterización del experimento de medición de flujo direccional de muones para el laboratorio subterráneo ANDES. El detector de muones construido será instalado en una mina en operación en la Provincia de San Juan a 330 m bajo el nivel del mar.
\end{itemize}

\subsection*{Proyecto LAGO (Latin American Giant Observatory) (2007-Presente)}
{\small{\textit{Ver \href{http://lagoproject.net}{http://lagoproject.net}}}}
\begin{itemize}

\item Integrante del Servicio Temático LAGO en el Proyecto Horizon-2020 EOSC-Synergy para el desarrollo e implementación en entornos de computación de alto rendimiento (HPC) y en la nube (cloud) de simulaciones, análisis de datos e integración del paradigma FAIR (\emph{Findable, Accessible, Interoperable and Reusable}) de datos.
\item Investigador Principal del Proyecto LAGO, período 2013--2016
\item Diseño y puesta en ejecución de la organización actual del Proyecto LAGO
\item Diseño y coordinación del programa de meteorología espacial del Proyecto LAGO
\item Diseño, desarrollo e implementación del programa de simulaciones y análisis de datos para la detección de eventos transitorios (GRBs y eventos Forbush), radiación de fondo y física de la atmósfera, mediante el paquete ARTI (\href{https://github.com/lagoproject/arti}{https://github.com/lagoproject/arti}).
\item Diseño, desarrollo e implementación de los códigos de análisis de datos del proyecto (ANNA, \href{https://github.com/lagoproject/anna}{https://github.com/lagoproject/anna}).
\item Diseño, desarrollo e implementación de los códigos de adquisición de datos de los detectores del proyecto LAGO (ACQUA, \href{https://github.com/lagoproject/acqua}{https://github.com/lagoproject/acqua}).
\item Investigación, desarrollo y construcción de detectores tipo Cherenkov en agua en el la Universidad Industrial de Santander y en el Centro Atómico Bariloche.
Uno de ellos ha sido instalado y actualmente está operando en la Península Antártica.
\item Diseño y desarrollo del experimento ``Determinación de la Vida Media del Muón en Agua'' para estudiantes de grado y posgrado de las universidades donde opera el proyecto LAGO\@.
\end{itemize}
\subsection*{Laboratorio Subterráneo ANDES (2011-2013, 2015--2016, 2018--2022)}
{\small{\textit{Ver \href{http://www.andeslab.org}{www.andeslab.org}}}}
\begin{itemize}
\item Estimación del fondo de radiación esperado en el laboratorio subterráneo ANDES debido a la radiactividad natural y al flujo de muones atmosféricos de alta energía.
\item Diseño del laboratorio.
\item Diseño y construcción de un detector para la medición direccional del flujo de muones esperado en ANDES. Será instalado en una mina en operación a 330 m bajo la superficie.
\item Diseño de vetos de muones para los experimentos de física de neutrinos y búsqueda de materia oscura que serán instalados en ANDES\@.
\end{itemize}

\subsection*{Observatorio Pierre Auger (2006-2022)}
{\small{\textit{Ver \href{http://www.auger.org/}{www.auger.org}}}}
\begin{itemize}
% \item Miembro de la Colaboración Internacional Auger desde el año 2006.
\item Líder de Grupo de Trabajo ``Cosmo-Geophysics'' del Observatorio Pierre Auger (2014--2018)
\item Análisis de datos del arreglo de detectores de superficie (SD) del Observatorio.
\item Física de Lluvias Atmosféricas Extendidas
\item Desarrollo de la cadena de reconstrucción de eventos registrados por el detector SD\@.
\item Desarrollo y aplicaciones de los modos de bajas energías (modo ``scaler'' y modo ``histograma'') para el estudio de eventos astrofísicos transitorios (GRBs y eventos Forbush), y sobre la modulación a corto y largo plazo del flujo de rayos cósmicos galácticos debida a la actividad solar.
\item Simulaciones del detector y de rayos cósmicos para la determinación de la respuesta de los detectores water-Cherenkov en los modos de baja energía.
\item Análisis de datos del sistema de monitoreo atmosférico del Observatorio.
\end{itemize}
\subsection*{Cherenkov Telescope Array (CTA) (2010-2014)}
{\small{\textit{Ver \href{http://www.cta-observatory.org}{www.cta-observatory.org}}}}
\begin{itemize}
\item Caracterización de los sitios Argentinos propuestos para la instalación del Observatorio (San Antonio de los Cobres y Leoncito). %, propuesto por la colaboración Argentina como candidato para la instalación del proyecto CTA.
\item Investigación y desarrollo de una estación autónoma y remota para el control y la adquisición de datos de una estación meteorológica y un medidor de calidad del cielo, instalados en la localidad de San Antonio de los Cobres, Salta, Argentina.
\end{itemize}

\subsection*{Docencia (2009-Presente)}
\begin{description}
\ifres
	\item [Profesor] categoría Asociado\footnote{Las categorías docentes en Argentina se ordenan de la siguiente forma: Profesor Titular, Profesor Asociado, Profesor Adjunto, Jefe de Trabajos Prácticos, Auxiliar de Primera y Auxiliar de Segunda.} en los cursos de grado: ``Física Moderna A'' (2015 y 2017), ``Física IA'' (2009-2012 y 2016), ``Física IB'' (2009-2012), ``Física IIB (Ondas)'' (2015), y actualmente ``Física III B (Termodinámica)'' (desde 2018) y ``Física IV B (Introducción a Física de Partículas, Astrofísica y Cosmología)'' (desde 2016); del Profesorado de Nivel Medio y Superior en Física, Sede Andina, Universidad Nacional de Río Negro (UNRN); cursos de posgrado ``Física de Astropartículas'' (2018-2021) y ``Técnicas en detección de partículas y radiación'' (2018-2021) de la Carrera del Doble Doctorado en Astrofísica, Universidad Nacional de San Martín (UNSAM).
\else
	\item [2015-presente] Profesor Asociado, cursos de: ``Física Moderna A'' (2015 y 2017) , ``Física I A'' (2016), ``Física II B (Ondas, 2015)'', ``Física III B (Termodinámica, 2018 hasta el presente)'' y ``Física IV B'' (Introducción a Física de Partículas, Astrofísica y Cosmología, 2016 hasta el presente) del Profesorado de Nivel Medio y Superior de Física, Universidad Nacional de Río Negro (UNRN)
	\item [2012-2020] Diseño y Dictado de los cursos ``La Física del Proyecto LAGO'', ``Medición de la Vida Media del Muón'' y ``Simulaciones de Astropartículas'', dirigidos a estudiantes avanzados de grado y posgrado en Física e Ingeniería, dictados durante los encuentros anuales de la Colaboración LAGO. Estos cursos aún continúan siendo dictados por algunos de mis anteriores estudiantes en LAGO\@.
	\item [2018-2021] Profesor Asociado, cursos de: ``Física de Astropartículas'' y ``Técnicas en detección de partículas'';
	de la Carrera del Doble Doctorado en Astrofísica, Universidad Nacional de San Martín (UNSAM)
 	\item [2015-2017] Jefe de Trabajos Prácticos en la cátedra de ``Introducción a Física de Partículas, Nuclear y Dosimetría'' y ``Física de Rayos Cósmicos'' (a cargo) del Instituto Balseiro, Universidad Nacional de Cuyo (UNC).
	\item [2014-2015] Profesor Invitado en los cursos ``Mecánica Teórica'' (posgrado) y ``Astronomía Planetaria'' de la Escuela de Física, Universidad Industrial de Santander (UIS).
	\item [2013-2014] Profesor Cátedra (equivalente Profesor Adjunto interino) de los cursos ``Introducción a la Física'',``Introducción a Física de Partículas'' y ``Mecánica Teórica'', para la Carrera de Física, Escuela de Física, Universidad Industrial de Santander (UIS).
    \item [2014] Diseño y participación en el ``Diplomado en Astronomía, Astrofísica y Ciencias Espaciales'' de la Escuela de Física de la UIS (Inicio Setiembre 2014).
	\item [2014] Diseño y dictado del curso ``Astroclima y la problemática del Cambio Climático'', orientado a Profesores de Escuelas y Bachilleratos, Universidad Industrial de Santander, Bucaramanga, Marzo de 2014
	\item [2012] Jefe de Trabajos Prácticos a cargo del dictado de las materias Física I A y Física I B del Profesorado de Nivel Medio y Superior en Física, Universidad Nacional de Río Negro
	\item [2009-2011] Jefe de Trabajos Prácticos de las cátedras Física I A y Física I B del Profesorado de Nivel Medio y Superior en Física, Universidad Nacional de Río Negro
	\item [2010-2012] Auxiliar de primera en la cátedra de ``Experimental III'' del Instituto Balseiro, Universidad Nacional de Cuyo (UNC), a cargo del experimento de física de rayos cósmicos de baja energía y medición de la vida media del muón, diseñados por mí y utilizando el detector Nahuelito del proyecto LAGO\@.
	\item [2010-2012] Auxiliar de primera en las cátedras ``Introducción a Física Nuclear y Física de Partículas'' del Instituto Balseiro, Universidad Nacional de Cuyo (UNC).
\fi
\end{description}
\fi