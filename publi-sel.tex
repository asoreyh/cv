\ifeng
\subsection*{Selected Works}
\noindent
During the development of my work within the Pierre Auger Observatory, I have
been acting as the Physics coordinator and responsible of one of the full
author list papers of the Pierre Auger Collaboration (The Pierre Auger
Collaboration, JINST {\bf 6} P01003--P01020 (2011)), using the surface detector
in a novel way, developed during my Ph.D. thesis, as a tool to study transient
solar phenomena and heliospheric modulation of galactic cosmic rays flux.

This list is a personal selection of the works I have been directly involved:

\else
\subsection*{Trabajos seleccionados}
\noindent
Durante el desarrollo de mi trabajo en el marco del Observatorio Pierre Auger, tuve la oportunidad de actuar como Coordinador de Física y responsable directo de la publicación de uno de los trabajos publicados por toda la Colaboración Auger (The Pierre Auger Collaboration, JINST {\bf 6} P01003--P01020 (2011)), que consiste en el uso del arreglo de detectores de supericie para el estudio de fenómenos heliosféricos mediante la modulación del flujo de rayos cósmicos galácticos. Este uso novedoso del arreglo SD fue desarrollado como parte de mi tesis de doctorado en Física.

La lista mostrada a continuación corresponde a una selección personal de los trabajos publicados en los cuales estuve directamente involucrado:

\fi

\noindent
\begin{etaremune}
\item \years{2012}H. Asorey and A. López Dávalos, {\emph{Fermi Problem: Power
developed at the eruption of the Puyehue-Cordón Caulle volcanic system in June
2011}}, Amer. Jour. Phys., {\emph{submitted}}, (2012).
\href{http://arxiv.org/abs/1109.1165}{arXiv:1109.1165v1}[physics.ed-ph]. 
\ifeng
Selected as the best \href{http://arxiv.org}{arXiv} paper of September 2011 by the
\else
Seleccionado como el mejor trabajo enviado al \href{http://arxiv.org}{arXiv} durante Setiembre del 2011 por el blog 
\fi
\href{http://www.technologyreview.com/blog/arxiv/27140/}{M.I.T. Technology
Review Physics arXiv Blog}, Sep. 2011.

\item \years{2012}S. Dasso and H. Asorey, for the Pierre Auger Collaboration,
\href{http://dx.doi.org/10.1016/j.asr.2011.12.028}{\emph{ The scaler mode in
the Pierre Auger Observatory to study heliospheric modulation of cosmic rays
}}, Adv. Space Res. {\bf{49}} (11), 1563--1569 (2012)

\item \years{2012} H. Asorey, M. Arribere, X. Bertou, M. Gómez Berisso, F. Sánchez,
{\emph{Expected Backgrounds at the ANDES Underground Laboratory}}
\ifeng
plenary talk given at the
\else 
charla plenaria dada en el
\fi
Third International Workshop for the Design of the ANDES Underground Laboratory, Valparaiso, Chile, 11--12 Jan 2012.

\item \years{2011}The Pierre Auger Collaboration,
\href{http://dx.doi.org/10.1088/1748-0221/6/01/P01003}{\emph{The Pierre Auger
Observatory Scaler Mode for the Study of the Modulation of Galactic Cosmic Rays
due to Solar Activity}}, JINST {\bf 6} P01003--
P01020 (2011).
\ifeng $^*${\bf{Coordinator}} \else $^*${\bf{Coordinador}} \fi

\item \years{2011}I. Allekotte, H. Arnaldi, H. Asorey, X. Bertou, M. Gómez Berisso,
M. Sofo Haro, {\emph{Development of ultra-fast and ultra low power consumption
electronics in the Bariloche Particle and Radiation Detection Laboratory}},
\ifeng
poster presentation in the 96$^{\mathrm{th}}$ National Reunion SUF-AFA2011 of the Argentinian Physics Association, Montevideo, Uruguay, 20--23 Sept 2011.
\else
poster presentado en la 96$^{\mathrm{th}}$ Reunión Nacional SUF-AFA2011 de la Asociación Argentina de Física, Montevideo, Uruguay, 20--23 Sept 2011.
\fi

\item \years{2011}H. Asorey[Pierre Auger Collaboration], {\emph{Low energy radiation
measurements with the water Cherenkov detector array of the Pierre Auger
Observatory}}, \en Proc. 32 International Cosmic Ray Conference, vol. 11
462--465, Beijing, China, 11--18 Ago 2011

\item \years{2010}J. Bl\"umer and The Pierre Auger Collaboration,
\href{http://dx.doi.org/10.1088/1367-2630/12/3/035001}{\emph{The Northern Site
of the Pierre Auger Observatory}}, Journal of Physics {\bf 12} (3) 035001

\item \years{2010}The Pierre Auger Collaboration,
\href{http://dx.doi.org/10.1016/j.physletb.2010.02.013}{\emph{Measurement of
the energy spectrum of cosmic rays above $10^{18}$ eV using the Pierre Auger
Observatory}}, Phys. Lett. {\bf B685} 239--246 (2010),\\
\href{http://arxiv.org/abs/1002.1975}{arXiv:1002.1975v1}[astro-ph.HE]

\item \years{2010}The Pierre Auger Collaboration,
\href{http://dx.doi.org/10.1016/j.nima.2009.11.018}{\emph{Trigger and Aperture
of the Surface Detector Array of the Pierre Auger Observatory}}, NIM {\bf A613}
29--39, (2010)

\item \years{2010}H. Asorey[LAGO Collaboration], {\emph{The Large Aperture Gamma Ray
Burst Observatory (LAGO)}}, plenary talk in the 3$^{\mathrm{rd}}$ International Workshop of
High Energy Physics in the LHC Era HEP2010, Valparaiso, Chile, 4--8 Jan 2010.

\item \years{2009}H. Asorey[Pierre Auger Collaboration], {\emph{Cosmic Ray Solar
Modulation Studies at the Pierre Auger Observatory}}, \en Proc. 31th
International Cosmic Ray Conference, Lodz, Poland, 8--15 Jul 2009.

\item \years{2009} The Pierre Auger Collaboration,
\href{http://dx.doi.org/10.1016/j.astropartphys.2009.06.004}{\emph{Atmospheric
effects on extensive air showers observed with the Surface Detector of the
Pierre Auger Observatory}}, Astropart. Phys. {\bf 32}, 89--99, (2009),
\href{http://arxiv.org/abs/0906.5497/}{arXiv:0906.5497v2}[astro-ph.IM]

\item \years{2008}D. Allard {\emph et al.} [LAGO Collaboration],
\href{http://dx.doi.org/10.1016/j.nima.2008.07.041}{\emph{Use of
water-Cherenkov detectors to detect Gamma Ray Bursts at the Large Aperture GRB
Observatory (LAGO)}}, NIM {\bf A595} 70--72 (2008)

\item \years{2007}D. Allard {\emph et al.} [LAGO Collaboration], {\emph{Looking for
the high energy component of GRBs at the Large Aperture GRB Observatory}}, \en
Proc. 30$^{\mathrm{th}}$ International Cosmic Ray Conference,  Mérida, Mexico, 3-11 Jul
2007.

\item \years{2007}The Pierre Auger Collaboration,
\href{http://dx.doi.org/10.1016/j.astropartphys.2006.11.002}{\emph{Anisotropy
studies around the galactic centre at EeV energies with the Auger
Observatory.}},  Astropart. Phys. {\bf 27} 244--253 (2007)

\item \years{2006}D. Allard {\emph et al.} [LAGO Collaboration], {\emph{The Large
Aperture GRB aperture}}, \en Proc. of the Observational Astronomy in Argentina
Workshop, Buenos Aires.

\end{etaremune}
