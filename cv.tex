%------------------------------------
% Dario Taraborelli
% Typesetting your academic CV in LaTeX
%
% URL: http://nitens.org/taraborelli/cvtex
% DISCLAIMER: This template is provided for free and without any guarantee 
% that it will correctly compile on your system if you have a non-standard  
% configuration.
% Some rights reserved: http://creativecommons.org/licenses/by-sa/3.0/
%------------------------------------

%!TEX TS-program = xelatex
%!TEX encoding = UTF-8 Unicode

%%% TODO %%%

% Agregar subsidios en los cuales participo
% pasar publicaciones a formato "bibtex"
% agregar una sección de desarrollo de prototipos: FOCA/FOG, ASCII, ANDES

\documentclass[11pt, a4paper]{article}
\usepackage{fontspec} 

% DOCUMENT LAYOUT
\usepackage{geometry} 
\geometry{a4paper, textwidth=5.5in, textheight=8.5in, marginparsep=7pt, marginparwidth=.6in}
\setlength\parindent{0in}

% FONTS
\usepackage[usenames,dvipsnames]{color}
\usepackage{xunicode}
\usepackage{xltxtra}
\defaultfontfeatures{Mapping=tex-text}
\setromanfont [Ligatures={Common}, Numbers={OldStyle}, Variant=01]{Linux Libertine O}
%\setmonofont[Scale=0.8]{Monaco}

% ---- CUSTOM COMMANDS
\chardef\&="E050
\newcommand{\html}[1]{\href{#1}{\scriptsize\textsc{[html]}}}
\newcommand{\pdf}[1]{\href{#1}{\scriptsize\textsc{[pdf]}}}
\newcommand{\doi}[1]{\href{#1}{\scriptsize\textsc{[doi]}}}
% ---- MARGIN YEARS
\usepackage{marginnote}
\newcommand{\amper{}}{\chardef\amper="E0BD }
\newcommand{\years}[1]{\marginnote{\scriptsize #1}}
\renewcommand*{\raggedleftmarginnote}{}
\setlength{\marginparsep}{7pt}
\reversemarginpar

% HEADINGS
\usepackage{sectsty} 
\usepackage[normalem]{ulem} 
\sectionfont{\mdseries\upshape\Large}
\subsectionfont{\mdseries\scshape\normalsize} 
\subsubsectionfont{\mdseries\upshape\large} 

% PDF SETUP
% ---- FILL IN HERE THE DOC TITLE AND AUTHOR
\usepackage[dvipdfm, bookmarks, colorlinks, breaklinks, 
% ---- FILL IN HERE THE TITLE AND AUTHOR
	pdftitle={Hernán Asorey - vita},
	pdfauthor={Hernán Asorey},
]{hyperref}
\hypersetup{linkcolor=blue,citecolor=blue,filecolor=black,urlcolor=MidnightBlue} 

% DOCUMENT
\begin{document}
{\LARGE Hernán Asorey}\\[1cm]
 Centro Atómico Bariloche\\
Av. E. Bustillo 9500\\
\texttt{(8400)}
San Carlos de Bariloche, R.N., Argentina\\[.2cm]
Tel.: \texttt{+54-2944-44-5151 int 38}\\[.2cm]
skype: \texttt{asoreyh}\\[.2cm]
email: \href{mailto:asoreyh@cab.cnea.gov.ar}{asoreyh@cab.cnea.gov.ar}\\
\textsc{url}: \href{http://fisica.cab.cnea.gov.ar/particulas/wiki/User:Asoreyh}{Página web}\\
Mayo, 2012.
\vfill
Nacido en Quilmes, Buenos Aires, Argentina el 05 de Febrero de 1974.\\
Nacionalidad argentina, casado, dos hijas.

%%\hrule\
\section*{Posición actual}
\emph{Posición permanente} en la Gerencia de Tecnología e Investigación en
Altas Energías, Centro Atómico Bariloche (CAB), Comisión Nacional de Energía
Atómica (concurso años
2008-2009).\\
\emph{Jefe de Trabajos Prácticos, a cargo del dictado de materia} del Área
Física, Sede Andina, Universidad Nacional de Río Negro (UNRN).\\
\emph{Auxiliar de primera, interino,} del Área Física, Instituto Balseiro,
UNC.\\

%%\hrule
\section*{Áreas de Especialización}
Física de astropartículas y Física de detección de radiación.

%%\hrule
\section*{Posiciones anteriores}
\noindent
\years{2006-2012}Doctorado en Física, Instituto Balseiro (UNC).\\
\years{2004-2005}Maestría en Ciencias Físicas, Instituto Balseiro (UNC).\\
\years{2002-2004}Licenciatura en Física, Instituto Balseiro (UNC).\\
\years{1994-1995}Auxiliar de Segunda Categoría con dedicación simple, ad
honorem, Universidad de Buenos Aires.\\
\years{1992-1996}Ingeniería Industrial (primeros cuatro años). Universidad de
Buenos Aires.\\
\years{1992-2001}AIM S.A., metalúrgica industrial, a cargo de diseño y
ejecución de proyectos industriales, Bernal, Buenos Aires, Argentina.\\

%\hrule
\section*{Educación}
\noindent
%%\years{2012}\textsc{Doctor en Física}, Instituto Balseiro (UNC).\\
\years{2005}\textsc{Magíster en Ciencias Físicas}, Orientación Partículas y
Campos, Instituto Balseiro (UNC).\\
\years{2004}\textsc{Licenciado en Física}, Instituto Balseiro (UNC).

%\hrule
\section*{Becas}
\noindent
\years{2008-2010}Beca de posgrado tipo II (CONICET), para la Carrera de
doctorado en Física en el Instituto Balseiro (UNC).\\
\years{2006-2007}Beca de posgrado tipo I (FUNC-CNEA), para la Carrera de
doctorado en Física en el Instituto Balseiro (UNC).\\
\years{2006}Beca de Iniciación a la Investigación (FUNC) para realizar tareas
de investigación en el Observatorio Pierre Auger.\\
\years{2005}Beca de maestría (CNEA), para la carrera de Maestría en Ciencias
Físicas en el Instituto Balseiro (UNC).\\
\years{2002-2004}Beca de grado (CNEA), para la carrera de Licenciatura en
Física, en el Instituto Balseiro (UNC).

\section*{Actividades}
\subsection*{Observatorio Pierre Auger}
\small\textit{Más información en \href{http://www.auger.org/}{www.auger.org} y
\href{http://www.auger.org.ar/}{www.auger.org.ar}.}
\begin{itemize}
\item Miembro de la Colaboración Internacional Pierre Auger.
\item Física de Rayos Cósmicos de las más altas energías.
\item Análisis de datos del Observatorio Pierre Auger.
\item Análisis de datos del Detector de Superficie del Observatorio Pierre
Auger, orientado a la detección de {\emph{Gamma Ray Bursts}} y eventos
transitorios de baja energía
\item Estudios de física heliosférica con los datos del Observatorio Pierre
Auger.
\item Análisis de datos de Monitoreo Atmosférico del Observatorio Pierre Auger
de Rayos Cósmicos.
\end{itemize}

\subsection*{Proyecto LAGO ({\emph{Large Aperture \textsc{Grb} Observatory}})}
\small\textit{Ver
\href{http://particulas.cnea.gov.ar/experiments/lago/}{particulas.cnea.gov.ar/experiments/lago}.
}
{\emph{Declarado de interés científico, académico y social por la Legislatura de la
Provincia de Río Negro, Declaración 42/2010.}}

\begin{itemize}
\item Miembro de la Colaboración internacional LAGO
\item Detección de GRB por el método de las partículas solitarias
\item Estudios de física heliosférica, fondos de radiación y física atmosférica
\end{itemize}

\subsection*{Laboratorio Subterráneo ANDES}
\small\textit{Ver
\href{http://www.andeslab.org}{www.andeslab.org}.
}
\begin{itemize}
\item Impulsor del proyecto
\item Estimación y mediciones del nivel de radiación de fondo natural en el
sitio del laboratorio. 
\end{itemize}

\subsection*{Docencia}
\begin{itemize}
\item Miembro de las cátedra de Experimental III del Instituto Balseiro (UNC), a cargo del experimento de física de rayos cósmicos de baja energía, utilizando el detector Nahuelito del proyecto LAGO.
\item Miembro de las cátedra de Introducción a Física de Partículas y Nuclear del Instituto Balseiro (UNC).
\item Miembro de la cátedra de Física de las carreras del profesorado de Física
y profesorado de Química de la UNRN, a cargo del dictado de las materias Física
1A y Física 1B.
\item Docente categoría 5 (resolución 10/01753) en el programa de incentivos a
Docentes Investigadores SPU/ME (convocatoria 2009). 
\end{itemize}

% \section*{Actuación profesional (síntesis)}
% 
% Desde el año 2005 realizo actividades relacionadas fundamentalmente con la
% física de rayos cósmicos, el Observatorio Pierre Auger, y la física de
% detectores de radiación. En ese mismo año ingresé a la Colaboración Auger,
% donde me orienté en primer lugar al análisis de datos del Observatorio, al
% desarrollo de los códigos de reconstrucción de eventos, y a la robustez de los
% mismos, es decir, que tan precisos son los resultados obtenidos por el
% Observatorio y cuales son los cortes que deben aplicarse sobre los datos para
% obtener resultados de la mayor calidad.  
% 
% Mi tesis de doctorado se centró preferentemente en los usos y la sensibilidad
% del detector de superficie del Observatorio Auger, que consiste en más de 1600
% detectores water-Cherenkov en un área de 3000 km$^2$.  Trabajando en
% colaboración con otros miembros del observatorio, hemos demostrado la
% sensibilidad de los detectores Cherenkov para realizar estudios de física
% atmosférica (estudios en tormentas e ionosféricos) y a la búsqueda de destellos
% de radiación gamma (GRBs).  Uno de los resultados de esos estudios es que el
% detector de superficie es sensible a la modulación de los rayos cósmicos
% galácticos por la actividad Solar. Las características únicas del detector
% permite realizar observaciones indirectas de la heliósfera con una estadística
% sin precedentes, y ha generado un gran interés en los resultados obtenidos, los
% cuales han producido una publicación de toda la colaboración Auger. En el año
% 2009 fui elegido para presentar los resultados en la International Cosmic Ray
% Conference (ICRC) en Lodz, Polonia.  
% 
% En el año 2007 participé en la construcción y el diseño del primer detector del
% proyecto LAGO, orientado a la búsqueda de destellos gamma en la alta montaña.
% Ese detector, ha sido el prototipo para los detectores que actualmente se
% encuentran en sitios en Bolivia, México, Perú, Venezuela.  También desarrollé
% estudios de física heliosférica con el mismo, estudios que han desembocado en
% el programa LAGO-Solar en los distintos sitios del proyecto LAGO. Actualmente,
% ese detector se encuentra en el Centro Atómico Bariloche y es utilizado para el
% monitoreo indirecto de la actividad solar, del fondo de radiación atmosférico
% de alta energía, y con fines didácticos en la medición de la vida media del
% muón en dos materias dictadas en el Instituto Balseiro, del cual formo parte
% del plantel docente.  
% 
% En forma paralela participé activamente del diseño y desarrollo del detector
% ASCII, basado en el detector de muones de AMIGA, que utiliza centelladores
% plásticos para la detección de lluvias de rayos cósmicos de las más altas
% energías. El prototipo se encuentra actualmente instalado en uno de los
% detectores de superficie del Observatorio Pierre Auger. Dada su portabilidad,
% el mismo detector puede ser usado para la medición de radiaciones ionizantes en
% distintos ambientes, como por ejemplo, en la cabina de un avión en altitud de
% crucero.
% 
% Desde el año 2009 participo en el diseño y construcción de un prototipo para la
% medición satelital en órbita geoestacionaria de la emisión terrestre en el
% ultravioleta cercano. Este instrumento, FOG (Fluoresencia desde Órbita
% Estacionaria), permitirá medir el fondo de radiación UV de la atmósfera, de
% suma importancia para la futura misión JEM/EUSO, así cómo la emisión de
% destellos UV medidos recientemente por el satelite ruso Tatiana y cuyo orígen
% es aún fruto de controversia.
% 
% He participado en la organización de eventos científicos nacionales e
% internaciones, escuelas y conferencias. También he realizado actividades
% docentes y de divulgación, ya que considero que la investigación en ciencias
% básicas debe vincularse estrechamente con otros sectores de la sociedad, con
% especial énfasis en la transferencia tecnológica.

\section*{Publicaciones \& presentaciones en conferencias}

\subsection*{Revistas con referato}
\noindent

\years{2012}H. Asorey y A. López Dávalos, {\emph{Fermi Problem: Power developed
at the eruption of the Puyehue-Cordón Caulle volcanic system in June 2011}},
Amer. Jour. Phys., {\emph{enviado}}, (2012)
\href{http://arxiv.org/abs/1109.1165}{arXiv:1109.1165v1}[physics.ed-ph]

\years{2012}S. Dasso y H. Asorey, for the Pierre Auger Collaboration,
\href{http://dx.doi.org/10.1016/j.asr.2011.12.028}{\emph{ The scaler mode in
the Pierre Auger Observatory to study heliospheric modulation of cosmic rays
}}, Adv. Space Res. {\bf{49}} (11), 1563--1569 (2012)

\years{2012}The Pierre Auger Collaboration,
\href{http://dx.doi.org/10.1007/s10686-011-9247-0}{\emph{Design concepts for
the Cherenkov Telescope Array CTA: an advanced facility for ground-based
high-energy gamma-ray astronomy}}, Exper. Astron. {\bf{32}} (3), 193--316
(2012)

\years{2012}The Pierre Auger Collaboration, 
\href{http://dx.doi.org/10.1016/j.astropartphys.2011.12.002}{\emph{Description
of atmospheric conditions at the Pierre Auger Observatory using the Global Data
Assimilation System (GDAS)}}, Astropart. Phys. {\bf{35}} (9), 591--607 (2012)

\years{2012}The Pierre Auger Collaboration, 
\href{http://dx.doi.org/10.1088/1475-7516/2011/11/022}{\emph{The effect of the
geomagnetic field on cosmic ray energy estimates and large scale anisotropy
searches on data from the Pierre Auger Observatory}}, JCAP {\bf{2011}} (022),
1--23 (2012)

\years{2012}The Pierre Auger Collaboration, 
\href{http://dx.doi.org/10.1016/j.astropartphys.2011.10.004}{\emph{Search for
signatures of magnetically-induced alignment in the arrival directions measured
by the Pierre Auger Observatory}}, Astropart. Phys. {\bf{35}} (6), 354--361
(2012)

\years{2011}The Pierre Auger Collaboration,
\href{http://dx.doi.org/10.1016/10.1103/PhysRevD.84.122005}{\emph{Search for
Ultra-High Energy Neutrinos in Highly Inclined Events at the Pierre Auger
Observatory}}, Phys.  Rev. {\bf D84}, 122005, 1--16 (2011)
\href{http://arxiv.org/abs/1202.1493}{arXiv:1202.1493}[astro-ph.HE]

\years{2011}The Pierre Auger Collaboration, 
\href{http://dx.doi.org/10.1016/j.astropartphys.2011.08.001}{\emph{The Lateral
Trigger Probability function for UHE Cosmic Rays Showers detected by the Pierre
Auger Observatory}}, Astropart. Phys. {\bf{35}} (5), 266--276 (2011)

\years{2011}The Pierre Auger Collaboration,
\href{http://dx.doi.org/10.1088/1475-7516/2011/06/022}{\emph{Anisotropy and
chemical composition of ultra-high energy cosmic rays using arrival directions
measured by the Pierre Auger Observatory}}, JCAP {\bf 06} 022 (2011),
\href{http://arxiv.org/abs/1106.3048}{arXiv:1101.3048v1}[astro-ph.HE]

\years{2011}The Pierre Auger Collaboration,
\href{http://dx.doi.org/10.1016/j.nima.2011.01.049}{{\emph{Advanced
functionality for radio analysis in the Offline software framework of the
cierre Auger Observatory}}}, NIM {\bf A635} 92--102
(2011),\href{http://arxiv.org/abs/1101.4473}{arXiv:1101.4473v1}[astro-ph.HE]

\years{2011}The Pierre Auger Collaboration,
\href{http://dx.doi.org/10.1016/j.astropartphys.2010.12.007}{\emph{Search for
First Harmonic Modulation in the Right Ascension Distribution of Cosmic Rays
Detected at the Pierre Auger Observatory}}, Astropart. Phys. {\bf 34} 627--639
(2011)

\years{2011}The Pierre Auger Collaboration,
\href{http://dx.doi.org/10.1088/1748-0221/6/01/P01003}{\emph{The Pierre Auger
Observatory Scaler Mode for the Study of the Modulation of Galactic Cosmic Rays
due to Solar Activity}}, JINST {\bf 6} P01003--P01020 (2011).
$^*${\bf{Coordinador}}

\years{2010}The Pierre Auger Collaboration,
\href{http://dx.doi.org/10.1016/j.astropartphys.2010.10.001}{\emph{The exposure
of the hybrid detector of the Pierre Auger Observatory}}, Astropart. Phys. {\bf
34}, 368--381 (2011)

\years{2010}The Pierre Auger Collaboration,
\href{http://dx.doi.org/10.1016/j.astropartphys.2010.08.010}{\emph{Update on
the correlation of the highest energy cosmic rays with nearby extragalactic
matter}},Astropart. Phys. {\bf 34}, 314--326 (2010),
\href{http://arxiv.org/abs/1009.1855}{arXiv:1009.1855v2}[astro-ph.HE]

\years{2010}The Pierre Auger Collaboration,
\href{http://dx.doi.org/10.1016/j.nima.2010.04.023}{\emph{The Fluorescence
Detector of the Pierre Auger Observatory}}, NIM {\bf A620}, 227 (2010),
\href{http://arxiv.org/abs/0907.4282}{arXiv:0907.4282v1}[astro-ph.IM]

\years{2010}J. Bl\"umer and The Pierre Auger Collaboration,
\href{http://dx.doi.org/10.1088/1367-2630/12/3/035001}{\emph{The Northern Site
of the Pierre Auger Observatory}}, Journal of Physics {\bf 12} (3) 035001
(2010)

\years{2010}The Pierre Auger Collaboration,
\href{http://dx.doi.org/10.1016/j.astropartphys.2009.12.005}{\emph{A Study of
the Effect of Molecular and Aerosol Conditions in the Atmosphere on Air
Fluorescence Measurements at the Pierre Auger Observatory}}, Astropart. Phys.
{\bf 33}, 108--129 (2010),
\href{http://arxiv.org/abs/0907.4282}{arXiv:1002.0366v1}[astro-ph.HE]

\years{2010}The Pierre Auger Collaboration,
\href{http://dx.doi.org/10.1016/j.physletb.2010.02.013}{\emph{Measurement of
the energy spectrum of cosmic rays above 1018 eV using the Pierre Auger
Observatory}}, Phys. Lett. {\bf B685} 239--246 (2010),
\href{http://arxiv.org/abs/1002.1975}{arXiv:1002.1975v1}[astro-ph.HE]

\years{2010}The Pierre Auger Collaboration,
\href{http://dx.doi.org/10.1103/PhysRevLett.104.091101}{\emph{Measurement of
the Depth of Maximum of Extensive Air Showers above 10$^{18}$ eV}}, PRL {\bf
104} 091101
(2010)\href{http://arxiv.org/abs/1002.0699}{arXiv:1002.0699v1}[astro-ph.HE]

\years{2010}The Pierre Auger Collaboration,
\href{http://dx.doi.org/10.1016/j.nima.2009.11.018}{\emph{Trigger and Aperture
of the Surface Detector Array of the Pierre Auger Observatory}}, NIM {\bf A613}
29--39, (2010)

\years{2009} The Pierre Auger Collaboration,
\href{http://dx.doi.org/10.1016/j.astropartphys.2009.06.004}{\emph{Atmospheric
effects on extensive air showers observed with the Surface Detector of the
Pierre Auger Observatory}}, Astropart. Phys. {\bf 32}, 89--99, (2009),
\href{http://arxiv.org/abs/0906.5497/}{arXiv:0906.5497v2}[astro-ph.IM]

\years{2009}The Pierre Auger Collaboration,
\href{http://dx.doi.org/10.1016/j.astropartphys.2009.04.003}{\emph{Upper limit
on the cosmic-ray photon fraction at EeV energies from the Pierre Auger
Observatory.}}, Astropart. Phys. {\bf 31} 399--406
(2009)\href{http://arxiv.org/abs/0903.1127/}{arXiv:0903.1127v1 [astro-ph.HE]}

\years{2009}The Pierre Auger Collaboration,
\href{http://dx.doi.org/10.1103/PhysRevD.79.102001}{\emph{Limit on the diffuse
flux of ultra-high energy tau neutrinos with the surface detector of the Pierre
Auger Observatory.}}, Phys. Rev. {\bf D79} 10:1--15
(2009)\href{http://arxiv.org/abs/0903.3385/}{arXiv:0903.3385v1 [astro-ph.HE]}

\years{2008}D. Allard {\emph et al.} [LAGO Collaboration],
\href{http://dx.doi.org/10.1016/j.nima.2008.07.041}{\emph{Use of
water-Cherenkov detectors to detect Gamma Ray Bursts at the Large Aperture GRB
Observatory (LAGO)}}, NIM {\bf A595} 70--72 (2008)

\years{2008}The Pierre Auger Collaboration,
\href{http://dx.doi.org/10.1103/PhysRevLett.101.061101}{\emph{Observation of
the Suppression of the Flux of Cosmic Rays above $4\times10^{19}$\,eV.}}, PRL
{\bf 101} 061101 (2008)

\years{2008}The Pierre Auger Collaboration,
\href{http://dx.doi.org/10.1103/PhysRevLett.100.211101}{\emph{Upper limit on
the diffuse flux of UHE tau neutrinos from the Pierre Auger Observatory.}}, PRL
{\bf 100} 21101 (2008)

\years{2008}The Pierre Auger Collaboration,
\href{http://dx.doi.org/10.1016/j.astropartphys.2008.01.003}{\emph{Upper limit
on the cosmic-ray photon flux above 10$^{19}$\,eV using the surface detector of
the Pierre Auger Observatory.}}, Astropart. Phys. {\bf 29} 243--256 (2008)

\years{2008}The Pierre Auger Collaboration,
\href{http://dx.doi.org/10.1016/j.astropartphys.2008.01.002}{\emph{Correlation
of the highest-energy cosmic rays with the positions of nearby active galactic
nuclei.}}, Astropart. Phys. {\bf 29} 188--204 (2008)

\years{2007}The Pierre Auger Collaboration,
\href{http://dx.doi.org/10.1126/science.1151124}{\emph{Correlation of the
highest energy cosmic rays with nearby extragalactic objects.}}, Science {\bf
318} 939--943 (2007)

\years{2007}The Pierre Auger Collaboration,
\href{http://dx.doi.org/10.1016/j.astropartphys.2006.11.002}{\emph{Anisotropy
studies around the galactic centre at EeV energies with the Auger
Observatory.}},  Astropart. Phys. {\bf 27} 244--253 (2007)

\years{2007}The Pierre Auger Collaboration,
\href{http://dx.doi.org/10.1016/j.astropartphys.2006.10.004}{\emph{An upper
limit to the photon fraction in cosmic rays above 10$^{19}$\,eV from the Pierre
Auger Observatory.}}, Astropart. Phys. {\bf 27} 155--168 (2007)

\subsection*{Conferencias \& Escuelas}
\noindent

\years{2012} H. Asorey, M. Arribere, X. Bertou, M. Gómez Berisso, F. Sánchez, 
{\emph{Expected Backgrounds at the ANDES Underground Laboratory}}, charla
plenaria en el Third International Workshop for the Design of the ANDES
Underground Laboratory, Valparaiso, Chile, 11--12 Ene 2012.

\years{2012}H. Asorey [Pierre Auger Collaboration], {\emph{Heliospheric
Modulation of Cosmic Rays Observed by the Pierre Auger Observatory and the LAGO
Project}} , charla paralela sección Astrofísica en el 4th International
Workshop of High Energy Physics in the LHC Era HEP2012, Valparaiso, Chile,
4--10 Ene 2012.

\years{2011}H. Asorey, A. López Dávalos y M. Gómez Berisso, {\emph{Potencia de
la erupción volcánica del complejo Puyehue-Cordón Caulle como un problema de
Fermi}}, charla de división Enseñanza de la Física en la II Reunión Conjunta
SUF-AFA2011, Montevideo, Uruguay, 20 al 23 Sep 2011.

\years{2011}I. Allekotte, H. Arnaldi, H. Asorey, X. Bertou, M. Gómez Berisso,
M. Sofo Haro, {\emph{Desarrollos de electrónica rápida y de bajo consumo en el
Laboratorio de Detección de Partículas y radiación de Bariloche}}, póster
presentado en la II Reunión Conjunta SUF-AFA2011, Montevideo, Uruguay, 20 al 23
Sep 2011.

\years{2011}H. Asorey[Pierre Auger Collaboration], {\emph{Low energy radiation
measurements with the water Cherenkov detector array of the Pierre Auger
Observatory}}, en Proc. 32th International Cosmic Ray Conference, vol. 11
462--465, Beijing, China, 11--18 Ago 2011

\years{2011}The Pierre Auger Collaboration,
\href{http://arxiv.org/abs/1107.4807}{\emph{The Pierre Auger Observatory V:
Enhancements}}, en Proc. 32th International Cosmic Ray
Conference, Beijing, China, 11--18 Ago 2011.

\years{2011}The Pierre Auger Collaboration,
\href{http://arxiv.org/abs/1107.4806}{\emph{The Pierre Auger Observatory IV:
Operation and Monitoring}}, en Proc. 32th International Cosmic Ray
Conference, Beijing, China, 11--18 Ago 2011.

\years{2011}The Pierre Auger Collaboration,
\href{http://arxiv.org/abs/1107.4805}{\emph{The Pierre Auger Observatory III:
Other Astrophysical Observations}}, en Proc. 32th International Cosmic Ray
Conference, Beijing, China, 11--18 Ago 2011.

\years{2011}The Pierre Auger Collaboration,
\href{http://arxiv.org/abs/1107.4804}{\emph{The Pierre Auger Observatory II:
Studies of Cosmic Ray Composition and Hadronic Interaction models}}, en Proc.
32th International Cosmic Ray Conference, Beijing, China, 11--18 Ago 2011.

\years{2011}The Pierre Auger Collaboration,
\href{http://arxiv.org/abs/1107.4809}{\emph{The Pierre Auger Observatory I: The
Cosmic Ray Energy Spectrum and Related Measurements}}, en Proc. 32th
International Cosmic Ray Conference, Beijing, China, 11--18 Ago 2011.

\years{2010}H. Asorey[Pierre Auger Collaboration],
\href{http://95rnf.afa.webfactional.com/tex\_files/Resumenes/DPyC/PyC\_6.pdf}{\emph{El
arreglo Infill del Observatorio Pierre Auger}}, charla de división Partículas y
Campos en la 95$^\mathrm{a}$ Reunión Nacional de Física AFA2010, Malargüe,
Argentina, 28 Sep al 01 Oct 2010.

\years{2010}H. Asorey, J. Castro, A. López Dávalos,
\href{http://95rnf.afa.webfactional.com/tex\_files/Resumenes/EF/asorey.pdf}{\emph{Kepler,
Newton, Feynman}}, poster presentado en la división Enseñanza de la física en
la 95$^\mathrm{a}$ Reunión Nacional de Física AFA2010, Malargüe, Argentina, 28
Sep al 01 Oct 2010.

\years{2010}H. Asorey[LAGO Collaboration], {\emph{The Large Aperture Gamma Ray
Burst Observatory (LAGO)}}, charla plenaria en el 3rd International Workshop of
High Energy Physics in the LHC Era HEP2010, Valparaiso, Chile, 4--8 Ene 2010.

\years{2009}H. Asorey[Pierre Auger Collaboration], {\emph{Cosmic Ray Solar
Modulation Studies at the Pierre Auger Observatory}}, en Proc. 31th
International Cosmic Ray Conference, Lodz, Polonia, 8--15 Jul 2009.

\years{2009}The Pierre Auger Collaboration,
\href{http://arxiv.org/abs/0906.2358}{\emph{Calibration and Monitoring of the
Pierre Auger Observatory.}}, en Proc. 31th International Cosmic Ray Conference,
Lodz, Polonia, 8--15 Jul 2009.

\years{2009}The Pierre Auger Collaboration,
\href{http://arxiv.org/abs/0906.2354}{\emph{Operations of and Future Plans for
the Pierre Auger Observatory}}, en Proc. 31th International Cosmic Ray
Conference, Lodz, Polonia, 8--15 Jul 2009.

\years{2009}The Pierre Auger Collaboration,
\href{http://arxiv.org/abs/0906.2347}{\emph{Astrophysical Sources of Cosmic
Rays and Related Measurements with the Pierre Auger Observatory}}, en Proc.
31th International Cosmic Ray Conference, Lodz, Polonia, 8--15 Jul 2009.

\years{2009}The Pierre Auger Collaboration,
\href{http://arxiv.org/abs/0906.2319}{\emph{Studies of Cosmic Ray Composition
and Air Shower Structure with the Pierre Auger Observatory}}, en Proc. 31th
International Cosmic Ray Conference, Lodz, Polonia, 8--15 Jul 2009.

\years{2009}The Pierre Auger Collaboration,
\href{http://arxiv.org/abs/0906.2189}{\emph{The Cosmic Ray Energy Spectrum and
Related Measurements with the Pierre Auger Observatory}}, en Proc. 31th
International Cosmic Ray Conference, Lodz, Polonia, 8--15 Jul 2009.

\years{2009}The LAGO Collaboration,
\href{http://arxiv.org/abs/0906.0820}{\emph{Water Cherenkov Detectors response
to a Gamma Ray Burst in the Large Aperture GRB Observatory}}, en Proc. 31th
International Cosmic Ray Conference, Lodz, Polonia, 8--15 Jul 2009.

\years{2009}The LAGO Collaboration,
\href{http://arxiv.org/abs/0906.0816}{\emph{The Large Aperture GRB
Observatory}}, en Proc. 31th International Cosmic Ray Conference, Lodz,
Polonia, 8--15 Jul 2009.

\years{2009}The LAGO Collaboration,
\href{http://arxiv.org/abs/0906.0814}{\emph{Operating Water Cherenkov Detectors
in high altitude sites for the Large Aperture GRB Observatory}}, en Proc. 31th
International Cosmic Ray Conference, Lodz, Polonia, 8--15 Jul 2009.

\years{2009}H. Asorey[Pierre Auger Collaboration], {\emph{The Acceptance of the
Pierre Auger Observatory}}, poster presentado en el VII Simposio
latinoamericano de Altas Energías SILAFAE - IX Simposio Anual de Partículas y
Campos, San Carlos de Bariloche, Argentina, 14-21 Ene 2009.

\years{2008}XVI Course of the ISCRA (International School of Cosmic Ray
Astrophysics) 2008: “Gamma Ray and Cosmic Ray Astrophysics: From below GeV to
beyond EeV Energies”, Erice, Italia, Julio 2008

\years{2008}Trabajo en Colaboración en la Universidad de Siegen, Siegen,
Alemania. Charla invitada: ``Towards Cosmic ray Solar Modulation Studies'', Julio
2008.

\years{2007}D. Allard {\emph et al.} [LAGO Collaboration], {\emph{Looking for
the high energy component of GRBs at the Large Aperture GRB Observatory}}, en
Proc. 30th International Cosmic Ray Conference,  Mérida, México, 3-11 Jul
2007.

\years{2007}IV Latin American School of Strings LASS 07, San Carlos de
Bariloche, Enero 2007.

\years{2006}H. Asorey[Pierre Auger Collaboration], {\emph{The Surface Detector
Array of the Pierre Auger Observatory}}, charla semiplenaria 1st International
Workshop of High Energy Physics in the LHC Era HEP2006, Valparaiso, Chile,
12--17 Dic 2006.

\years{2006}D. Allard {\emph et al.} [LAGO Collaboration], {\emph{The Large
Aperture GRB aperture}}, en Actas del Workshop Astronomia Observacional en
Argentina. Buenos Aires, 2006

\years{2005}Third CERN-CLAF Latin American School Of High Energy Physics, CERN,
Malargüe, Pcia. Mendoza, Argentina. Asistencia a los cursos y Presentación del
Poster: “Event Reconstruction using the Surface Detectors At UHECR Pierre Auger
Observatory”, Marzo 2005.

\years{2004}Sixth J. J. Giambiagi Winter School on Particle Physics, Facultad
de Ciencias Exactas y Naturales, Universidad de Buenos Aires. Julio 2004.

\years{2005-2012} Numerosas charlas dadas en los Encuentros de la Colaboración
Pierre Auger, Malargüe, Pcia. de Mendoza.

\subsection*{Notas técnicas del Observatorio Pierre Auger (GAP Notes)}

Las notas GAP son notas técnicas internas del Proyecto Pierre Auger. Es posible
acceder a los artículos de carácter público en
\href{http://www.auger.org/admin-cgi-bin/woda/gap\_notes.pl/Search?search=asorey}
{www.auger.org/admin/GAP\_Notes}.\\

\years{2011-010} R. Ravignani, H. Asorey, D. Melo, G. De La Vega, A. Etchegoyen, A.
Ferrero, R. F. Gamarra, B. García, M. Josebachuili, F. Sánchez, I. Sidelnik, A.
Tapia, B. Wundheiler, {\emph{Observation of the spectrum with the AMIGA
infill}}\\
\years{2009-155}H. Asorey, I. Allekotte, X. Bertou, M. Gómez~Berisso,
{\emph{Acceptance of generalized Surface Detector Arrays from real data}}\\
\years{2009-154}H. Asorey, X. Bertou, D. Thomas, M. Mostafá, {\emph{The OMG
Hybrid Event}}\\
\years{2009-112}H. Asorey, I. Allekotte, X. Bertou, M.
Gómez~Berisso, {\emph{Determining the acceptance of the Pierre Auger Surface
Detector with the Infill Array}}.\\
\years{2009-019}I. Allekotte, H. Asorey, M. Gómez~Berisso, {\emph{Improving the
determination of the Auger Surface Detector Single Station Trigger Probability
from real data}}.\\
\years{2008-117}H. Asorey, X. Bertou, {\emph{Determining the Dynamic Range
needed for new Surface Detectors.}}\\
\years{2008-114}I. Allekotte, H. Asorey, X. Bertou, M. G\'omez~Berisso,
{\emph{You thought you understood hexagons?}}\\
\years{2008-112}S. Grebe, I. Allekotte, H. Asorey, X. Bertou, P. Buchholz,
{\emph{Robustness of the CDAS reconstruction algorithm.}}\\
\years{2008-072}H. Asorey, X. Bertou, {\emph{First large timescale analysis of
Auger SD scaler data: Towards cosmic ray Solar modulation studies.}}\\
\years{2007-088}H. Asorey, I. Allekotte, {\emph{Towards a complete set of
weather data.}}\\
\years{2006-052}H. Asorey, X. Bertou, E. Roulet, {\emph{How to improve the SD
arrival direction reconstruction by correcting the start-time of individual
detectors.}}\\
\years{2005-107}H. Asorey, I. Allekotte, M. Gómez~Berisso, X. Bertou,
{\emph{Robustness of the angular reconstruction with the Surface Array of the
Auger Observatory.}}\\
\years{2005-084}H. Asorey, I. Allekotte, M. G\'omez~Berisso, X. Bertou,
{\emph{Robustness of the energy reconstruction with the Surface Array of the
Auger Observatory.}}\\

\section*{Actividades docentes}

\years{2011}Miembro de la cátedra de Introducción a Física de Partículas y Nuclear del Instituto Balseiro, Universidad Nacional de Cuyo.

\years{2012}A cargo del dictado de las materias Física 1 A y Física 1 B de las
carreras de Profesorado de Física y Profesorado de Química de la Universidad
Nacional de Río Negro.

\years{2009-2011}Miembro de las cátedras de Física 1 A y Física 1 B de las
carreras de Profesorado de Física y Profesorado de Química de la Universidad
Nacional de Río Negro.

\years{2010}Miembro de la cátedra de Física Experimental III del Instituto
Balseiro, Universidad Nacional de Cuyo. A cargo del experimento de física de
rayos cósmicos y determinación del tiempo de decaimiento del muón, utilizando
el detector ``Nahuelito'' del proyecto LAGO.

\years{2010}Docente en el curso del Centro de Formación Continua ``Óptica para
lunáticos - CFC 2010'' para profesores de Enseñanza Media en el Instituto
Balseiro, Julio 2010.

\section*{Extensión y Divulgación}
\noindent
\years{2011}H. Asorey, A. Clúa, A. López Dávalos
\href{http://www.clarin.com/sociedad/Cien-millones-toneladas-cenizas-solo_0_517148395.html}{Cien
millones de toneladas de cenizas en un solo día}, Clarín, 2011. Reproducida en
cientos de medios nacionales e internacionales.  

\years{2011}H. Asorey, A. Clúa, A. López Dávalos
\href{http://www.anbariloche.com.ar/noticia.php?nota=22468}{Potencia de la
erupción del Puyehue se asemeja a la de 70 bombas atómicas}, AN Bariloche,
2011. 

\years{2011}H. Asorey, {\emph{Viviendo con una estrella}}, ciclo de charlas
sobre Física Solar, Astronomía, Astrofísica para alumnos de establecimientos de
enseñanza media públicos y privados de la Provincia de Río Negro. Inicio Marzo
2011.

\years{2011}Miembro del Comité Local de Organización del ``First International Workshop
for the Design of the ANDES Underground Laboratory'', Centro Atómico
Constituyentes, Buenos Aires, Argentina, 11-14 April 2011

\years{2010}Miembro del Comité Local de Organización de la "95$^\mathrm{ta}$
Reunión Nacional de Física, AFA 2010", Malargüe, Argentina, Setiembre 2010.

\years{2010}Miembro del Comité Local de Organización de la "XI ICFA School on
Instrumentation in Elementary Particle Physics", San Carlos de Bariloche,
Argentina, Enero 2010.

\years{2010}{\emph{Solicitan destacar labor de científicos del Centro Atómico
Bariloche}} (H. Asorey, X. Bertou, M. Gómez Berisso), El Cordillerano,
Bariloche 2000 y ANBariloche, 2010.

\years{2010}{\emph{Larraburu quiere destacar labor de científicos del Centro
Atómico Bariloche}} (H. Asorey, X. Bertou, M. Gómez Berisso), El Cordillerano,
2010.

\years{2010}Laura García, {\emph{Red latinoamericana de detectores estudia la
radiación gamma}} (H. Asorey, X. Bertou, M. Gómez Berisso), El Cordillerano,
Bariloche 2000 y ANBariloche, 2010.

\years{2009}Miembro del Comité Local de Organización del "VII Simposio
Latinoamericano de Física de Altas Energías (SILAFAE) - IX Simposio Argentino
de Partículas y Campos (SAPyC)", San Carlos de Bariloche, Argentina, Enero
2009.

\years{2009-2011}H. Asorey,
\href{http://fisica1-unrn.blogspot.com}{\emph{Física 1 en la UNRN}}, blog de
física en castellano para todo público. Hasta 5000 visitas anuales.

\years{2009-2010}H. Asorey, {\emph{Que ¡NO! se quede el infinito sin
estrellas}}, ciclo de charlas sobre Astronomía, Astrofísica y Cosmología para
alumnos de establecimientos de enseñanza media públicos y privados de la
Provincia de Río Negro. Inicio Sep 2009.

\years{2009}H. Asorey, {\emph{Astrofísica para todos - Noticias del cielo - El
cielo del mes}}, columna bimestral en la revista ``Bariloche Naturaleza y
Tecnología''. Columnas: Ecos de Luz, Nuevas Tierras, Habitabilidad, SETI en
casa.

\years{2009}H. Asorey, {\emph{Se inaguró el mayor Observatorio de Rayos
Cómsmicos del Mundo}}, revista ``Bariloche Naturaleza y Tecnología'', Número
33, Año VII (2009).

\years{2008}H. Asorey, {\emph{El Observatorio Pierre Auger, una mirada al
Universo a las más altas energías}}, charla invitada, Universidad Nacional de
Quilmes, Abril 2008.

%\vspace{1cm}
\vfill{}
%\hrulefill
\end{document}
