\ifeng
\section*{Research \& Teaching Activities}

Since I have earned my master degree in December 2005, I have been involved in
the following projects:

\subsection*{Pierre Auger Observatory}

{\small{\textit{See \href{http://www.auger.org/}{www.auger.org}}}}
\begin{itemize}
\item Member of the Pierre Auger Collaboration since 2006
\item Ultra High-Energy Cosmic Rays Physics 
\item Data analysis of the Surface Detector
\item Development of the reconstruction event chain of the Surface Detector
\item Development and applications of the low energy modes (scaler and histogram
modes) of the surface detectors of the Pierre Auger Observatory, for the study
of transient events (Gamma Ray Bursts and Forbush events), and short and long
term modulation of the galactic cosmic rays flux due to solar activity
\item CORSIKA and detector simulations, oriented to determine the
water-Cherenkov response working in the low energy modes
\item Data analysis of the weather monitoring system of the Pierre Auger
Observatory
\end{itemize}

\subsection*{Large Aperture Grb Observatory (LAGO)}
{\emph{Declared of Scientific, Academic and Social interest by the Honourable
House of Representatives of the Rio Negro Province, Dec. 42/2010.}}\\
{\small{\textit{See
\href{http://fisica.cab.cnea.gov.ar/particulas/laboratorio/lago}
{http://fisica.cab.cnea.gov.ar/particulas/laboratorio/lago}}}}\\
\begin{itemize}
\item Member of the LAGO International Collaboration since 2006
\item Simulations and data analysis for the detection of transient events
(GRB and Forbush events), background radiation and atmospheric physics.
\item Research, development and building of three water-Cherenkov detector
prototypes for the LAGO project at Bariloche Atomic Centre. One of them will be
installed at the Antarctic Peninsula.
\item Design and coordination of the experiment ``Measurement of Muon Lifetime in
Water'', done by undergraduate students at Instituto Balseiro.
\end{itemize}

\subsection*{Cherenkov Telescope Array (CTA)}
{\small{\textit{See \href{http://www.cta-observatory.org}{www.cta-observatory.org}}}}
\begin{itemize}
\item Member of the CTA consortium since 2010
\item Research and development of the autonomous station for control and data
acquisition of the weather station and sky quality meter installed in San
Antonio de los Cobres, Argentina, one of the site candidates for the CTA
observatory.
\end{itemize}

\subsection*{ANDES Underground Laboratory}
{\small{\textit{See \href{http://www.andeslab.org}{www.andeslab.org}}}}
\begin{itemize}
\item Estimation and measurements of the expected backgrounds at the ANDES
underground lab due to natural radioactivity and high energy atmospheric muons
\end{itemize}

\subsection*{Teaching}
{\small{\textit{See \href{http://www.ib.edu.ar}{www.ib.edu.ar} and
\href{http://www.unrn.edu.ar}{www.unrn.edu.ar}}}}
\begin{itemize}
\item Teaching assistant, Experimental Physics III and Introduction to nuclear
and particle physics courses, Instituto Balseiro (UNC)
\item Senior teaching assistant, Physics I (introductory physics) course, UNRN.
\end{itemize}
\else
\section*{Actividades de Investigación \& Docencia}

Desde que obtuve mi Maestría en Diciembre de 2005, me he involucrado en los siguientes proyectos:

\subsection*{Observatorio Pierre Auger}

{\small{\textit{Ver \href{http://www.auger.org/}{www.auger.org}}}}
\begin{itemize}
\item Miembro de la Colaboración Internacional Auger desde el añó 2006.
\item Física de Rayos Cósmicos de Ultra Alta Energía.
\item Análisis de datos del arreglo de detectores de superficie (SD) del Observatorio.
\item Desarrollo de la cadena de reconstrucción de eventos registrados por el detector SD.
\item Desarrollo y aplicaciones de los modos de bajas energías (modo ``scaler'' y modo ``histograma'') para el estudio de eventos astrofísicos transitorios (GRBs y eventos Forbush), y sobre la modulación a corto y largo plazo del flujo de rayos cósmicos galácticos debida a la actividad solar.
\item Simulaciones del detector y de rayos cósmicos para la determinación de la respuesta de los detectores water-Cherenkov en los modos de baja energía.
\item Análisis de datos del sistema de sensado atmosféricos del Observatorio.
\end{itemize}

\subsection*{Proyecto LAGO (Large Aperture Grb Observatory)}
{\emph{Declarado de interés científico, académico y social por la Legislatura de la
Provincia de Río Negro, Declaración 42/2010.}}
{\small{\textit{Ver \href{http://fisica.cab.cnea.gov.ar/particulas/laboratorio/lago}{http://fisica.cab.cnea.gov.ar/particulas/laboratorio/lago}}}}\\
\begin{itemize}
\item Miembro de la Colaboración Internacional LAGO desde 2006
\item Simulaciones y análisis de datos para la detección de eventos transitorios (GRBs y eventos Forbush), radiación de fondo y física de la atmósfera.
\item Investigación, desarrollo y construcción de tres detectores prototipos tipo Cherenkov en agua en el Centro Atómico Bariloche. Uno de ellos será instalado en la Península Antártica. 
\item Diseño y coordinación del experimento ``Determinación de la Vida Media del Muón en Agua'', hecho por los estudiantes de grado del Instituto Balseiro.
\end{itemize}

\subsection*{Cherenkov Telescope Array (CTA)}
{\small{\textit{Ver \href{http://www.cta-observatory.org}{www.cta-observatory.org}}}}
\begin{itemize}
\item Miembro del consorcio CTA desde el año 2010.
\item Investigación y desarrollo de una estación autónoma y remota para el control y la adquisición de datos de una estación meteorológica y un medido de calidad del cielo, instalados en la localidad de San Antonio de los Cobres, Salta, Argentina (uno de los sitios candidatos para la instalación del Observatorio CTA).
\end{itemize}

\subsection*{Laboratorio Subterráneo ANDES}
{\small{\textit{Ver \href{http://www.andeslab.org}{www.andeslab.org}}}}
\begin{itemize}
\item Estimación del fondo de radiación esperado en el laboratorio subterráneo ANDES debido a la radiactividad natural y al flujo de muones atmosféricos de alta energía.
\end{itemize}

\subsection*{Docencia}
{\small{\textit{Ver \href{http://www.ib.edu.ar}{www.ib.edu.ar} \& \href{http://www.unrn.edu.ar}{www.unrn.edu.ar}}}}
\begin{itemize}
\item Miembro de las cátedras de Experimental III del Instituto Balseiro (UNC), a cargo del experimento de física de rayos cósmicos de baja energía, utilizando el detector Nahuelito del proyecto LAGO.
\item Miembro de las cátedra de Introducción a Física de Partículas y Nuclear del Instituto Balseiro (UNC).
\item Miembro de la cátedra de Física de las carreras del profesorado de Física y profesorado de Química de la UNRN, a cargo del dictado de las materias Física 1A y Física 1B.
\item Docente categoría 5 (resolución 10/01753) en el programa de incentivos a Docentes Investigadores SPU/ME (convocatoria 2009).
\end{itemize}
\fi
