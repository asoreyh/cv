\ifeng
\section*{Research \& Teaching Activities}

Since I have earned my master degree in December 2005, I have been involved in the following projects:

\subsection*{Medical Physics Department, CAB,(2016-Present)}
\begin{itemize}
	\item Task leader of the Advanced Techniques in Radiotherapy and Dosimetry
		group.
	\item Intensity modulated radiotherapy (IMRT) simulations based on the
		Geant4 based OpenGate implementation.
\end{itemize}

\subsection*{Pierre Auger Observatory (2006-Present)}
{\small{\textit{See \href{http://www.auger.org/}{www.auger.org}}}}
\begin{itemize}
\item Task leader of the ``Cosmo-Geophysics'' task of the Pierre Auger Observatory
\item Data analysis of the Surface Detector
\item Extensive Atmospheric Shower Physics
\item Development of the reconstruction event chain of the Surface Detector
\item Development and applications of the low energy modes (scaler and histogram
modes) of the surface detectors of the Pierre Auger Observatory, for the study
of transient events (Gamma Ray Bursts and Forbush events), and short and long
term modulation of the galactic cosmic rays flux due to solar activity
\item CORSIKA and detector simulations, oriented to determine the
water-Cherenkov response working in the low energy modes
\item Data analysis of the weather monitoring system of the Pierre Auger
Observatory
\end{itemize}

\subsection*{Latin American Giant Observatory (LAGO) (2007-Present)}
{\small{\textit{See \href{http://lagoproject.org}{lagoproject.org}}}}
\begin{itemize}
\item Principal Investigator, period 2013-2016
\item Design and execution of the project new organization
\item Design and coordination of the LAGO Space Weather program 
\item Simulations and data analysis for the detection of transient events
(GRB and Forbush events), background radiation and atmospheric physics.
\item Research, development and building of water-Cherenkov detectors for the
	LAGO project at Universidad Industrial de Santander and Centro Atómico
	Bariloche. One of them will be
installed at the Antarctic Peninsula.
\item Design and coordination of the experiment ``Measurement of Muon Lifetime in Water'', done by undergraduate students at Instituto Balseiro.
\end{itemize}

\subsection*{Cherenkov Telescope Array (CTA) (2010-2014)}
{\small{\textit{See \href{http://www.cta-observatory.org}{www.cta-observatory.org}}}}
\begin{itemize}
\item San Antonio de los Cobres site characterization
\item Research and development of the autonomous station for control and data
acquisition of the weather station and sky quality meter installed in San
Antonio de los Cobres, Argentina, one of the site candidates for the CTA
observatory.
\end{itemize}

\subsection*{ANDES Underground Laboratory (2010-2013, 2015-2016)}
{\small{\textit{See \href{http://www.andeslab.org}{www.andeslab.org}}}}
\begin{itemize}
\item Estimation and measurements of the expected backgrounds at the ANDES
underground lab due to natural radioactivity and high energy atmospheric muons
\end{itemize}

\subsection*{Teaching (2009-Present)}
{\small{\textit{See \href{http://www.ib.edu.ar}{www.ib.edu.ar}, \href{http://www.uis.edu.co}{www.uis.edu.co}, and \href{http://www.unrn.edu.ar}{www.unrn.edu.ar}}}}
\begin{itemize}
\item Associated Profesor, Modern Physics A and Physics II B, Profesorado de Física, Sede Andina, Universidad Nacional de Río Negro (UNRN)
\item Professor, Classical Mechanics (Graduate) and General Astronomy, School of Physics, UIS.
\item Professor, Introductory Physics course and Introductory Particle Physics course, UIS.
\item Design and lecture of the course ``Astro-meteorology and Climate Change'', intended for High Schools teachers, UIS, March 2014.
\item Professor, Advanced Mathematical Methods for Physics course, UIS.
\item Senior teaching assistant, Physics I (introductory physics) course, UNRN.
\item Teaching assistant, Experimental Physics III and Introduction to nuclear
and particle physics courses, Instituto Balseiro (UNC)
\item Member of the Academic Committee of the Master in Medical Physics program of the Instituto Balseiro, Universidad Nacional de Cuyo.
\end{itemize}
\else
\section*{Actividades de Investigación \& Docencia}

Desde que obtuve mi Maestría en 2005, me he involucrado en los siguientes proyectos:

\subsection*{Departamento de Física Médica, CAB,(2016-Present)}
\begin{itemize}
	\item Líder del grupo de trabajo de Investigación en Técnicas Avanzadas de
		Radioterapia y Dosimetría
	\item Simulaciones basadas en la implementación OpenGate de Geant4 sobre
		Radioterapia de Intensidad Modulada (IMRT).
\end{itemize}

\subsection*{Observatorio Pierre Auger (2006-Presente)}

{\small{\textit{Ver \href{http://www.auger.org/}{www.auger.org}}}}
\begin{itemize}
% \item Miembro de la Colaboración Internacional Auger desde el año 2006.
\item {\bf{Líder de Grupo de Trabajo ``Cosmo-Geophysics'' del Observatorio Pierre Auger}}
\item Análisis de datos del arreglo de detectores de superficie (SD) del Observatorio.
\item Física de Lluvias Atmosféricas Extendidas
\item Desarrollo de la cadena de reconstrucción de eventos registrados por el detector SD.
\item Desarrollo y aplicaciones de los modos de bajas energías (modo ``scaler'' y modo ``histograma'') para el estudio de eventos astrofísicos transitorios (GRBs y eventos Forbush), y sobre la modulación a corto y largo plazo del flujo de rayos cósmicos galácticos debida a la actividad solar.
\item Simulaciones del detector y de rayos cósmicos para la determinación de la respuesta de los detectores water-Cherenkov en los modos de baja energía.
\item Análisis de datos del sistema de monitoreo atmosférico del Observatorio.
\end{itemize}

\subsection*{Proyecto LAGO (Latin American Giant Observatory) (2006-Presente)}
% {\emph{Declarado de interés científico, académico y social por la Legislatura de la Provincia de Río Negro, Declaración 42/2010.}}
{\small{\textit{Ver \href{http://lagoproject.org}{http://lagoproject.org}}}}
\begin{itemize}
  \item {\bf{Investigador Principal del Proyecto LAGO, período 2013-2016}}
% \item Representante de contacto por Argentina frente a la Colaboración LAGO en el período 2012-2013
% \item Miembro de la Colaboración Internacional LAGO desde 2006
% \item Estudio de variables atmosféricas y climáticas registradas por los detectores del proyecto LAGO
\item Diseño y puesta en ejecución de la organización actual del Proyecto
\item Diseño y coordinación del programa de meteorología espacial del Proyecto
\item Simulaciones y análisis de datos para la detección de eventos transitorios (GRBs y eventos Forbush), radiación de fondo y física de la atmósfera.
\item Investigación, desarrollo y construcción de detectores tipo Cherenkov en agua en el la Universidad Industrial de Santander y en el Centro Atómico Bariloche. Uno de ellos será instalado en la Península Antártica. 
\item Diseño y coordinación del experimento ``Determinación de la Vida Media del Muón en Agua'', hecho por los estudiantes de grado del Instituto Balseiro.
\end{itemize}

\subsection*{Cherenkov Telescope Array (CTA) (2010-2014)}
{\small{\textit{Ver \href{http://www.cta-observatory.org}{www.cta-observatory.org}}}}
\begin{itemize}
%\item Miembro del consorcio CTA desde el año 2010.
\item Caracterización del sitio San Antonio de los Cobres y Leoncito. %, propuesto por la colaboración Argentina como candidato para la instalación del proyecto CTA.
\item Investigación y desarrollo de una estación autónoma y remota para el control y la adquisición de datos de una estación meteorológica y un medidor de calidad del cielo, instalados en la localidad de San Antonio de los Cobres, Salta, Argentina.
\end{itemize}

\subsection*{Laboratorio Subterráneo ANDES (2011-2013, 2015-2016)}
{\small{\textit{Ver \href{http://www.andeslab.org}{www.andeslab.org}}}}
\begin{itemize}
\item Estimación del fondo de radiación esperado en el laboratorio subterráneo ANDES debido a la radiactividad natural y al flujo de muones atmosféricos de alta energía.
\end{itemize}

\subsection*{Docencia (2009-Presente)}
{\small{\textit{Ver \href{http://www.ib.edu.ar}{www.ib.edu.ar}, \href{http://www.uis.edu.co}{www.uis.edu.co}, \& \href{http://www.unrn.edu.ar}{www.unrn.edu.ar}}}}
\begin{itemize}
	\item Profesor Asociado, cursos de: ``Física Moderna A'', ``Física IA'', ``Física II B'', y ``Astrofísica y Cosmología''; del Profesorado de Física, Sede Andina, Universidad Nacional de Río Negro (UNRN)
\item Profesor Adjunto en los cursos ``Mecánica Teórica'' (posgrado) y ``Astronomía Planetaria'' de la Escuela de Física, Universidad Industrial de Santander (UIS).
\item Profesor Cátedra en los cursos ``Introducción a la Física'',``Introducción a Física de Partículas'' y ``Mecánica Teórica'', para la Carrera de Física, Escuela de Física, Universidad Industrial de Santander (UIS).
\item Diseño y participación en el ``Diplomado en Astronomía, Astrofísica y Ciencias Espaciales'' de la Escuela de Física de la UIS (Inicio Setiembre 2014).
\item Diseño y dictado del curso ``Astroclima y la problemática del Cambio Climático'', orientado a Profesores de Escuelas y Bachilleratos, Universidad Industrial de Santander, Bucaramanga, Marzo de 2014
\item Miembro de las cátedras de ``Experimental III'' del Instituto Balseiro (UNC), a cargo del experimento de física de rayos cósmicos de baja energía, utilizando el detector Nahuelito del proyecto LAGO.
\item Miembro de las cátedra de Introducción a ``Física de Partículas,y Nuclear y Dosimetría'' del Instituto Balseiro (UNC).
\item Docente categoría III en el programa de incentivos a Docentes Investigadores SPU/ME (convocatoria 2015).
\item Miembro del Comité Académico de la Maestría en Física Médica del Instituto Balseiro, Universidad Nacional de Cuyo.
\end{itemize}
\fi
