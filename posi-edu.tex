\ifeng
\section*{Current Positions}
\begin{itemize}
\item \emph{Post-doctoral position} at Grupo de Investigación en Relatividad y Gravitación and at Grupo Halley de Astronomía y Ciencias Aeroespaciales, Physics Department, Universidad Industrial de Santander, Bucaramanga, Colombia.
\item \emph{Assistant Professor} at Physics Department, Universidad Industrial de Santander, Bucaramanga, Colombia.
\item \emph{Permanent Position} at Laboratorio de Detección de Partículas y Radiación, Gerencia de Tecnología e Investigación en Altas Energías (Technology and Research in High Energy Physics Department), Bariloche Atomic Centre, National Commission of Atomic Energy (CNEA)
\end{itemize}
\else
\section*{Posiciones actuales}
\begin{itemize}
\item \emph{Estadía post-doctoral} en el Grupo de Investigación en Relatividad y Gravitación y en el Grupo Halley de Astronomía y Ciencias Aeroespaciales, Escuela de Física, Universidad Industrial de Santander, Bucaramanga, Colombia. Reconocido como Investigador categoría junior en la convocatoria COLCIENCIAS 640/2013.

\item \emph{Profesor Cátedra} en la Escuela de Física, Universidad Industrial de Santander, Bucaramanga, Colombia.

\item \emph{Posición permanente} en el Laboratorio de Detección de Partículas y Radiación, Gerencia de Tecnología e Investigación en Altas Energías, Centro Atómico Bariloche (CAB), Comisión Nacional de Energía Atómica (CNEA).\\

\end{itemize}
\fi

\ifeng
\section*{Education}
\noindent
\years{2012}\textsc{Doctor in Physics (Ph.D.)}\\
{\emph{Institution}}: Particles and Fields Group, Bariloche Atomic Centre - Instituto Balseiro, CNEA-UNC. {\emph{Thesis}}: The Water Cherenkov Detectors of the Pierre Auger Observatory and their Application to the Study of Background Radiation. {\emph{Advisor}}: Dr. Ingomar Allekotte.
\years{2005}\textsc{Master in Science, Physics}\\
{\emph{Orientation}}: High Energy Physics. {\emph{Institution}}: Particles and Fields Group, Instituto Balseiro, Bariloche Atomic Centre (CNEA-UNC). {\emph{Thesis}}: Event Reconstruction with the Surface Detectors of the Pierre Auger Observatory. {\emph{Advisor}}: Dr. Ingomar Allekotte\\
\years{2004}\textsc{``Licenciado'' in Physics}\\
{\emph{Institution}}: Instituto Balseiro, Bariloche Atomic Centre (CNEA-UNC)\\
\else
\section*{Educación}
\noindent
\years{2012}\textsc{Doctor en Física}\\
{\emph{Institución}}: Grupo de Partículas y Campos, Instituto Balseiro, Centro Atómico Bariloche (CNEA-UNC). {\emph{Tesis}}: Los Detectores Cherenkov del Observatorio Pierre Auger y su Aplicación al Estudio de Fondos de Radiación. {\emph{Director}}: Dr. Ingomar Allekotte\\
\years{2005}\textsc{Magister en Ciencias Físicas}\\
{\emph{Orientación}}: Física de Partículas y Campos. {\emph{Institución}}: Grupo de Partículas y Campos, Instituto Balseiro, Centro Atómico Bariloche (CNEA-UNC). {\emph{Tesis}}: Reconstrucción de eventos con el Detector de Superficie del Observatorio Auger. {\emph{Director}}: Dr. Ingomar Allekotte\\
\years{2004}\textsc{Licenciado en Física (Título equivalente ``Físico'')}\\
{\emph{Institución}}: Instituto Balseiro, Centro Atómico Bariloche (CNEA-UNC)\\
\fi
