%------------------------------------
% Dario Taraborelli
% Typesetting your academic CV in LaTeX
%
% URL: http://nitens.org/taraborelli/cvtex
% DISCLAIMER: This template is provided for free and without any guarantee 
% that it will correctly compile on your system if you have a non-standard  
% configuration.
% Some rights reserved: http://creativecommons.org/licenses/by-sa/3.0/
%------------------------------------

%!TEX TS-program = xelatex
%!TEX encoding = UTF-8 Unicode

%%% TODO %%%

% Agregar subsidios en los cuales participo
% pasar publicaciones a formato "bibtex"
% agregar una sección de desarrollo de prototipos: FOCA/FOG, ASCII, ANDES

\documentclass[11pt, a4paper]{article}
\usepackage{fontspec} 
\usepackage[british]{babel}
\usepackage{etaremune}

% DOCUMENT LAYOUT
\usepackage{geometry} 
\geometry{a4paper, textwidth=5.5in, textheight=9.0in, marginparsep=7pt, marginparwidth=.6in}
\setlength\parindent{0in}

% FONTS
\usepackage[usenames,dvipsnames]{color}
\usepackage{xunicode}
\usepackage{xltxtra}
\defaultfontfeatures{Mapping=tex-text}
\setromanfont [Ligatures={Common}, Numbers={OldStyle}, Variant=01]{Linux Libertine O}
%\setmonofont[Scale=0.8]{Monaco}

% ---- CUSTOM COMMANDS
\chardef\&="E050
\newcommand{\html}[1]{\href{#1}{\scriptsize\textsc{[html]}}}
\newcommand{\pdf}[1]{\href{#1}{\scriptsize\textsc{[pdf]}}}
\newcommand{\doi}[1]{\href{#1}{\scriptsize\textsc{[doi]}}}
% ---- MARGIN YEARS
\usepackage{marginnote}
\newcommand{\amper{}}{\chardef\amper="E0BD }
\newcommand{\years}[1]{\marginnote{\scriptsize #1}}
\renewcommand*{\raggedleftmarginnote}{}
\setlength{\marginparsep}{7pt}
\reversemarginpar

% HEADINGS
\usepackage{sectsty} 
\usepackage[normalem]{ulem} 
\sectionfont{\mdseries\upshape\Large}
\subsectionfont{\mdseries\scshape\normalsize} 
\subsubsectionfont{\mdseries\upshape\large} 

% PDF SETUP
% ---- FILL IN HERE THE DOC TITLE AND AUTHOR
\usepackage[dvipdfm, bookmarks, colorlinks, breaklinks, 
% ---- FILL IN HERE THE TITLE AND AUTHOR
	pdftitle={Hernán Asorey - vita},
	pdfauthor={Hernán Asorey},
]{hyperref}
\hypersetup{linkcolor=blue,citecolor=blue,filecolor=black,urlcolor=MidnightBlue} 
\setlength\parindent{0em}

% Make lists without bullets and compact spacing
\renewenvironment{itemize}{
  \begin{list}{}{
    \setlength{\leftmargin}{0em}
    \setlength{\itemsep}{0.25em}
    \setlength{\parskip}{0pt}
    \setlength{\parsep}{.250em}
  }
}{
  \end{list}
}


% DOCUMENT
\begin{document}
\begin{center}{\huge \bf Hernán Asorey}\\[1cm]\end{center}
\begin{minipage}[t]{0.495\textwidth}
  Centro Atómico Bariloche\\
  Department of Technology and\\
  Research in High Energy Physics\\
  Av. E. Bustillo 9500\\
  (8400) San Carlos de Bariloche\\
  Río Negro, Argentina\\[.2cm]
\end{minipage}
\begin{minipage}[t]{0.495\textwidth}
  Phone: (+54-294) 444-5151 ext 38\\
  Fax: (+54-294) 444-5199\\
  Email: \href{mailto:asoreyh@cab.cnea.gov.ar}{asoreyh@cab.cnea.gov.ar}\\
  \href{http://fisica.cab.cnea.gov.ar/particulas/wiki/User:Asoreyh}{Home page}\\
  twitter: \href{https://twitter.com/#!/asoreyh}{@asoreyh}\\
  skype: asoreyh\\
\end{minipage}
\hrule

\section*{Personal Information}
Born in Quilmes, Buenos Aires, Argentina, on February 05$^\mathrm{th}$, 1974 (38 years old)\\
Argentinian, married, two daughters.

%%\hrule\
\section*{Current Positions}
\begin{itemize}
\item \emph{Permanent Position} at Gerencia de Tecnología e Investigación en Altas
Energías (Technology and Research in High Energy Physics Department), Bariloche
Atomic Centre, National Commission of Atomic Energy (CNEA)
\item \emph{Senior Teaching Assistant (Jefe de Trabajos Prácticos)} at
Physics Department of Rio Negro National University (UNRN)
\item \emph{Teaching Assistant} at Physics Department of Instituto Balseiro, Cuyo
National University (UNC)
\end{itemize}

%%\hrule
\section*{Previous positions}
\noindent
\years{2006-2012}Ph.D. student, Instituto Balseiro (UNC).\\
\years{2004-2005}Master in Science, Instituto Balseiro (UNC).\\
\years{2002-2004}Physics undergraduate student, Instituto Balseiro (UNC).\\
\years{1994-1995}Teaching assistant at Physics Department, Engineering Faculty,
University of Buenos Aires.\\
\years{1992-1996}Industrial Engineering (first four of five years). University
of Buenos Aires.\\
\years{1992-2001}AIM S.A., metal mechanical industry, R+D department on
industrial projects, Buenos Aires, Argentina.\\

%\hrule
\section*{Education}
\noindent
%\years{2012}\textsc{Ph.D. in Physics}\\
%{\emph{Institution}}: Bariloche Atomic Centre - Instituto Balseiro, CNEA-UNC\\
%{\emph{Thesis}}: Present and Future Applications for the Water-Cherenkov Detectors of the Pierre Auger Observatory. 
%{\emph{Advisor}}: Dr. Ingomar Allekotte 
\years{2005}\textsc{Master in Science, Physics}\\
{\emph{Orientation}}: Fields and particle physics\\
{\emph{Institution}}: Instituto Balseiro, Bariloche Atomic Centre (CNEA-UNC)\\
{\emph{Thesis}}: Event Reconstruction with the Surface Detectors of the Pierre
Auger Observatory\\
{\emph{Advisor}}: Dr. Ingomar Allekotte\\ 
\years{2004}\textsc{``Licenciado'' in Physics}\\
{\emph{Institution}}: Instituto Balseiro, Bariloche Atomic Centre (CNEA-UNC)\\

%\hrule
\section*{Honours, Awards \& Fellowships}
\noindent
\years{2011}Balseiro Foundation ``Best Teacher Award'' for outstanding teaching
skills at Instituto Balseiro.\\
\years{2008-2010}Fellowship awarded by the National Council of Scientific and
Technical Investigations (CONICET) to obtain a Ph.D. degree.\\
\years{2006-2007}Fellowship awarded by the Balseiro Foundation and the National
Commission of Atomic Energy (FUNC-CNEA).\\
\years{2005}Fellowship awarded by the National Commission of Atomic Energy
(CNEA) to obtain a Master degree in Physics.\\
\years{2002-2004}Fellowship awarded by the National Commission of Atomic
Energy (CNEA) to obtain a Master to study ``Licenciatura en Física'' at
Instituto Balseiro.\\

\section*{Research \& Teaching Activities}

Since I have earned my master degree in December 2005, I have been involved in
the following projects:

\subsection*{Pierre Auger Observatory}

{\small{\textit{See \href{http://www.auger.org/}{www.auger.org}}}}
\begin{itemize}
\item Member of the Pierre Auger Collaboration since 2006
\item Ultra High-Energy Cosmic Rays Physics 
\item Data analysis of the Surface Detector
\item Development of the reconstruction event chain of the Surface Detector
\item Development and applications of the low energy modes (scaler and histogram
modes) of the surface detectors of the Pierre Auger Observatory, for the study
of transient events (Gamma Ray Bursts and Forbush events), and short and long
term modulation of the galactic cosmic rays flux due to solar activity
\item CORSIKA and detector simulations, oriented to determine the
water-Cherenkov response working in the low energy modes
\item Data analysis of the weather monitoring system of the Pierre Auger
Observatory
\end{itemize}

\subsection*{Large Aperture Grb Observatory (LAGO)}
{\emph{Declared of Scientific, Academic and Social interest by the Honourable
House of Representatives of the Rio Negro Province, Dec. 42/2010.}}\\
{\small{\textit{See
\href{http://fisica.cab.cnea.gov.ar/particulas/laboratorio/lago}
{http://fisica.cab.cnea.gov.ar/particulas/laboratorio/lago}}}}\\
\begin{itemize}
\item Member of the LAGO International Collaboration since 2006
\item Simulations and data analysis for the detection of transient events
(GRB and Forbush events), background radiation and atmospheric physics.
\item Research, development and building of three water-Cherenkov detector
prototypes for the LAGO project at Bariloche Atomic Centre. One of them will be
installed at the Antarctic Peninsula.
\item Design and coordination of the experiment ``Measurement of Muon Lifetime in
Water'', done by undergraduate students at Instituto Balseiro.
\end{itemize}

\subsection*{Cherenkov Telescope Array (CTA)}
{\small{\textit{See \href{http://www.cta-observatory.org}{www.cta-observatory.org}}}}
\begin{itemize}
\item Member of the CTA consortium since 2010
\item Research and development of the autonomous station for control and data
acquisition of the weather station and sky quality meter installed in San
Antonio de los Cobres, Argentina, one of the site candidates for the CTA
observatory.
\end{itemize}

\subsection*{ANDES Underground Laboratory}
{\small{\textit{See \href{http://www.andeslab.org}{www.andeslab.org}}}}
\begin{itemize}
\item Estimation and measurements of the expected backgrounds at the ANDES
underground lab due to natural radioactivity and high energy atmospheric muons
\end{itemize}

\subsection*{Teaching}
{\small{\textit{See \href{http://www.ib.edu.ar}{www.ib.edu.ar} and
\href{http://www.unrn.edu.ar}{www.unrn.edu.ar}}}}
\begin{itemize}
\item Teaching assistant, Experimental Physics III and Introduction to nuclear
and particle physics courses, Instituto Balseiro (UNC)
\item Senior teaching assistant, Physics I (introductory physics) course, UNRN.
\end{itemize}

\section*{Publications}
\noindent
\subsection*{Selected Works}
\noindent
During the development of my work within the Pierre Auger Observatory, I have
been acting as the Physics coordinator and responsible of one of the full
author list papers of the Pierre Auger Collaboration (The Pierre Auger
Collaboration, JINST {\bf 6} P01003--P01020 (2011)), using the surface detector
in a novel way, developed during my Ph.D. thesis, as a tool to study transient
solar phenomena and heliospheric modulation of galactic cosmic rays flux. 

This list is a personal selection of the works I have been directly involved:

\begin{etaremune}
\item \years{2012}H. Asorey and A. López Dávalos, {\emph{Fermi Problem: Power
developed at the eruption of the Puyehue-Cordón Caulle volcanic system in June
2011}}, Amer. Jour. Phys., {\emph{submitted}}, (2012).
\href{http://arxiv.org/abs/1109.1165}{arXiv:1109.1165v1}[physics.ed-ph]
Selected as the best arXiv paper of September 2011 by the
\href{http://www.technologyreview.com/blog/arxiv/27140/}{M.I.T. Technology
Review Physics arXiv Blog}, Sept. 2011.

\item \years{2012}S. Dasso and H. Asorey, for the Pierre Auger Collaboration,
\href{http://dx.doi.org/10.1016/j.asr.2011.12.028}{\emph{ The scaler mode in
the Pierre Auger Observatory to study heliospheric modulation of cosmic rays
}}, Adv. Space Res. {\bf{49}} (11), 1563--1569 (2012)

\item \years{2012} H. Asorey, M. Arribere, X. Bertou, M. Gómez Berisso, F. Sánchez,
{\emph{Expected Backgrounds at the ANDES Underground Laboratory}}
plenary talk given at the Third International Workshop for the Design of the
ANDES Underground Laboratory, Valparaiso, Chile, 11--12 Jan 2012.

\item \years{2011}The Pierre Auger Collaboration,
\href{http://dx.doi.org/10.1088/1748-0221/6/01/P01003}{\emph{The Pierre Auger
Observatory Scaler Mode for the Study of the Modulation of Galactic Cosmic Rays
due to Solar Activity}}, JINST {\bf 6} P01003--
P01020 (2011).
$^*${\bf{Coordinator}}

\item \years{2011}I. Allekotte, H. Arnaldi, H. Asorey, X. Bertou, M. Gómez Berisso,
M. Sofo Haro, {\emph{Development of ultra-fast and ultra low power consumption
electronics in the Bariloche Particle and Radiation Detection Laboratory}},
poster presentation in the 96$^{\mathrm{th}}$ National Reunion SUF-AFA2011 of the Argentinian
Physics Association, Montevideo, Uruguay, 20--23 Sept 2011.

\item \years{2011}H. Asorey[Pierre Auger Collaboration], {\emph{Low energy radiation
measurements with the water Cherenkov detector array of the Pierre Auger
Observatory}}, in Prom. 32H International Cosmic Ray Conference, vol. 11
462--465, Beijing, China, 11--18 Ago 2011

\item \years{2010}J. Bl\"umer and The Pierre Auger Collaboration,
\href{http://dx.doi.org/10.1088/1367-2630/12/3/035001}{\emph{The Northern Site
of the Pierre Auger Observatory}}, Journal of Physics {\bf 12} (3) 035001

\item \years{2010}The Pierre Auger Collaboration,
\href{http://dx.doi.org/10.1016/j.physletb.2010.02.013}{\emph{Measurement of
the energy spectrum of cosmic rays above $10^{18}$ eV using the Pierre Auger
Observatory}}, Phys. Lett. {\bf B685} 239--246 (2010),\\
\href{http://arxiv.org/abs/1002.1975}{arXiv:1002.1975v1}[astro-ph.HE]

\item \years{2010}The Pierre Auger Collaboration,
\href{http://dx.doi.org/10.1016/j.nima.2009.11.018}{\emph{Trigger and Aperture
of the Surface Detector Array of the Pierre Auger Observatory}}, NIM {\bf A613}
29--39, (2010)

\item \years{2010}H. Asorey[LAGO Collaboration], {\emph{The Large Aperture Gamma Ray
Burst Observatory (LAGO)}}, plenary talk in the 3$^{\mathrm{rd}}$ International Workshop of
High Energy Physics in the LHC Era HEP2010, Valparaiso, Chile, 4--8 Jan 2010.

\item \years{2009}H. Asorey[Pierre Auger Collaboration], {\emph{Cosmic Ray Solar
Modulation Studies at the Pierre Auger Observatory}}, in Proc. 31th
International Cosmic Ray Conference, Lodz, Poland, 8--15 Jul 2009.

\item \years{2009} The Pierre Auger Collaboration,
\href{http://dx.doi.org/10.1016/j.astropartphys.2009.06.004}{\emph{Atmospheric
effects on extensive air showers observed with the Surface Detector of the
Pierre Auger Observatory}}, Astropart. Phys. {\bf 32}, 89--99, (2009),
\href{http://arxiv.org/abs/0906.5497/}{arXiv:0906.5497v2}[astro-ph.IM]

\item \years{2008}D. Allard {\emph et al.} [LAGO Collaboration],
\href{http://dx.doi.org/10.1016/j.nima.2008.07.041}{\emph{Use of
water-Cherenkov detectors to detect Gamma Ray Bursts at the Large Aperture GRB
Observatory (LAGO)}}, NIM {\bf A595} 70--72 (2008)

\item \years{2007}D. Allard {\emph et al.} [LAGO Collaboration], {\emph{Looking for
the high energy component of GRBs at the Large Aperture GRB Observatory}}, in
Proc. 30$^{\mathrm{th}}$ International Cosmic Ray Conference,  Mérida, Mexico, 3-11 Jul
2007.

\item \years{2007}The Pierre Auger Collaboration,
\href{http://dx.doi.org/10.1016/j.astropartphys.2006.11.002}{\emph{Anisotropy
studies around the galactic centre at EeV energies with the Auger
Observatory.}},  Astropart. Phys. {\bf 27} 244--253 (2007)

\item \years{2006}D. Allard {\emph et al.} [LAGO Collaboration], {\emph{The Large
Aperture GRB aperture}}, in Proc. of the Observational Astronomy in Argentina
Workshop, Buenos Aires.

\end{etaremune}

\subsection*{Complete list of Journal papers}

\begin{etaremune}
\item \years{2012}H. Asorey and A. López Dávalos, {\emph{Fermi Problem: Power developed
at the eruption of the Puyehue-Cordón Caulle volcanic system in June 2011}},
Amer. Jour. Phys., {\emph{submitted}}, (2012)
\href{http://arxiv.org/abs/1109.1165}{arXiv:1109.1165v1}[physics.ed-ph]

\item \years{2012}S. Dasso and H. Asorey, for the Pierre Auger Collaboration,
\href{http://dx.doi.org/10.1016/j.asr.2011.12.028}{\emph{ The scaler mode in
the Pierre Auger Observatory to study heliospheric modulation of cosmic rays
}}, Adv. Space Res. {\bf{49}} (11), 1563--1569 (2012)

\item \years{2012}The Pierre Auger Collaboration,
\href{http://dx.doi.org/10.1007/s10686-011-9247-0}{\emph{Design concepts for
the Cherenkov Telescope Array CTA: an advanced facility for ground-based
high-energy gamma-ray astronomy}}, Exper. Astron. {\bf{32}} (3), 193--316
(2012)

\item \years{2012}The Pierre Auger Collaboration, 
\href{http://dx.doi.org/10.1016/j.astropartphys.2011.12.002}{\emph{Description
of atmospheric conditions at the Pierre Auger Observatory using the Global Data
Assimilation System (GDAS)}}, Astropart. Phys. {\bf{35}} (9), 591--607 (2012)

\item \years{2012}The Pierre Auger Collaboration, 
\href{http://dx.doi.org/10.1088/1475-7516/2011/11/022}{\emph{The effect of the
geomagnetic field on cosmic ray energy estimates and large scale anisotropy
searches on data from the Pierre Auger Observatory}}, JCAP {\bf{2011}} (022),
1--23 (2012)

\item \years{2012}The Pierre Auger Collaboration, 
\href{http://dx.doi.org/10.1016/j.astropartphys.2011.10.004}{\emph{Search for
signatures of magnetically-induced alignment in the arrival directions measured
by the Pierre Auger Observatory}}, Astropart. Phys. {\bf{35}} (6), 354--361
(2012)

\item \years{2011}The Pierre Auger Collaboration,
\href{http://dx.doi.org/10.1016/10.1103/PhysRevD.84.122005}{\emph{Search for
Ultra-High Energy Neutrinos in Highly Inclined Events at the Pierre Auger
Observatory}}, Phys.  Rev. {\bf D84}, 122005, 1--16 (2011)
\href{http://arxiv.org/abs/1202.1493}{arXiv:1202.1493}[astro-ph.HE]

\item \years{2011}The Pierre Auger Collaboration, 
\href{http://dx.doi.org/10.1016/j.astropartphys.2011.08.001}{\emph{The Lateral
Trigger Probability function for UHE Cosmic Rays Showers detected by the Pierre
Auger Observatory}}, Astropart. Phys. {\bf{35}} (5), 266--276 (2011)

\item \years{2011}The Pierre Auger Collaboration,
\href{http://dx.doi.org/10.1088/1475-7516/2011/06/022}{\emph{Anisotropy and
chemical composition of ultra-high energy cosmic rays using arrival directions
measured by the Pierre Auger Observatory}}, JCAP {\bf 06} 022 (2011),
\href{http://arxiv.org/abs/1106.3048}{arXiv:1101.3048v1}[astro-ph.HE]

\item \years{2011}The Pierre Auger Collaboration,
\href{http://dx.doi.org/10.1016/j.nima.2011.01.049}{{\emph{Advanced
functionality for radio analysis in the Offline software framework of the
Pierre Auger Observatory}}}, NIM {\bf A635} 92--102
(2011),\\
\href{http://arxiv.org/abs/1101.4473}{arXiv:1101.4473v1}[astro-ph.HE]

\item \years{2011}The Pierre Auger Collaboration,
\href{http://dx.doi.org/10.1016/j.astropartphys.2010.12.007}{\emph{Search for
First Harmonic Modulation in the Right Ascension Distribution of Cosmic Rays
Detected at the Pierre Auger Observatory}}, Astropart. Phys. {\bf 34} 627--639
(2011)

\item \years{2011}The Pierre Auger Collaboration,
\href{http://dx.doi.org/10.1088/1748-0221/6/01/P01003}{\emph{The Pierre Auger
Observatory Scaler Mode for the Study of the Modulation of Galactic Cosmic Rays
due to Solar Activity}}, JINST {\bf 6} P01003--
P01020 (2011). $^*${\bf{Coordinator}}

\item \years{2010}The Pierre Auger Collaboration,
\href{http://dx.doi.org/10.1016/j.astropartphys.2010.10.001}{\emph{The exposure
of the hybrid detector of the Pierre Auger Observatory}}, Astropart. Phys. {\bf
34}, 368--381 (2011)

\item \years{2010}The Pierre Auger Collaboration,
\href{http://dx.doi.org/10.1016/j.astropartphys.2010.08.010}{\emph{Update on
the correlation of the highest energy cosmic rays with nearby extragalactic
matter}},Astropart. Phys. {\bf 34}, 314--326 (2010),
\href{http://arxiv.org/abs/1009.1855}{arXiv:1009.1855v2}[astro-ph.HE]

\item \years{2010}The Pierre Auger Collaboration,
\href{http://dx.doi.org/10.1016/j.nima.2010.04.023}{\emph{The Fluorescence
Detector of the Pierre Auger Observatory}}, NIM {\bf A620}, 227 (2010),
\href{http://arxiv.org/abs/0907.4282}{arXiv:0907.4282v1}[astro-ph.IM]

\item \years{2010}J. Bl\"umer and The Pierre Auger Collaboration,
\href{http://dx.doi.org/10.1088/1367-2630/12/3/035001}{\emph{The Northern Site
of the Pierre Auger Observatory}}, Journal of Physics {\bf 12} (3) 035001
(2010)

\item \years{2010}The Pierre Auger Collaboration,
\href{http://dx.doi.org/10.1016/j.astropartphys.2009.12.005}{\emph{A Study of
the Effect of Molecular and Aerosol Conditions in the Atmosphere on Air
Fluorescence Measurements at the Pierre Auger Observatory}}, Astropart. Phys.
{\bf 33}, 108--129 (2010),
\href{http://arxiv.org/abs/0907.4282}{arXiv:1002.0366v1}[astro-ph.HE]

\item \years{2010}The Pierre Auger Collaboration,
\href{http://dx.doi.org/10.1016/j.physletb.2010.02.013}{\emph{Measurement of
the energy spectrum of cosmic rays above $10^{18}$ eV using the Pierre Auger
Observatory}}, Phys. Lett. {\bf B685} 239--246 (2010),
\href{http://arxiv.org/abs/1002.1975}{arXiv:1002.1975v1}[astro-ph.HE]

\item \years{2010}The Pierre Auger Collaboration,
\href{http://dx.doi.org/10.1103/PhysRevLett.104.091101}{\emph{Measurement of
the Depth of Maximum of Extensive Air Showers above 10$^{18}$ eV}}, PRL {\bf
104} 091101
(2010)\href{http://arxiv.org/abs/1002.0699}{arXiv:1002.0699v1}[astro-ph.HE]

\item \years{2010}The Pierre Auger Collaboration,
\href{http://dx.doi.org/10.1016/j.nima.2009.11.018}{\emph{Trigger and Aperture
of the Surface Detector Array of the Pierre Auger Observatory}}, NIM {\bf A613}
29--39, (2010)

\item \years{2009} The Pierre Auger Collaboration,
\href{http://dx.doi.org/10.1016/j.astropartphys.2009.06.004}{\emph{Atmospheric
effects on extensive air showers observed with the Surface Detector of the
Pierre Auger Observatory}}, Astropart. Phys. {\bf 32}, 89--99, (2009),
\href{http://arxiv.org/abs/0906.5497/}{arXiv:0906.5497v2}[astro-ph.IM]

\item \years{2009}The Pierre Auger Collaboration,
\href{http://dx.doi.org/10.1016/j.astropartphys.2009.04.003}{\emph{Upper limit
on the cosmic-ray photon fraction at EeV energies from the Pierre Auger
Observatory.}}, Astropart. Phys. {\bf 31} 399--406 (2009) 
\href{http://arxiv.org/abs/0903.1127/}{arXiv:0903.1127v1} [astro-ph.HE]

\item \years{2009}The Pierre Auger Collaboration,
\href{http://dx.doi.org/10.1103/PhysRevD.79.102001}{\emph{Limit on the diffuse
flux of ultra-high energy tau neutrinos with the surface detector of the Pierre
Auger Observatory.}}, Phys. Rev. {\bf D79} 10:1--15
(2009)\href{http://arxiv.org/abs/0903.3385/}{arXiv:0903.3385v1}[astro-ph.HE]

\item \years{2008}D. Allard {\emph et al.} [LAGO Collaboration],
\href{http://dx.doi.org/10.1016/j.nima.2008.07.041}{\emph{Use of
water-Cherenkov detectors to detect Gamma Ray Bursts at the Large Aperture GRB
Observatory (LAGO)}}, NIM {\bf A595} 70--72 (2008)

\item \years{2008}The Pierre Auger Collaboration,
\href{http://dx.doi.org/10.1103/PhysRevLett.101.061101}{\emph{Observation of
the Suppression of the Flux of Cosmic Rays above $4\times10^{19}$\,eV.}}, PRL
{\bf 101} 061101 (2008)

\item \years{2008}The Pierre Auger Collaboration,
\href{http://dx.doi.org/10.1103/PhysRevLett.100.211101}{\emph{Upper limit on
the diffuse flux of UHE tau neutrinos from the Pierre Auger Observatory.}}, PRL
{\bf 100} 21101 (2008)

\item \years{2008}The Pierre Auger Collaboration,
\href{http://dx.doi.org/10.1016/j.astropartphys.2008.01.003}{\emph{Upper limit
on the cosmic-ray photon flux above 10$^{19}$\,eV using the surface detector of
the Pierre Auger Observatory.}}, Astropart. Phys. {\bf 29} 243--256 (2008)

\item \years{2008}The Pierre Auger Collaboration,
\href{http://dx.doi.org/10.1016/j.astropartphys.2008.01.002}{\emph{Correlation
of the highest-energy cosmic rays with the positions of nearby active galactic
nuclei.}}, Astropart. Phys. {\bf 29} 188--204 (2008)

\item \years{2007}The Pierre Auger Collaboration,
\href{http://dx.doi.org/10.1126/science.1151124}{\emph{Correlation of the
highest energy cosmic rays with nearby extragalactic objects.}}, Science {\bf
318} 939--943 (2007)

\item \years{2007}The Pierre Auger Collaboration,
\href{http://dx.doi.org/10.1016/j.astropartphys.2006.11.002}{\emph{Anisotropy
studies around the galactic centre at EeV energies with the Auger
Observatory.}},  Astropart. Phys. {\bf 27} 244--253 (2007)

\item \years{2007}The Pierre Auger Collaboration,
\href{http://dx.doi.org/10.1016/j.astropartphys.2006.10.004}{\emph{An upper
limit to the photon fraction in cosmic rays above 10$^{19}$\,eV from the Pierre
Auger Observatory.}}, Astropart. Phys. {\bf 27} 155--168 (2007)
\end{etaremune}

\subsection*{Participation \& presentations at Schools \& Conferences}
\noindent
\begin{etaremune}
\item \years{2012} H. Asorey, M. Arribere, X. Bertou, M. Gómez Berisso, F. Sánchez, 
{\emph{Expected Backgrounds at the ANDES Underground Laboratory}}
plenary talk given at the Third International Workshop for the Design of the
ANDES Underground Laboratory, Valparaiso, Chile, 11--12 Jan 2012.

\item \years{2012}H. Asorey [Pierre Auger Collaboration], {\emph{Heliospheric
Modulation of Cosmic Rays Observed by the Pierre Auger Observatory and the LAGO
Project}}, parallel talk at the 4$^{\mathrm{th}}$ International Workshop of High Energy
Physics in the LHC Era HEP2012, Valparaiso, Chile, 4--10 Jan 2012.

\item \years{2011}H. Asorey, {\emph{Fermi Problem: Power developed at the eruption of
the Puyehue-Cordón Caulle volcanic system in June 2011}}, invited talk in the
96$^{\mathrm{th}}$ National Reunion SUF-AFA2011 of the Argentinian Physics Association,
Montevideo, Uruguay, 20--23 Sept 2011.

\item \years{2011}I. Allekotte, H. Arnaldi, H. Asorey, X. Bertou, M. Gómez Berisso,
M. Sofo Haro, {\emph{Development of ultra fast and ultra low power consumption
electronics in the Bariloche Particle and Radiation Detection Laboratory}},
poster presentation in the 96$^{\mathrm{th}}$ National Reunion SUF-AFA2011 of the Argentinian
Physics Association, Montevideo, Uruguay, 20--23 Sept 2011.

\item \years{2011}H. Asorey[Pierre Auger Collaboration], {\emph{Low energy radiation
measurements with the water Cherenkov detector array of the Pierre Auger
Observatory}}, in Proc. 32th International Cosmic Ray Conference, vol. 11
462--465, Beijing, China, 11--18 Ago 2011

\item \years{2011}The Pierre Auger Collaboration,
\href{http://arxiv.org/abs/1107.4805}{\emph{The Pierre Auger Observatory III:
Other Astrophysical Observations}}, en Proc. 32th International Cosmic Ray
Conference, Beijing, China, 11--18 Ago 2011.

\item \years{2010}H. Asorey[Pierre Auger Collaboration],
\href{http://95rnf.afa.webfactional.com/tex\_files/Resumenes/DPyC/PyC\_6.pdf}{\emph{The
infill array of the Pierre Auger Observatory}}, talk given in the Particle and
Fields Division in the 95$^{\mathrm{th}}$ National Reunion AFA2011 of the Argentinian Physics
Association, Malargüe, Argentina, 28 Sept--01 Oct 2010.

\item \years{2010}H. Asorey, J. Castro, A. López Dávalos,
\href{http://95rnf.afa.webfactional.com/tex\_files/Resumenes/EF/asorey.pdf}{\emph{Kepler,
Newton, Feynman}}, poster presentation in the 95$^{\mathrm{th}}$ National Reunion AFA2011 of
the Argentinian Physics Association, Malargüe, Argentina, 28 Sept--01 Oct 2010.

\item \years{2010}H. Asorey[LAGO Collaboration], {\emph{The Large Aperture Gamma Ray
Burst Observatory (LAGO)}}, plenary talk in the 3$^{\mathrm{rd}}$ International Workshop of
High Energy Physics in the LHC Era HEP2010, Valparaiso, Chile, 4--8 Jan 2010.

\item \years{2009}H. Asorey[Pierre Auger Collaboration], {\emph{Cosmic Ray Solar
Modulation Studies at the Pierre Auger Observatory}}, in Proc. 31th
International Cosmic Ray Conference, Lodz, Poland, 8--15 Jul 2009.

\item \years{2009}The Pierre Auger Collaboration,
\href{http://arxiv.org/abs/0906.2347}{\emph{Astrophysical Sources of Cosmic
Rays and Related Measurements with the Pierre Auger Observatory}}, en Proc.
31th International Cosmic Ray Conference, Lodz, Poland, 8--15 Jul 2009.

\item \years{2009}The LAGO Collaboration,
\href{http://arxiv.org/abs/0906.0816}{\emph{The Large Aperture GRB
Observatory}}, en Proc. 31th International Cosmic Ray Conference, Lodz,
Poland, 8--15 Jul 2009.

\item \years{2009}H. Asorey[Pierre Auger Collaboration], {\emph{The Acceptance of the
Pierre Auger Observatory}}, poster presentation in the VII Latinoamerican
Symposium of High Energy Physics (SILAFAE), San Carlos de Bariloche, Argentina,
14-21 Jan 2009.

\item \years{2008}XVI Course of the ISCRA (International School of Cosmic Ray
Astrophysics) 2008: ``Gamma Ray and Cosmic Ray Astrophysics: From below GeV to
beyond EeV Energies'', Erice, Italia, Julio 2008

\item \years{2008} Invited talk``Towards Cosmic ray Solar Modulation Studies'',
University of Siegen, Siegen, Germany, 2008.

\item \years{2007}D. Allard {\emph et al.} [LAGO Collaboration], {\emph{Looking for
the high energy component of GRBs at the Large Aperture GRB Observatory}}, in
Proc. 30$^{\mathrm{th}}$ International Cosmic Ray Conference,  Mérida, Mexico, 3-11 Jul
2007.

\item \years{2007}IV Latin American School of Strings LASS 07, San Carlos de
Bariloche, January 2007.

\item \years{2006}H. Asorey[Pierre Auger Collaboration], {\emph{The Surface Detector
Array of the Pierre Auger Observatory}}, parallel talk in the 1$^{\mathrm{st}}$ International
Workshop of High Energy Physics in the LHC Era HEP2006, Valparaiso, Chile,
12--17 Dec 2006.

\item \years{2006}D. Allard {\emph et al.} [LAGO Collaboration], {\emph{The Large
Aperture GRB aperture}}, in Proc. of the Observational Astronomy in Argentina
Workshop, Buenos Aires.

\item \years{2005}Third CERN-CLAF Latin American School Of High Energy Physics, CERN,
Malargüe, Argentina. Poster presentation: ``Event Reconstruction using the
Surface Detectors At UHECR Pierre Auger Observatory''

\item \years{2004}Sixth J. J. Giambiagi Winter School on Particle Physics, Facultad
de Ciencias Exactas y Naturales, University of Buenos Aires. July 2004.

\item \years{2005-2012} Sixteen technical and physics talks given at the Pierre Auger
Collaboration meetings, Malargüe, Argentina.
\end{etaremune}

\section*{Internal notes of the Pierre Auger Observatory (GAP Notes)}

See \href{http://www.auger.org/admin-cgi-bin/woda/gap\_notes.pl/Search?search=asorey}
{www.auger.org/admin/GAP\_Notes}.\\

\begin{etaremune}

\item \years{2011} R. Ravignani, H. Asorey, D. Melo, G. De La Vega, A. Etchegoyen, A.
Ferrero, R. F. Gamarra, B. García, M. Josebachuili, F. Sánchez, I. Sidelnik, A.
Tapia, B. Wundheiler, {\emph{Observation of the spectrum with the AMIGA
infill}}, Note 2011-010.
\item \years{2009}H. Asorey, I. Allekotte, X. Bertou, M. Gómez~Berisso,
{\emph{Acceptance of generalised Surface Detector Arrays from real data}}, Note 2009-155.
\item \years{2009}H. Asorey, X. Bertou, D. Thomas, M. Mostafá, {\emph{The OMG
Hybrid Event}}, Note 2011-154.
\item \years{2009}H. Asorey, I. Allekotte, X. Bertou, M.
Gómez~Berisso, {\emph{Determining the acceptance of the Pierre Auger Surface
Detector with the Infill Array}}, Note 2009-112.
\item \years{2009}I. Allekotte, H. Asorey, M. Gómez~Berisso, {\emph{Improving the
determination of the Auger Surface Detector Single Station Trigger Probability
from real data}}, Note 2009-019.
\item \years{2008}H. Asorey, X. Bertou, {\emph{Determining the Dynamic Range
needed for new Surface Detectors.}}, Note 2008-117.
\item \years{2008}I. Allekotte, H. Asorey, X. Bertou, M. Gómez~Berisso,
{\emph{You thought you understood hexagons?}}, Note 2008-114
\item \years{2008}S. Grebe, I. Allekotte, H. Asorey, X. Bertou, P. Buchholz,
{\emph{Robustness of the CDAS reconstruction algorithm.}}, Note 2008-112.
\item \years{2008}H. Asorey, X. Bertou, {\emph{First large timescale analysis of
Auger SD scaler data: Towards cosmic ray Solar modulation studies.}}, Note 2008-072.
\item \years{2007}H. Asorey, I. Allekotte, {\emph{Towards a complete set of
weather data.}}, Note 2007-088.
\item \years{2006}H. Asorey, X. Bertou, E. Roulet, {\emph{How to improve the SD
arrival direction reconstruction by correcting the start-time of individual
detectors.}}, Note 2006-052.
\item \years{2005}H. Asorey, I. Allekotte, M. Gómez~Berisso, X. Bertou,
{\emph{Robustness of the angular reconstruction with the Surface Array of the
Auger Observatory.}}, Note 2005-107.
\item \years{2005}H. Asorey, I. Allekotte, M. Gómez~Berisso, X. Bertou,
{\emph{Robustness of the energy reconstruction with the Surface Array of the
Auger Observatory.}}, Note 2005-084.
\end{etaremune}

\section*{Organising \& other Academic Activities}

\years{2011}Member of the local organising committee of the ``First
International Workshop for the Design of the ANDES Underground Laboratory'',
Constituyentes Atomic Centre, Buenos Aires, Argentina, 11-14 April 2011

\years{2010}Member of the local organising committee of the ``XI ICFA School on
Instrumentation in Elementary Particle Physics'', San Carlos de Bariloche,
Argentina, Jan 2010.

\years{2010}Member of the local organising committee of the ``95$^{\mathrm{th}}$ National
Reunion of the Argentinian Physics Association'', Malargüe, Argentina, Sept-Oct
2010.

\years{2009}Member of the local organising committee of the ``VII
Latinoamerican Symposium of High Energy Physics (SILAFAE)'', San Carlos de
Bariloche, Argentina, January 2009.

\years{2005}
Member of the Instituto Balseiro Academic Council, elected by the Physics
students. 


\section*{Outreach \& Complementary Activities}
\noindent
\years{2011}H. Asorey, A. Clúa, A. López Dávalos
\href{http://www.clarin.com/sociedad/Cien-millones-toneladas-cenizas-solo_0_517148395.html}{One
hundred millions of tons in only one day}, Clarín (national circulation
newspaper), 2011. Reproduced in hundreds of Argentinian and international
newspapers and media. 

\years{2011}H. Asorey, {\emph{Living with a star}}, Solar physics and space
weather phenomena talk, oriented to general public and high-school students of
the Rio Negro Province. Begin: March-2011

\years{2010}{\emph{Work of Bariloche Atomic Centre Scientists distinguished}}
(H. Asorey, X. Bertou, M. Gómez Berisso), El Cordillerano, Bariloche 2000 y
ANBariloche, 2010.

\years{2010}Laura García, {\emph{Latinoamerican network of detectors to study
gamma radiation}} (H. Asorey, X. Bertou, M. Gómez Berisso), El Cordillerano,
Bariloche 2000 and ANBariloche, 2010.

\years{2009}H. Asorey, {\emph{Astrophysics for everyone}}, bimonthly column in
the ``Nature and technology'' local magazine. 

\years{2008}H. Asorey, {\emph{The Pierre Auger Observatory: a look to the
Universe to the highest energies}}, invited talk oriented for general public,
National University of Quilmes, Argentina, April 2008.

\section*{Additional Information}

\begin{itemize}
\item Languages: Spanish (mother tongue); English (speaks, read, write); French
(read), Italian (read and speaks)
\item Computing skills: Linux and Windows operative system. Preferred editor:
VIm.
\item Programming skills: C/C++, Perl, Python, HTML, PHP, SQL, and Bash.
\item Technical computing and data analysis software skills: root, gnuplot,
spyder, Mathematica, AutoCAD design software.
\end{itemize}

\section*{References}

For references of my work, please contact the following persons:

Dr. Alberto Etchegoyen (\href{mailto:alberto.etchegoyen@iteda.cnea.gov.ar}{alberto.etchegoyen@iteda.cnea.gov.ar})

Dr. Ingomar Allekotte (\href{mailto:ingo@cab.cnea.gov.ar}{ingo@cab.cnea.gov.ar})

Dr. Esteban Roulet (\href{mailto:roulet@cab.cnea.gov.ar}{roulet@cab.cnea.gov.ar})

Dr. Xavier Bertou (\href{mailto:bertou@cab.cnea.gov.ar}{bertou@cab.cnea.gov.ar})

Dr. Piera Luisa Ghia (\href{mailto:piera.ghia@lpnhe.in2p3.fr}{piera.ghia@lpnhe.in2p3.fr})

\vspace{2cm}
\begin{flushright}
Hernán Asorey\\
May 2012.
\end{flushright}
\hrule
\end{document}
