\ifeng
\section*{Current Positions}
\begin{itemize}
\item \emph{Permanent Position} at Gerencia de Tecnología e Investigación en Altas Energías (Technology and Research in High Energy Physics Department), Bariloche Atomic Centre, National Commission of Atomic Energy (CNEA)
\item \emph{Senior Teaching Assistant (Jefe de Trabajos Prácticos)} at Physics Department of Rio Negro National University (UNRN)
\item \emph{Teaching Assistant} at Physics Department of Instituto Balseiro, Cuyo National University (UNC)
\end{itemize}
\else
\section*{Posiciones actuales}
\begin{itemize}
\item \emph{Posición permanente} en la Gerencia de Tecnología e Investigación en Altas Energías, Centro Atómico Bariloche (CAB), Comisión Nacional de Energía Atómica (CNEA).\\
\item \emph{Jefe de Trabajos Prácticos, a cargo del dictado de materia} del Área Física, Sede Andina, Universidad Nacional de Río Negro (UNRN).\\
\item \emph{Auxiliar de primera, interino,} del Área Física, Instituto Balseiro, UNC.\\
\end{itemize}
\fi

\ifeng
\section*{Previous positions}
\noindent
\years{2006-2012}Ph.D. student, Instituto Balseiro (UNC).\\
\years{2004-2005}Master in Science, Instituto Balseiro (UNC).\\
\years{2002-2004}Physics undergraduate student, Instituto Balseiro (UNC).\\
\years{1994-1995}Teaching assistant at Physics Department, Engineering Faculty, University of Buenos Aires.\\
\years{1992-1996}Industrial Engineering (first four of five years). University of Buenos Aires.\\
\years{1992-2001}AIM S.A., metal mechanical industry, R+D department on industrial projects, Buenos Aires, Argentina.\\
\else
\section*{Posiciones anteriores}
\noindent
\years{2006-2012}Doctorado en Física, Instituto Balseiro (UNC).\\
\years{2004-2005}Maestría en Ciencias Físicas, Instituto Balseiro (UNC).\\
\years{2002-2004}Licenciatura en Física, Instituto Balseiro (UNC).\\
\years{1994-1995}Auxiliar de Segunda Categoría con dedicación simple, ad honorem, Universidad de Buenos Aires.\\
\years{1992-1996}Ingeniería Industrial (primeros cuatro años). Universidad de Buenos Aires.\\
\years{1992-2001}AIM S.A., metalúrgica industrial, a cargo de diseño y ejecución de proyectos industriales, Bernal, Buenos Aires, Argentina.\\
\fi


\ifeng
\section*{Education}
\noindent
%\years{2012}\textsc{Ph.D. in Physics}\\
%{\emph{Institution}}: Bariloche Atomic Centre - Instituto Balseiro, CNEA-UNC\\
%{\emph{Thesis}}: Present and Future Applications for the Water-Cherenkov Detectors of the Pierre Auger Observatory. 
%{\emph{Advisor}}: Dr. Ingomar Allekotte 
\years{2005}\textsc{Master in Science, Physics}\\
{\emph{Orientation}}: Fields and particle physics\\
{\emph{Institution}}: Instituto Balseiro, Bariloche Atomic Centre (CNEA-UNC)\\
{\emph{Thesis}}: Event Reconstruction with the Surface Detectors of the Pierre
Auger Observatory\\
{\emph{Advisor}}: Dr. Ingomar Allekotte\\ 
\years{2004}\textsc{``Licenciado'' in Physics}\\
{\emph{Institution}}: Instituto Balseiro, Bariloche Atomic Centre (CNEA-UNC)\\
\else
\section*{Educación}
\noindent
%\years{2012}\textsc{Ph.D. in Physics}\\
%{\emph{Institution}}: Bariloche Atomic Centre - Instituto Balseiro, CNEA-UNC\\
%{\emph{Thesis}}: Present and Future Applications for the Water-Cherenkov Detectors of the Pierre Auger Observatory. 
%{\emph{Advisor}}: Dr. Ingomar Allekotte 
\years{2005}\textsc{Magister en Ciencias Físicas}\\
{\emph{Orientación}}: Física de Partículas y Campos\\
{\emph{Institución}}: Instituto Balseiro, Centro Atómico Bariloche (CNEA-UNC)\\
{\emph{Tesis}}: Reconstrucción de eventos con el Detector de Superficie del Observatorio Pierre Auger\\
{\emph{Director}}: Dr. Ingomar Allekotte\\ 
\years{2004}\textsc{Licenciado en Física}\\
{\emph{Institución}}: Instituto Balseiro, Centro Atómico Bariloche (CNEA-UNC)\\
\fi

\section*{Premios \& Becas}
\noindent
\years{2011}Premio ``Mejor Profesor del Instituto Balseiro 2011'' otorgado por la Fundación Balseiro.\\
\years{2008-2010}Beca de posgrado tipo II (CONICET), para la Carrera de doctorado en Física en el Instituto Balseiro (UNC).\\
\years{2005}Beca de posgrado tipo I (FUNC-CNEA), para la Carrera de doctorado en Física en el Instituto Balseiro (UNC).\\
\years{2002-2004}Beca de Iniciación a la Investigación (FUNC) para realizar tareas de investigación en el Observatorio Pierre Auger.\\
\years{2005}Beca de maestría (CNEA), para la carrera de Maestría en Ciencias Físicas en el Instituto Balseiro (UNC).\\
\years{2002-2004}Beca de grado (CNEA), para la carrera de Licenciatura en Física, en el Instituto Balseiro (UNC).
