%! suppress = LineBreak
\ifeng
\section*{Selected Works}
\noindent

This list is a personal selection of the published works I have been directly involved.
In the appendix I include a complete list of publications and presentations at Congresses and Conferences.

\else
\subsection*{Trabajos seleccionados}
\noindent

La lista mostrada a continuación corresponde a una selección personal de los trabajos publicados en los cuales estuve directamente involucrado.
En los apéndices agrego la lista completa de publicaciones y presentaciones en Congresos y Conferencias.
\fi

\noindent
\begin{etaremune}

\item \years{2023} N.A. Santos, S. Dasso, A.M. Gulisano, O. Areso, M. Pereira, H. Asorey, L. Rubinstein, for the LAGO collaboration \href{https://doi.org/10.1016/j.asr.2022.11.041}{First measurements of periodicities and anisotropies of cosmic ray flux observed with a water-Cherenkov detector at the Marambio Antarctic base} Adv. Spa. Res. {\textbf 71}(6) 2967--2976 (2023)

\item \years{2023} J. Sánchez-Villafrades, J. Peña-Rodríguez, H. Asorey, L. A. Núñez, \href{https://doi.org/10.3390/instruments7010007}{Characterization and on-field performance of the MuTe Silicon Photomultipliers} JINST {\textbf 2023} 7(1) (2023) \href{https://arxiv.org/abs/2102.01119}{arXiv:2102.01119}[physics.ins-det]

\item \years{2023} H. Asorey, M. Suárez-Durán and R. Mayo-García, \href{https://doi.org/10.1016/j.apradiso.2023.110752}{ACORDE: A new application for estimating the dose absorbed by passengers and crews in commercial flights} Applied Radiation and Isotopes {\textbf 196} 110752 (2023).

\item \years{2022} H. Asorey and R. Mayo-García, \href{https://doi.org/10.1007/s11227-022-04981-8}{Calculation of the high-energy neutron flux for anticipating errors and recovery techniques in exascale supercomputer centres} J Supercomput, s11227-022-04981-8 (2022).

\item \years{2022} C. Sarmiento-Cano, M. Suárez-Durán, R. Calderón-Ardila, A. Vásquez-Ramírez, A. Jaimes-Motta, S. Dasso, I. Sidelnik, L. A. Núñez, H. Asorey, for the LAGO Collaboration, \href{https://doi.org/10.1140/epjc/s10052-022-10883-z}{The ARTI Framework: Cosmic Rays Atmospheric Background Simulations} Eur. J. Phys C {\textbf 82}(11) 1019 (2022) \href{https://arxiv.org/abs/2010.14591}{arXiv:2010.14591}[astro-ph.IM]

\item \years{2022} R. Calderon-Ardila, H. Asorey, A. Almela, A. Sedoski, C. Varela, N. Leal and M. Gomez-Berisso \href{http://doi.org/10.31526/jais.2022.300}{Development of Mudulus, a Muography detector based on double-synchronized electronics for Geophysical applications}, J. Adv.\ Inst\  Sci. {\textbf{2022}}(January) 300 (2022)

\item \years{2022} A Taboada, C Sarmiento-Cano, A Sedoski, H Asorey\href{https://doi.org/10.31526/jais.2022.266}{Meiga, a Dedicated Framework Used for Muography Applications},  J. Adv. Inst. Sci. {\textbf{2022}}(January), (2022) % \href{ http://arxiv.org/abs/2201.11160}{arXiv:2201.11160}[astro-ph.IM]

\item \years{2022} C. Pérez Bertolli, C. Sarmiento-Cano and H. Asorey, \href{https://afan.df.uba.ar/journal/index.php/analesafa/article/view/2300}{Estimación del Flujo de Muones en el Laboratorio Subterráneo ANDES}, ANALES AFA {\textbf{32}} (4) 106--111 (2022). \ifeng Másperi Price 2020 \else Premio Másperi 2020\fi.

\item \years{2022} A. Días for the TRACE Collaboration, \href{}{PlomBOX - development of a low-cost CMOS device for environmental monitoring}, \en Proceedings of the 17 International Conference on Environmental Science \& Technology, 2021, Athens, Greece, \in Proceedings of the 17th International Conference on Environmental Science and Technology, (2022). \href{http://arxiv.org/abs/2201.03348}{arXiv:2201.03348}[physics.ins-det]

\item \years{2022} J. Peña-Rodríguez, P. A. Salgado-Meza, H. Asorey, L. A. Núñez, A. Núñez-Castiñeyra, C. Sarmiento-Cano, M. Suárez-Durán \href{}{RACIMO@Bucaramanga: A Citizen Science Project on Data Science and Climate Awareness}, Journal of Instrumentation \ifeng submitted \else enviado\fi, (2022). \href{http://arxiv.org/abs/2203.05431}{arXiv:2203.05431}[astro-ph.IM]

\item \years{2022} J. Peña-Rodríguez, A. Vesga-Ramírez, A. Vásquez-Ramírez, M. Suárez-Durán, R. de León-Barrios, D. Sierra-Porta, R. Calderón-Ardila, J. Pisco-Guavabe, H. Asorey, J. D. Sanabria-Gómez, L. A. Núñez \href{https://doi.org/10.31526/jais.2022.271}{Muography in Colombia: simulation framework, instrumentation and data analysis}, J. Adv. Inst. Sci. 2022(June), (2022). \href{ http://arxiv.org/abs/2201.11160}{arXiv:2201.11160}[astro-ph.IM]

\item \years{2021} A.J.\ Rubio-Montero, R. Pagán-Muñoz, R. Mayo-García, A. Pardo-Diaz, I. Sidelnik, H. Asorey, \href{https://doi.org/10.1109/WSC52266.2021.9715360}{A Novel Cloud-Based Framework For Standardized Simulations In The Latin American Giant Observatory (LAGO)}, \en IEEE Proceedings of the 2021 Winter Simulation Conference (WSC), (2021). \href{http://arxiv.org/abs/2204.02716}{arXiv:2204.02716}[astro-ph.IM]

\item \years{2021} H. Asorey for the MuAr group (A. Almela et al), \href{https://indico.cern.ch/event/1033631/contributions/4530674/}{Muography developments within the MuAR project: advances in simulations and new detectors designs}, \en International Workshop on Cosmic-Ray Muography (Muography2021), Ghent, Belgium, 2021.

\item \years{2021} H. Asorey, R. Calderón-Ardila, R. Mayo-García, L.A.\ Núñez, R. Pagán-Muñoz, A.J.\ Rubio-Montero, C. Sarmiento-Cano, I. Sidelnik, M. Suárez-Durán and A. Taboada, for the LAGO Collaboration, \href{https://ccc.ciencias.uchile.cl/2021colage/abstracts/Hern%C3%A1n%20Asorey.pdf}{Extensive Air Showers Simulations: Applications to Geophysics and Astroparticle Physics}, \en XII Latin American Conference on Space Geophysics (COLAGE 2021), Villarrica, Chile, 2021.

\item \years{2021} A.J.\ Rubio-Montero, R. Pagán-Muñoz, R. Mayo-García, A. Pardo-Diaz, I. Sidelnik, H. Asorey for the LAGO Collaboration, \href{https://doi.org/10.22323/1.395.0261}{The EOSC-Synergy cloud services implementation for the Latin American Giant Observatory (LAGO)}, \en Proc.
37th International Cosmic Ray Conference ICRC2021, PoS(ICRC2021)261, Berlín, Germany, 2021.

\item \years{2021} L. Otiniano, H. Asorey, C. Sarmiento-Cano, I. Sidelnik and M. Suárez-Duran for the LAGO Collaboration, \href{https://doi.org/10.22323/1.395.0267}{Simultaneous particles influence on the LAGO’s Water Cherenkov Detectors signals}, \en Proc.
37th International Cosmic Ray Conference ICRC2021, PoS(ICRC2021)267, Berlín, Germany, 2021.

\item \years{2021} R de Leon-Barrios, J Peña-Rodríguez, JD Sanabria-Gómez, A Vásquez-Ramírez, R Calderón-Ardila, C Sarmiento-Cano, A Vesga-Ramirez, D Sierra-Porta, M Suárez-Durán, H Asorey, Luis A Núñez \href{https://doi.org/10.22323/1.395.0280}{Muography for the Colombian Volcanoes}, \en Proc.
37th International Cosmic Ray Conference ICRC2021, PoS(ICRC2021)280, Berlín, Germany, 2021.

\item \years{2021} J Peña-Rodríguez, R de León-Barrios, A Ramírez-Muñóz, D Villabona-Ardila, M Suárez-Durán, A Vásquez-Ramírez, H Asorey, LA Núñez, \href{https://doi.org/10.22323/1.395.0400}{Muography background sources: simulation, characterization, and machine-learning rejection}, \en Proc.
37th International Cosmic Ray Conference ICRC2021, PoS(ICRC2021)400, Berlín, Germany, 2021.

\item \years{2021} J Peña-Rodríguez, R de León-Barrios, A Ramírez-Muñóz, D Villabona-Ardila, M Suárez-Durán, A Vásquez-Ramírez, H Asorey, LA Núñez, \href{https://doi.org/10.22323/1.395.0400}{Muography background sources: simulation, characterization, and machine-learning rejection}, \en Proc.
37th International Cosmic Ray Conference ICRC2021, PoS(ICRC2021)400, Berlín, Germany, 2021.

\item \years{2021} C. Sarmiento-Cano, H. Asorey, J. Sacahui, L. Otiniano, I. Sidelnik for the LAGO Collaboration, \href{https://doi.org/10.22323/1.395.0929}{The Latin American Giant Observatory (LAGO) capabilities for detecting Gamma Ray Bursts}, \en Proc.
37th International Cosmic Ray Conference ICRC2021, PoS(ICRC2021)929, Berlín, Germany, 2021.

\item \years{2021} N.A.\ Santos, S. Dasso, A.M.\ Gulisano, O. Areso, M. Pereira and H. Asorey for the LAGO Collaboration, \href{https://doi.org/10.22323/1.395.304}{Observations of the cosmic ray detector at the Argentine Marambio base in the Antarctic Peninsula}, \en Proc.
37th International Cosmic Ray Conference ICRC2021, PoS(ICRC2021)304, Berlín, Germany, 2021.

\item \years{2021} A Vesga-Ramírez, JD Sanabria-Gómez, D Sierra-Porta, L Arana-Salinas, H Asorey, VA Kudryavtsev, R Calderón-Ardila, LA Núñez, \href{https://doi.org/10.1016/j.jsames.2021.103248}{{Simulated Annealing for Volcano Muography}}, Journal of South American Earth Sciences {\textbf{109}} 103248 (2021) \href{https://arxiv.org/abs/2005.08295}{arXiv:2005.08295}[physics.geo-ph]

\item \years{2020}The Pierre Auger Collaboration, \href{https://doi.org/10.1088/1748-0221/15/10/P10021}{Reconstruction of Events Recorded with the Surface Detector of the Pierre Auger Observatory} JINST {\textbf{15}} P10021 (2020) \href{https://arxiv.org/abs/2007.04139}{arXiv:2007.04139}[astro-ph.IM]

\item \years{2020}The Pierre Auger Collaboration, \href{https://doi.org/10.52712/sciencereviews.v1i4.31}{The Pierre Auger Observatory and its Upgrade} Sci.
Rev. End World {\textbf{1}} (4) 31 (2020)

\item \years{2020}The Pierre Auger Collaboration, \href{https://doi.org/10.1088/1748-0221/15/09/P09002}{Studies on the response of a water-Cherenkov detector of the Pierre Auger Observatory to atmospheric muons using an RPC hodoscope} JINST {\textbf{15}} P09002 (2020) \href{https://arxiv.org/abs/2007.04139}{arXiv:2007.04139}[astro-ph.IM]

\item \years{2020} J. Peña-Rodríguez, L.A.\ Núñez, H. Asorey, \href{https://doi.org/10.22323/1.390.0984}{Characterization of the muography background using the Muon Telescope (MuTe)}, \en Proc.
40th International International Conference on High Energy physics (ICHEP2020), PoS(ICHEP2020)984, Prague,  Czech Republic, 2020. \href{http://arxiv.org/abs/2102.11483}{arXiv:2102.11483}[hep-ex]

\item \years{2020} R. Calderón-Ardila, A Vesga-Ramírez, C Pérez-Bertolli, A Almela, C Sarmiento-Cano, A Taboada, A Sedoski, C Varela, M Gómez, M Gómez-Berisso, H Asorey, \href{https://ui.adsabs.harvard.edu/abs/2020AGUFMNS0130015C/abstract}{Muography Applications in Argentina}, American Geophysical Union Fall Meeting Abstracts, NS013-0015 (2020)

\item \years{2020} R. Calderón-Ardila, H. Asorey, A. Almela, \href{https://doi.org/10.33414/ajea.5.758.2020}{Desarrollo de Técnicas de Muongrafía para Estudios Densitométricos de Objetos de Importancia Estratégica}, AJEA {\textbf{5}} 758 (2020)

\item \years{2020} H. Asorey \ifeng for the TRACE Collaboration \else para la Colaboración TRACE\fi, PlomBOX: Un dispositivo para combatir la contaminación por plomo en agua potable, \ifeng Invited Talk to the Technology and Industry Divission of the Argentinian Physics Association, 105th Annual Meeting of the Argentinian Physics Association, Córdoba, Argentina, 2020 \else Charla Invitada en la División de Industria y Tecnología de la Asociación de Física Argentina, presentada en la 105º Reunión Anual de la Asociación de Física Argentina, Córdoba, Argentina, 2020.\fi

\item \years{2020} J Peña-Rodríguez, J Pisco-Guabave, D Sierra-Porta, M Suárez-Durán, M Arenas-Flórez, LM Pérez-Archila, JD Sanabria-Gómez, LA Núñez \& H Asorey, \href{https://doi.org/10.1088/1748-0221/15/09/P09006}{{Design and construction of MuTe: a hybrid Muon Telescope to study Colombian Volcanoes}}, JINST {\textbf{15}} P09006 (2020) \href{https://arxiv.org/abs/2004.09364}{arXiv:2004.09364}[physics.ins-det]

\item \years{2020} Iván Sidelnik, Hernán Asorey, Nicolás Guarin, Mauricio Suaréz Durán, José Lipovetzky, Luis Horacio Arnaldi, Martín Pérez, Miguel Sofo Haro, Mariano Gómez Berisso, Fabricio Alcalde Bessia \& Juan Jerónimo Blostein, \href{https://doi.org/10.1016/j.nima.2019.163172}{{Enhancing neutron detection capabilities of a water Cherenkov detector}}, NIM {\textbf{A955}} 163172 (2020) %\href{https://arxiv.org/abs/1912.10081}{arXiv:1912.10081}[physics.ins-det]

\item \years{2020} Iván Sidelnik, Hernán Asorey, Nicolás Guarin, Mauricio Suaréz Durán, Mariano Gómez Berisso, José Lipovetzky \& Juan Jerónimo Blostein, \href{https://doi.org/10.1016/j.asr.2020.02.024}{{Simulation of 500 MeV neutrons by using NaCl doped Water Cherenkov detector}}, Adv.
Space Res. {\textbf{65}}(9) 2216--2222 (2020) %\href{https://arxiv.org/abs/1912.10081}{arXiv:1912.10081}[physics.ins-det]

\item \years{2020} Iván Sidelnik, Hernán Asorey, Nicolás Guarin, Mauricio Suaréz Durán, Fabricio Alcalde Bessia, Luis Horacio Arnaldi, Mariano Gómez Berisso, José Lipovetzky, Martín Pérez, Miguel Sofo Haro \& Juan Jerónimo Blostein, \href{https://doi.org/10.1016/j.nima.2019.03.017}{{Neutron detection capabilities of Water Cherenkov Detectors}}, NIM {\textbf{A952}} 161962 (2020) %\href{https://arxiv.org/abs/1912.10081}{arXiv:1912.10081}[physics.ins-det]

\item \years{2020} A Vásquez-Ramírez, M Suárez-Durán, A Jaimes-Motta, R Calderón-Ardila, J Peña-Rodríguez, J Sánchez-Villafrades, JD Sanabria-Gómez, L. A. Núñez \& H Asorey, \href{https://doi.org/10.1088/1748-0221/15/08/P08004}{{Simulated Response of MuTe, a Hybrid Muon Telescope}}, JINST {\textbf{15}} O8004 (2020) \href{https://arxiv.org/abs/1912.10081}{arXiv:1912.10081}[physics.ins-det]

\item \years{2019} The LAGO Collaboration, \href{https://arxiv.org/abs/1909.10039}{Contributions of the LAGO Collaboration to the 36th ICRC}, \en Proc.
36th International Cosmic Ray Conference, PoS(ICRC2019)358, Madison, USA, 2019. \href{http://arxiv.org/abs/1909.10039}{arXiv:1909.10039}[physics.astro-ph]

\item \years{2019} Jesús Peña-Rodríguez, Adriana Vásquez-Ramírez, José D Sanabria-Gómez, Luis A Núñez, David Sierra-Porta \& Hernán Asorey, \href{https://pos.sissa.it/358/381/}{Calibration and first measurements of MuTe: a hybrid Muon Telescope for geological structures}, \en Proc.
36th International Cosmic Ray Conference, PoS(ICRC2019)358 381, Madison, USA, 2019. \href{http://arxiv.org/abs/1909.09732}{arXiv:1909.09732}[physics.ins-det]

\item \years{2018} H Asorey, R Calderón-Ardila, K Forero-Gutiérrez, et al., \href{http://dx.doi.org/10.22517/23447214.17501}{miniMuTe: A muon telescope prototype for studying volcanic structures with cosmic ray flux}, Scientia et technica {\textbf{23}}(3) 386--391 (2018)

\item \years{2018}H. Asorey, R. Calderón-Ardila, C. R. Carvajal-Bohorquez, et al \href{http://dx.doi.org/10.22517/23447214.17561}{Astroparticle projects at the Eastern Colombia region: facilities and instrumentation}, Scientia et technica {\textbf{23}}(3) 392--397 (2018)

\item \years{2018}The Pierre Auger Collaboration, \href{}{{Large-scale cosmic-ray anisotropies above 4 EeV measured by the Pierre Auger Observatory}}, APJ {\textbf{868}}(1) 4 (2018) \href{https://arxiv.org/abs/1808.03579}{arXiv:1808.03579}[astro-ph.IM]

\item \years{2018}H. Asorey, L. A. Núñez, M. Suarez-Duran \href{https://doi.org/10.1002/2017SW001774}{{Preliminary Results from The Latin American Giant Observatory Space Weather Simulation Chain}} Space Weather {\textbf{16}}(5) 461--475 (2018) \href{https://arxiv.org/abs/1802.08867}{arXiv:1802.08867}[physics.geo-ph]

\item \years{2018} H. Asorey, L. A. Nunez \& C. Sarmiento-Cano, \href{http://dx.doi.org/10.1590/1806-9126-rbef-2018-0092}{{Early Exposure of Digital Natives to Environments, Methodologies and Research Techniques in University Physics}} Rev.
Bras.
Ensino Fís {\textbf{40}}(4) e5407 (2018) \href{http://arxiv.org/abs/1501.04916}{arXiv:1501.04916}[physics.ed-ph]

\item \years{2018}The Pierre Auger Collaboration, \href{https://doi.org/10.3847/2041-8213/aaa66d}{{An Indication of Anisotropy in Arrival Directions of Ultra-high-energy Cosmic Rays through Comparison to the Flux Pattern of Extragalactic Gamma-Ray Sources}}, ApJ {\textbf{L853}}(2) L29 (2018) \href{https://arxiv.org/abs/1801.06160}{arXiv:1801.06160}[astro-ph.CO]

\item \years{2017}The Pierre Auger Collaboration, \href{https://doi.org/10.1126/science.aan4338}{{Observation of a large-scale anisotropy in the arrival directions of cosmic rays above $8\times 10^{18}$ eV}}, Science {\textbf{357}}(6357) 1266--1270 (2017) \href{https://arxiv.org/abs/1709.07321}{arXiv:1709.07321}[astro-ph.HE]

\item \years{2017} H. Asorey, A. Jaimes-Motta, L. A. Núñez, J. Peña-Rodríguez, C. Sarmiento-Cano \& M. Súarez-Duran for the LAGO Collaboration, \href{http://www.astroscu.unam.mx/rmaa/RMxAC..49/PDF/RMxAC..49\_poster3.pdf}{{The Calibration of the GUANE Array: Extensive Air Showers Reconstruction and Space Weather Studies}} \en Proc.
XV Latin American Regional IAU Meeting LARIM2016, Cartagena, Colombia, Rev.
Mex.
AA, {\textbf{49}} 145--145 (2017)

\item \years{2017} H. Asorey, A. Balaguera-Rojas, A. Martínez-Méndez, L. A. Núñez, J. Peña-Rodríguez, P. Salgado-Meza, C. Sarmiento-Cano \& M. Súarez-Duran, \href{http://www.astroscu.unam.mx/rmaa/RMxAC..49/PDF/RMxAC..49\_poster2.pdf}{{Astro-climate: A citizen Science Climate Awareness}} \en Proc.
XV Latin American Regional IAU Meeting LARIM2016, Cartagena, Colombia, Rev.
Mex.
AA, {\textbf{49}} 144--144 (2017)

\item \years{2017} H. Asorey, A. Balaguera-Rojas, R. Calderón Ardila, L. A. Núñez, J. D. Sanabria-Gómez, M. Súarez-Duran \& A. Tapia, \href{http://www.astroscu.unam.mx/rmaa/RMxAC..49/PDF/RMxAC..49\_poster2.pdf}{{Muon Telescope (MUTE): A first study using Geant4}} \en Proc.
XV Latin American Regional IAU Meeting LARIM2016, Cartagena, Colombia, Rev.
Mex.
AA, {\textbf{49}} 144--144 (2017)

\item \years{2017} H. Asorey, L. A. Núñez \& M. Súarez-Duran, \href{http://www.astroscu.unam.mx/rmaa/RMxAC..49/PDF/RMxAC..49\_oral6.pdf}{{A Simulation Chain for the LAGO Space Weather Program}} \en Proc.
XV Latin American Regional IAU Meeting LARIM2016, Cartagena, Colombia, Rev.
Mex.
AA, {\textbf{49}} 56--56 (2017) \href{http://arxiv.org/abs/1704.07681}{arXiv:1704.07681}[physics.space-ph]

\item \years{2017} H. Asorey, A. Balaguera-Rojas, L. A. Núñez, J. D. Sanabria-Gómez, C. Sarmiento-Cano, M. Súarez-Duran, M. Valencia-Otero, \& A. Vesga-Ramírez, \href{http://www.astroscu.unam.mx/rmaa/RMxAC..49/PDF/RMxAC..49\_oral4.pdf}{{Astroparticle Techniques: Colombia active volcano candidates for Muon Telescope}} \en Proc.
XV Latin American Regional IAU Meeting LARIM2016, Cartagena, Colombia, Rev.
Mex.
AA, {\textbf{49}} 54--54 (2017) \href{http://arxiv.org/abs/1704.04967}{arXiv:1704.04967}[physics.geo-ph]

\item \years{2017}H. Asorey, A. Martínez-Méndez, L. A. Núñez \& A. Valbuena-Delgado, \href{http://www.astroscu.unam.mx/rmaa/RMxAC..49/PDF/RMxAC..49\_oral5.pdf}{{LAGO Distributed Network Of Data Repositories}} \en Proc.
XV Latin American Regional IAU Meeting LARIM2016, Cartagena, Colombia, Rev.
Mex.
AA {\textbf{49}} 55--55 (2017) \href{http://arxiv.org/abs/1704.03885}{arXiv:1704.03885}[cs.DL]

\item \years{2017}H. Asorey, L. Núñez, C. Y. Pérez Arias, S. Pinilla, F. Quiñonez \& M. Suárez-Durán, \href{http://www.astroscu.unam.mx/rmaa/RMxAC..49/PDF/RMxAC..49\_oral7.pdf}{{Astroparticle Techniques: Simulating Cosmic Rays induced Background Radiation on Aircrafts}} \en Proc.
XV Latin American Regional IAU Meeting LARIM2016, Cartagena, Colombia, Rev.
Mex.
AA, {\textbf{49}} 57--57 (2017) \href{http://arxiv.org/abs/1704.03419}{arXiv:1704.03419}[physics.space-ph]

\item \years{2017}H. Asorey, L. A. Núñez, J. D. Sanabria-Gomez, C. Sarmiento-Cano, D. Sierra-Porta, M. Suarez-Duran, M. Valencia-Otero, A. Vesga-Ramírez, \href{}{{Muon Tomography sites for Colombia volcanoes}} (2017) \href{http://arxiv.org/abs/1705.09884}{arXiv:1705.09884}[physics.geo-ph]

\item \years{2017}The Pierre Auger Collaboration, \href{https://doi.org/10.1088/1748-0221/12/03/P03002}{{Muon counting using silicon photomultipliers in the AMIGA detector of the Pierre Auger observatory}} JINST {\textbf 12} P03002 (2017) \href{http://arxiv.org/abs/1703.06193}{arXiv:1703.06193}[astro-ph.IM]

\item \years{2017}I. Sidelnik \& H. Asorey, \href{https://doi.org/10.1016/j.nima.2017.02.069}{{LAGO: the Latin American Giant Observatory}}, NIM-A {\textbf{876}} 173--175 (2017) \href{http://arxiv.org/abs/1703.05337}{arXiv:1703.05337}[astro-ph.IM]

\item \years{2017} I. Sidelnik, H. Asorey, J. J. Blostein \& M. Gómez Berisso, \href{https://doi.org/10.1016/j.nima.2017.02.048}{{Neutron Detection Using a Water Cherenkov Detector with Pure Water and a Single PMT}}, NIM-A {\textbf{876}} 153--155 (2017)

\item \years{2017}The Pierre Auger Collaboration, \href{https://doi.org/10.1088/1748-0221/12/02/P02006}{{Impact of atmospheric effects on the energy reconstruction of air showers observed by the surface detectors of the Pierre Auger Observatory}} JINST {\textbf 12} P02006 (2017) \href{http://arxiv.org/abs/1702.02835}{arXiv:1702.02835}[astro-ph.IM]

\item \years{2016} The Pierre Auger Collaboration, {{The Pierre Auger Observatory Upgrade-Preliminary Design Report}}, \href{http://arxiv.org/abs/1604.03637}{arXiv:1604.03637}[astro-ph.IM]

\item \years{2015}H. Asorey for the LAGO Collaboration, {{LAGO: the Latin American Giant Observatory}}, \en Proc.
34th International Cosmic Ray Conference, PoS(ICRC2015)247, The Hague, The Netherlands, 2015

\item \years{2016} H. Asorey, R. Mayo-García, L.A.\ Núñez, M. Rodríguez-Pascual, A. J. Rubio-Montero, M. Suarez Durán, \& L.A.\ Torres-Niño for the LAGO Collaboration, {{\href{http://dx.doi.org/10.1109/CCGrid.2016.110}{The Latin American Giant Observatory: a successful collaboration in Latin America based on Cosmic Rays and computer science domains}}}, \en Proc.
2016--16th IEEE/ACM International Symposium on Cluster, Cloud and Grid Computing (CCGrid), IEEE Proceedings, pp 707--711, Cartagena, Colombia, 2016, \href{http://arxiv.org/abs/1605.09295}{arXiv:1605.09295}[astro-ph.IM]

\item \years{2015} I. Sidelnik, H. Asorey, J. J. Blostein, M. Gómez Berisso, H. Arnaldi, M. Sofo Haro, {{Detección de Neutrones mediante efecto Cherenkov en Agua}}, Actas de la Reunión Anual de la Asociación Argentina de Tecnología Nuclear (2015).

\item \years{2015}The Pierre Auger Collaboration, \href{http://dx.doi.org/10.1016/j.nima.2015.06.058}{{The Pierre Auger Cosmic Ray Observatory}} NIM {\textbf{A 798}} 172--213 (2015) \href{http://arxiv.org/abs/1502.01323}{arXiv:1502.01323}[astro-ph.HE]

\item \years{2015}H. Asorey \& L. A. Núñez, {{Astroparticle Physics at Eastern Colombia}}, \en Proc.
César Lattes Meeting, \ifeng accepted \else aceptado,\fi Niterói, Brazil, 2015 \href{http://arxiv.org/abs/1510.01305}{arXiv:1510.01305}[astro-ph.IM]

\item \years{2015}H. Asorey for the LAGO Collaboration, {{LAGO: the Latin American Giant Observatory}}, \en Proc.
34th International Cosmic Ray Conference, PoS(ICRC2015)247, The Hague, The Netherlands, 2015

\item \years{2015}S. Dasso, A.M.\ Gulisano, J.J\  Masías-Meza \& H. Asorey for the LAGO Collaboration, {{A Project to Install Water-Cherenkov Detectors in the Antarctic Peninsula as part of the LAGO Detection Network}}, \en Proc.
34th International Cosmic Ray Conference, PoS(ICRC2015)105, The Hague, The Netherlands, 2015

\item \years{2015}H. Asorey, S. Dasso, L.A.\ Núñez, Y. Perez, C. Sarmiento \& M. Suárez-Durán for the LAGO Collaboration, {{The LAGO Space Weather Program: Directional Geomagnetic Effects, Background Fluence Calculations and Multi-Spectral Data Analysis}}, \en Proc.
34th International Cosmic Ray Conference, PoS(ICRC2015)142, The Hague, The Netherlands, 2015

\item \years{2015}H. Asorey, P. Miranda, A. Núñez-Castiñeyra, L.A.\ Núñez, J. Salinas, C. Sarmiento-Cano, R. Ticona \& A. Velarde for the LAGO Collaboration, {{Analysis of Background Cosmic Ray Rate in the 2010--2012 Period from the LAGO-Chacaltaya Detectors}}, \en Proc.
34th International Cosmic Ray Conference, PoS(ICRC2015)414, The Hague, The Netherlands, 2015

\item \years{2015}H. Asorey, D. Cazar-Ramírez, R. Mayo-García, L.A.\ Núñez, M. Rodríguez-Pascual \& L.A.\ Torres-Niño for the LAGO Collaboration, {{Data Accessibility, Reproducibility and Trustworthiness with LAGO Data Repositories}}, \en Proc.
34th International Cosmic Ray Conference, PoS(ICRC2015)672, The Hague, The Netherlands, 2015


\item \years{2014}S. Pinilla, H. Asorey, L.A.\ Núñez, {{Cosmic Rays Induced Background Radiation on Board of Commercial Flights}}, \en Proc.
X SILAFAE, Nuc.
Phys.
B Proc.
Supp., \ifeng accepted, \else aceptado,\fi\ Medellín, Colombia, 2014

\item \years{2014}H. Asorey for the LAGO Collaboration, {{The Latin American Giant Observatory}}, \en Proc.
X SILAFAE, Nuc.
Phys.
B Proc.
Supp., \ifeng accepted, \else aceptado,\fi\ Medellín, Colombia, 2014

\item \years{2014}S. Pinilla, H. Asorey, L.A.\ Núñez, {{Cosmic Rays Induced Background Radiation on Board of Commercial Flights}}, \en Proc.
X SILAFAE,  Nuc.
Part.
Phys.
Proc. {\textbf{267-269}} 418--420 (2015), Medellín, Colombia, 2014

\item \years{2014}R. Calderón, H. Asorey, L.A.\ Núñez for the LAGO Collaboration, {{Geant4 based simulation of the Water Cherenkov Detectors of the LAGO Project}}, \en Proc.
X SILAFAE, Nuc.
Part.
Phys.
Proc. {\textbf{267-269}} 424--426 (2015), Medellín, Colombia, 2014

\item \years{2014}The Pierre Auger Collaboration, \href{http://dx.doi.org/10.1088/0004-637X/794/2/172}{{Searches for Large-scale Anisotropy in the Arrival Directions of Cosmic Rays Detected above Energy of 1019 eV at the Pierre Auger Observatory and the Telescope Array}} ApJ {\textbf{794}}(2), 172 (2014)

\item \years{2014}H. Asorey \& S. Dasso for the LAGO Collaboration, {{The LAGO Project Space Weather Program}}, \en Proc. 40th COSPAR Scientific Assembly, Adv. Space Res. \in Proceedings of the 40th COSPAR Scientific Assembly, Moscú, Rusia, 2014

\item \years{2014}H. Asorey, J.I.\ Castro \& A. López Dávalos, \href{http://www.revistas.unc.edu.ar/index.php/revistaEF/article/view/9512}{{Una deducción analítica simple de la hodógrafa para el problema de Kepler}}, Rev. Ens. Fís. {\textbf{26}}(1), 63--73 (2014).

\item \years{2013}H. Asorey \& L. Núñez, {{Astronomy and Astrophysics in the Colombian Andes: the PAS Project}} \en Proc.
XIV Latin American Regional IAU Meeting LARIM2014, Rev. Mex. AA Conf. Series, Florianopolis, Brazil (2013).

\item \years{2013}H. Asorey for the LAGO Collaboration, \href{https://www.cbpf.br/~icrc2013/papers/icrc2013-0856.pdf}{The LAGO Solar Project}, \en Proc. 33th International Cosmic Ray Conference, Rio de Janeiro, Brazil, ICRC2013-0856 (2013)

\item \years{2013}H. Asorey, D. Melo {{et al.}}, \href{https://www.cbpf.br/~icrc2013/papers/icrc2013-1236.pdf}{Characterization of San Antonio de los Cobres for a Cherenkov telescope array in energy range from 20 GeV to 130 GeV}, \en Proc. 33 International Cosmic Ray Conference, Rio de Janeiro, Brazil, ICRC2013-1236 (2013)

\item \years{2012}S. Dasso \& H. Asorey, for the Pierre Auger Collaboration, \href{http://dx.doi.org/10.1016/j.asr.2011.12.028}{{ The scaler mode in
the Pierre Auger Observatory to study heliospheric modulation of cosmic rays}}, Adv. Space Res. {\textbf{49}} (11), 1563--1569 (2012)

\item \years{2012} H. Asorey, M. Arribere, X. Bertou, M. Gómez Berisso, F. Sánchez,
{{Expected Backgrounds at the ANDES Underground Laboratory}}
\ifeng
plenary talk given at the
\else 
charla plenaria dada en el
\fi
Third International Workshop for the Design of the ANDES Underground Laboratory, Valparaiso, Chile, 11--12 Jan 2012.

\item \years{2011}The Pierre Auger Collaboration,
\href{http://dx.doi.org/10.1088/1748-0221/6/01/P01003}{{The Pierre Auger
Observatory Scaler Mode for the Study of the Modulation of Galactic Cosmic Rays
due to Solar Activity}}, JINST {\textbf 6} P01003--P01020 (2011). \ifeng $^*${\textbf{Coordinator}} \else $^*${\textbf{Coordinador}} \fi

\item \years{2011} The Pierre Auger Collaboration, \href{http://dx.doi.org/10.1016/j.astropartphys.2011.08.001}{{The Lateral Trigger Probability function for UHE Cosmic Rays Showers detected by the Pierre Auger Observatory}}, Astropart. Phys. {\textbf{35}} (5), 266--276 (2011)

\item \years{2011}H. Asorey \& A. López Dávalos, {{Fermi Problem: Power developed at the eruption of the Puyehue-Cordón Caulle volcanic system in June
2011}}, \href{http://arxiv.org/abs/1109.1165}{arXiv:1109.1165v1}[physics.ed-ph]. \ifeng Selected as the best \href{http://arxiv.org}{arXiv} paper of September 2011 by the \else Seleccionado como el mejor trabajo enviado al \href{http://arxiv.org}{arXiv} durante Setiembre del 2011 por el blog \fi \href{http://www.technologyreview.com/blog/arxiv/27140/}{M.I.T. Technology Review Physics arXiv Blog}, (2011)

\item \years{2011}H. Asorey, A. López Dávalos \& A. Clúa, \href{https://dialnet.unirioja.es/servlet/articulo?codigo=4026852}{{Potencia de la Erupción del Volcán Puyehue como un Problema de Fermi}}, \ifeng plenary talk given in the XVII Physics Education National Meeting APFA 2011 of the Argentinian Professors in Physics Association, Villa Giardino, Argentina, Oct 2011. \else charla plenaria presentada en la XVII Reunión Nacional de Educación en Física APFA 2011 de la Asociación de Profesores de Física de Argentina, Villa Giardino, Argentina, Oct 2011. \fi Rev. Ens. Fís. {\textbf{24}}(2), 49--54 (2011)

\item \years{2011}I. Allekotte, H. Arnaldi, H. Asorey, X. Bertou, M. Gómez Berisso, \& M. Sofo Haro, {{Development of ultra-fast and ultra low power consumption electronics in the Bariloche Particle and Radiation Detection Laboratory}},
\ifeng
poster presentation in the 96$^{\mathrm{th}}$ National Reunion SUF-AFA2011 of the Argentinian Physics Association, Montevideo, Uruguay, 20--23 Sept 2011.
\else
poster presentado en la 96$^{\mathrm{th}}$ Reunión Nacional SUF-AFA2011 de la Asociación Argentina de Física, Montevideo, Uruguay, 20--23 Sept 2011.
\fi

\item \years{2011}H. Asorey[Pierre Auger Collaboration], {{Low energy radiation
measurements with the water Cherenkov detector array of the Pierre Auger
Observatory}}, \en Proc.
32 International Cosmic Ray Conference, vol.
11
462--465, Beijing, China, 11--18 Ago 2011

\item \years{2011}The Pierre Auger Collaboration,
\href{http://dx.doi.org/10.1016/j.astropartphys.2010.12.007}{{Search for
First Harmonic Modulation in the Right Ascension Distribution of Cosmic Rays
Detected at the Pierre Auger Observatory}}, Astropart.
Phys. {\textbf 34} 627--639
(2011)

\item \years{2010}J. Blümer \& The Pierre Auger Collaboration,
\href{http://dx.doi.org/10.1088/1367-2630/12/3/035001}{{The Northern Site
of the Pierre Auger Observatory}}, Journal of Physics {\textbf 12} (3) 035001

\item \years{2010}The Pierre Auger Collaboration,
\href{http://dx.doi.org/10.1016/j.physletb.2010.02.013}{{Measurement of
the energy spectrum of cosmic rays above $10^{18}$ eV using the Pierre Auger
Observatory}}, Phys.
Lett. {\textbf B685} 239--246 (2010),\\
\href{http://arxiv.org/abs/1002.1975}{arXiv:1002.1975v1}[astro-ph.HE]

\item \years{2010}The Pierre Auger Collaboration,
\href{http://dx.doi.org/10.1016/j.nima.2009.11.018}{{Trigger and Aperture
of the Surface Detector Array of the Pierre Auger Observatory}}, NIM {\textbf A613}
29--39, (2010)

\item \years{2010}H. Asorey[LAGO Collaboration], {{The Large Aperture Gamma Ray
Burst Observatory (LAGO)}}, plenary talk in the 3$^{\mathrm{rd}}$ International Workshop of
High Energy Physics in the LHC Era HEP2010, Valparaiso, Chile, 4--8 Jan 2010.

\item \years{2009}H. Asorey[Pierre Auger Collaboration], {{Cosmic Ray Solar
Modulation Studies at the Pierre Auger Observatory}}, \en Proc.
31th
International Cosmic Ray Conference, Lodz, Poland, 8--15 Jul 2009.

\item \years{2009} The Pierre Auger Collaboration,
\href{http://dx.doi.org/10.1016/j.astropartphys.2009.06.004}{{Atmospheric
effects on extensive air showers observed with the Surface Detector of the
Pierre Auger Observatory}}, Astropart.
Phys. {\textbf 32}, 89--99, (2009),
\href{http://arxiv.org/abs/0906.5497/}{arXiv:0906.5497v2}[astro-ph.IM]

\item \years{2008}The Pierre Auger Collaboration,
\href{http://dx.doi.org/10.1103/PhysRevLett.101.061101}{{Observation of
the Suppression of the Flux of Cosmic Rays above $4\times10^{19}$\,eV.}}, PRL
{\textbf 101} 061101 (2008)

\item \years{2008}The Pierre Auger Collaboration,
\href{http://dx.doi.org/10.1016/j.astropartphys.2008.01.003}{{Upper limit
on the cosmic-ray photon flux above 10$^{19}$\,eV using the surface detector of
the Pierre Auger Observatory.}}, Astropart.
Phys. {\textbf 29} 243--256 (2008)

\item \years{2008}The Pierre Auger Collaboration,
\href{http://dx.doi.org/10.1016/j.astropartphys.2008.01.002}{{Correlation
of the highest-energy cosmic rays with the positions of nearby active galactic
nuclei.}}, Astropart.
Phys. {\textbf 29} 188--204 (2008)

\item \years{2007}The Pierre Auger Collaboration,
\href{http://dx.doi.org/10.1126/science.1151124}{{Correlation of the
highest energy cosmic rays with nearby extragalactic objects.}}, Science {\textbf
318} 939--943 (2007)

\item \years{2008}D. Allard { et al.} [LAGO Collaboration],
\href{http://dx.doi.org/10.1016/j.nima.2008.07.041}{{Use of
water-Cherenkov detectors to detect Gamma Ray Bursts at the Large Aperture GRB
Observatory (LAGO)}}, NIM {\textbf A595} 70--72 (2008)

\item \years{2007}D. Allard { et al.} [LAGO Collaboration], {{Looking for
the high energy component of GRBs at the Large Aperture GRB Observatory}}, \en
Proc.
30$^{\mathrm{th}}$ International Cosmic Ray Conference,  Mérida, Mexico, 3--11 Jul
2007.

\item \years{2007}The Pierre Auger Collaboration,
\href{http://dx.doi.org/10.1016/j.astropartphys.2006.11.002}{{Anisotropy
studies around the galactic centre at EeV energies with the Auger
Observatory.}},  Astropart.
Phys. {\textbf 27} 244--253 (2007)

\item \years{2006}D. Allard { et al.} [LAGO Collaboration], {{The Large
Aperture GRB aperture}}, \en Proc.
of the Observational Astronomy in Argentina
Workshop, Buenos Aires.

\end{etaremune}