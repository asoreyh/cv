%faltan las fechas de designacion de los cargos docentes (y expedientes!)
%faltan URLS de las publicaciones
%falta la cantidad de paginas de cada publicacion
%hay que mejorar la descripcion de los TPs de compu grafica
%falta agregar algebra 1

\documentclass[11pt, a4paper]{moderncv}
% \usepackage[spanish]{babel}
\usepackage{amsmath, amssymb}
\usepackage[utf8]{inputenc}
\moderncvtheme[blue]{classic}
\usepackage{enumerate}

%Ajuste de margenes de pagina
\usepackage[scale=0.8]{geometry}
% \setlength{\hintscolumnwidth}{3cm}
% \AtBeginDocument{\setlength{\maketitlenamewidth}{6cm}} %Solo para el tema clasico
\AtBeginDocument{\recomputelengths}

%Datos personales
\firstname{Iván Pedro}
\familyname{\\Sidelnik}
\title{Curriculum Vit\ae}
\address{Huilqui 12872}{CP 8400, San Carlos de Bariloche\\ Prov. Rio Negro, Argentina}
%\phone{+54--11--4855-5983}
\phone{(work) +54--294--444--5151-int. 36}
\mobile{+54--9--11--6446--2184}
\email{ivan.sidelnik@cnea.gov.ar \\ ivan.sidelnik@iteda.cnea.gov.ar}
%\fax{}
\extrainfo{June 23th of 1980}
%\photo[64pt]{picture_file}
%\quote{}

\begin{document}
\maketitle
 \section{Personal data}
 \cvlistitem{\textbf{Name:} Iván Pedro}
 \cvlistitem{\textbf{Family name:} Sidelnik}
 \cvlistitem{\textbf{Date of birth:} June 23th of 1980}
 \cvlistitem{\textbf{Address:} Huilqui 12872, CP 8400, San Carlos de Bariloche, Rio Negro, Argentina}
% \cvlistitem{\textbf{Phone:} +54--11--4855--5983}
  \cvlistitem{\textbf{Phone (work):} +54--294--444--5151-int. 36}
 \cvlistitem{\textbf{Mobile:} +54--9--11--6446--2184}
 \cvlistitem{\textbf{e-mail:} ivan.sidelnik@cnea.gov.ar \\ ivan.sidelnik@iteda.cnea.gov.ar}

\section{Education, investigation}

\cventry{2012--2014}{Post doctoral fellow of CONICET}{Centro Atómico Bariloche, Particle and Field Group}{Cosmic ray physics}{}{}

\cventry{2007--2012}{PhD. in Physics, thesis title: Study of ultra high energy cosmic rays with AMIGA an extension of the Pierre Auger Observatory}
{Facultad de Ciencias Exactas y Naturales, Universidad de Buenos Aires}{Instituto de Tecnologías y Detección en Astropartículas (ITeDA), CNEA, CONICET, UNSAM.}{CONICET scholarship (March 2012)}{}

\cventry{2000--2007}{Licenciatura en Ciencias Físicas}{Thesis title : Ultra energetic air shower propagation on the Earth over Pierre Auger Observatory muon counters.}
{Facultad de Ciencias Exactas y Naturales, Universidad de Buenos Aires}{Pierre Auger Observatory - Centro Atómico Constituyentes}{}

\cventry{2006}{Laboratorio de Procesado de Imágenes}{}{Departamento de física, Facultad de Ciencias Exactas y Naturales, Universidad de Buenos Aires}
{Director: Dra. Silvia Ledesma}{Experimental experience in Optics in the frame of the subjects ``Laboratorio 6'' and ``Laboratorio 7''}

\cventry{1999}{Secondary school}{}{ EET N$^{\circ}$ 1 ``Almirante Guillermo Brown''}{``Técnico en Mecánico''}{Zárate Pcia. de Buenos Aires, Argentina}{}

\cventry{1997}{Secondary school}{}{Escuela de Educación Técnica ``Henry Ford''}{``Auxiliar Técnico Mecánico de Mantenimiento y de Planta''}{Gral. Pacheco Pcia. de Buenos Aires, Argentina}

\clearpage
% proyectos como integrante (con cantidad de plata y fechas mas o menos de aplicacion)
% becas de investigacion (tipo 1 y tipo 2 por separado)
% becas de participacion en congresos, etc.
%\section{Investigación}
%\cventry{2007--2011}{Tesis de Licenciatura en Ciencias Físicas}{Métodos numéricos en física atómica}
%{Grupo de Colisiones Atómicas, Instituto de Astronomía y Física del Espacio}{}{}
%
%\cventry{2007--2008}{Laboratorio de Espectroscopía Mössbauer}{}{Centro Atómico Constituyentes, CNEA}
%{Directora: Dra. Cinthia Ramos}{En el marco de las asignaturas ``Laboratorio 6'' y ``Laboratorio 7''}

\section{Awards and scholarships}

\cventry{04/2010--04/2013}{Member of the project ``Subsidios para infraestructura y equipamiento CyT''}{Title: Celda Unitaria de AMIGA}
{Brief description: The project AMIGA (Auger Muons and Infill for the Ground Array) is being installed in a small area of 23,5 km$^2$ in the Pierre Auger
Observatory, consisting in pair of detectors of 433 and 750 m of distance between each other. Due to the requisits of good energy resolution and primary
mass compisiton identification of the cosmic ray, those pairs will be composed of surface detectors of Cherenkov in water and muon counters buried, of 
plastic scintillators. The installation of the engineering array has begun, called Unitary cell, an hexagon of of seven pairs of detectors one in each 
vertex and in the center. This work is aimed for the electronic tests of the Unitary Cell. Prototipes of the electronics boards will be redisigned and 
final prototypes will be built.}{}{Fund of 330000 pesos argentinos (60000 euros). ID code: UTN-FRBA-25/C094}

\cventry{01/2007--12/2008}{Member of the project PIP (Project of plurianual Investigation) - CONICET (Consejo Nacional de Investigaciones Científicas y Técnicas)}{Title: Aceptancia, calibración y análisis de datos de los instrumentos del Observatorio Pierre Auger.}{Description: participation in the project to study the acceptance of the Pierre Auger Observatory surface detector, energy calibration with data aquired from the Observatory}{Principal investigator: Alberto Etchegoyen}{Fund of 116000 pesos argentinos (21091 euros). ID number of the project 6126}

\cventry{12/2006--12/2009}{Member of the project UTN-FRBA with the title: Diseño y Electrónica de AMIGA del Proyecto Auger}{}
{Brief description: Design and construction of prototypes of the AMIGA electronics, including the test facility of PMTs (photomultiplier tubes) and the first prototipes of muon counters to be buried at the Centro Atómico Constityente}{}{Fund of 900000 pesos argentinos (163637 euros). ID code: UTN-FRBA-25/C094}

\cventry{04/2012--04/2014}{CONICET (Consejo Nacional de Investigaciones Científicas y Técnicas) post doctoral scholarship}{}{}{Title: study of anisotropy in the arrival directions and composition of ultra high energy cosmic rays detcted using the Pierre Auger Observatory}{}

\cventry{04/2010--04/2012}{CONICET (Consejo Nacional de Investigaciones Científicas y Técnicas) scholarship, type II}{}{Scholarship granted to finish my PhD, 2 years}{Title: study of ultra high energy cosmic rays with the enhancements of the Pierre Auger Observatory}{}

\cventry{04/2007--04/2010}{CONICET (Consejo Nacional de Investigaciones Científicas y Técnicas) scholarship, type I}{}{Scholarship granted to start my PhD, 3 years}{Title: study of ultra high energy cosmic rays with the enhancements of the Pierre Auger Observatory}{}

\cventry{03/2006--03/2007}{FUNC (Fundación Nacional Universidad de Cuyo) scholarship}{}{Scholarship granted for the last year of my physics degree `Licenciatura', 1 year}{Title: Propagación de chubascos cósmicos Ultra energéticos an la Tierra para contadores de muones del Proyecto Auger}{}

%\clearpage

%cada materia por separado, especificando la materia (si es para un tano, ponele de que se trata o el programa sintetico)
%pone material que hayas desarrollado para alguna materia que hayas dado
\section{Teaching experience}

\cventry{Second semester 2011}{Teaching asistance (1$^{\circ}$ class)}{Laboratorio de Física 2 (for physicists)}
{Facultad de Ciencias Exactas y Naturales}{Universidad de Buenos Aires}{}

\cventry{First semester 2011}{Teaching asistance (1$^{\circ}$ class)}{Laboratorio de Física 2 (for physicists)}
{Facultad de Ciencias Exactas y Naturales}{Universidad de Buenos Aires}{}

\cventry{Second semester 2010}{Teaching asistance (1$^{\circ}$ class)}{Laboratorio de Física 2 (for physicists)}
{Facultad de Ciencias Exactas y Naturales}{Universidad de Buenos Aires}{}

\cventry{Summer 2010}{Teaching asistance (1$^{\circ}$ class)}{Laboratorio de Física 5 (for physicists)}
{Facultad de Ciencias Exactas y Naturales}{Universidad de Buenos Aires}{}

\cventry{Second semester 2009}{Teaching asistance (1$^{\circ}$ class)}{Física 3 (for physicists)}
{Facultad de Ciencias Exactas y Naturales}{Universidad de Buenos Aires}{}

\cventry{Second semester 2008}{Teaching asistance (1$^{\circ}$ class)}{Laboratorio de Física 1 (for chemist)}
{Facultad de Ciencias Exactas y Naturales}{Universidad de Buenos Aires}{}

\cventry{First semester 2008}{Teaching asistance (1$^{\circ}$ class)}{Laboratorio de Física 4 (for physicists)}
{Facultad de Ciencias Exactas y Naturales}{Universidad de Buenos Aires}{}

\cventry{Second semester 2007}{Teaching asistance (2$^{\circ}$ class)}{Laboratorio de Física 4 (for physicists)}
{Facultad de Ciencias Exactas y Naturales}{Universidad de Buenos Aires}{}

\cventry{First semester 2007}{Teaching asistance (2$^{\circ}$ class)}{Laboratorio de Física 2 (for physicists)}
{Facultad de Ciencias Exactas y Naturales}{Universidad de Buenos Aires}{}

\cventry{Second semester 2006}{Teaching asistance (2$^{\circ}$ class)}{Laboratorio de Física 2 (for biologists and geologists)}
{Facultad de Ciencias Exactas y Naturales}{Universidad de Buenos Aires}{}

\cventry{First semester 2006}{Teaching asistance (2$^{\circ}$ class)}{Laboratorio de Física 1 (for physicists)}
{Facultad de Ciencias Exactas y Naturales}{Universidad de Buenos Aires}{}

\cventry{Second semester 2005}{Teaching asistance (2$^{\circ}$ class)}{Física 3 (for physicists)}
{Facultad de Ciencias Exactas y Naturales}{Universidad de Buenos Aires}{}

\cventry{Summer 2005}{Teaching asistance (2$^{\circ}$ class)}{Laboratorio de Física 4 (for physicists)}
{Facultad de Ciencias Exactas y Naturales}{Universidad de Buenos Aires}{}

\cventry{Summer 2004}{Teaching asistance (2$^{\circ}$ class)}{Laboratorio de Física 1 (for biologists and geologists)}
{Facultad de Ciencias Exactas y Naturales}{Universidad de Buenos Aires}{}

All courses were given at the physics department of the University of Buenos Aires. Where is no named the course was given for physicists.
The explanation of the courses are:\\

\cvlistitem {Laboratorio de Física 1 (for biologists and geologists) is the laboratory of the introductory course of mechanics and fluids}
\cvlistitem {Laboratorio de Física 1 (for chemist) is the laboratory of the introductory course of optics and fluids}
\cvlistitem {Laboratorio de Física 1, mechanics}
\cvlistitem {Laboratorio de Física 2, optics and waves}
\cvlistitem {Laboratorio de Física 4, thermodynamics and instrumentation}
\cvlistitem {Laboratorio de Física 5, nuclear and particle physics}
\cvlistitem {Física 3: electromagnetism}

\cventry{2011}{Teacher ad-honorem of the Bachillerato Popular de jóvenes y adultos ``1$^{\circ}$ de Mayo''}{}{Secondary school for adults and young people that could not finish on regular basis}{Subject: Ciencias básicas aplicadas a la producción. First and second year.}{}{}

\clearpage
%pone la informacion del curso, como las horas de cursada totales, las fechas, el regimen de aprobacion, y los temas
%las optativas de la carrera y el doctorado tambien, sobre todo porque los tanos no tienen idea!

\section{Courses taken}

All courses with two middle term exams and a final exams, except where is indicated.\\

\cventry{2010}{Advanced classical mechanics}{final grade: 10}{Semestral course of 10 hours per week dictated by professor G. Mindlin, Facultad de Ciencias Exactas y Naturales, Universidad de Buenos Aires}{Course in elementary non linear dinamics.}{http://www.lsd.df.uba.ar/materias/dnl/}

\cventry{2009}{Introduction to Astronomy}{final grade: 6}{Semestral course of 10 hours per week dictated by professor G. Dubner, at Universidad de San Martin}{An introduction to the study of the cosmos, the stars and galaxies as long as the different instruments that are used to study them.}{}

\cventry{2009}{Physics of elementary particles}{final grade: 8}{Semestral course of 10 hours per week dictated by professor D. De Florian, Facultad de Ciencias Exactas y Naturales, Universidad de Buenos Aires}{Course of Loops, renormalization, structure of hadrons, partons, etc.}{}

\cventry{2008}{Experimental Astrophysics}{final grade: 10}{Semestral course of 10 hours per week dictated by professor A. Etchegoyen and A. Arazi, at Universidad de San Martin}{This course was dedicated to the different techniques that are used in particle accelerator to study astrophysics and astronomical processes.}{}

Extra courses taken during my Licenciatura degree\\

\cventry{2006}{Laboratory of elemental electronics}{Final grade: 10}{Semestral course of 6 hours per week dictated by professor C. Moreno at Facultad de Ciencias Exactas y Naturales, Universidad de Buenos Aires}{Weekly report on the work in the laboratory and a final exam consistent in an Oral presentation of the final work.}{http://focuslab.lfp.uba.ar/public/Electronica/}

\cventry{2005}{Introduction to quantum field theory}{Final grade: 10}{Semestral course of 10 hours per week dictated by professor G. Lozano, at Facultad de Ciencias Exactas y Naturales, Universidad de Buenos Aires}{}{http://www.df.uba.ar/~marc/Campos/programa.html}

\cventry{2008}{Laboratory of plasma and fluids}{Final grade: 10}{Semestral course of 6 hours of class per week plus 4 laboratory hours dictated by professor H. Kelly at Facultad de Ciencias Exactas y Naturales, Universidad de Buenos Aires}{}{}

%ponencias orales, trabajos murales, y asistencias!
%\section{Participación en Reuniones}
%\cventry{2011}{Asistente}{$2^\text{da}$ reunión conjunta AFA-SUF}
%{organizada por la Asociación de Física Argentina y la Sociedad Uruguaya de Física}
%{Montevideo, Uruguay. 20 a 23 de Septiembre.}{}\\

%\cventry{2009}{Ponencia oral}{VIII Congreso Internacional sobre Investigación en Didáctica de las Ciencias}
%{organizado por la Universidad Autónoma de Barcelona}{Barcelona, España. 7 a 10 de Septiembre.}
%{Enseñanza de la capacidad eléctrica por analogía con un cilindro de gas natural comprimido.}\\
%
%\cventry{2008}{Trabajo mural}{XI Latin American Conference on the Applications of the Mössbauer Effect}
%{organizada por la Sociedad Argentina de Materiales}{La Plata, Buenos Aires, Argentina. 9 a 14 de Noviembre.}
%{Mössbauer spectroscopy analysis on a tempered martensitic 9 \% Cr steel.}\\
%
%\cventry{2008}{Trabajo mural}{$2^{do}$ Encuentro de Jóvenes Investigadores en Ciencia y Tecnología de Materiales}
%{organizada por la Sociedad Argentina de Materiales}{Posadas, Misiones. 16 y 17 de Octubre.}
%{Análisis Mössbauer de un acero martensítico-ferrítico 9\% Cr con distintos tiempos de revenido.}\\
%
%\cventry{2008}{Trabajos murales}{$1^\text{era}$ reunión conjunta AFA-SUF}
%{organizada por la Asociación de Física Argentina y la Sociedad Uruguaya de Física}
%{Buenos Aires, Argentina. 15 a 19 de Septiembre}{
%	Problemas en la propagación temporal de la ecuación de Schrödinger. \\
%	Contribución resonante a la excitación de iones por impacto electrónico.
%}\\
%
%\cventry{2008}{Asistente}{8th conference of the Debian project - Debconf 8}
%{Mar del Plata}{Buenos Aires, Argentina. 10 a 16 de Agosto.}{}\\
%
%\cventry{2008}{Asistente}{New Developments on Metallurgy and Applications of High Strength Steels}
%{Buenos Aires}{Argentina. 26 a 28 de Mayo.}{}\\
%
%\cventry{2008}{Trabajo mural}{$3^{eras}$ Jornadas de Educación en Física y Biofísica}
%{organizada por el departamento de física del Ciclo Básico Común, Universidad de Buenos Aires}
%{Facultad de Agronomía, Buenos Aires, Argentina. 14 y 15 de Marzo.}
%{John Rambo y la conservación del momento lineal.}\\
%
%\cventry{2007}{Trabajo mural}{$92^{\circ}$ Reunión Nacional de Física}
%{organizada por la Asociación de Física Argentina}{Salta, Argentina. 24 a 28 de Septiembre.}
%{Métodos numéricos en física atómica: condiciones de borde transparentes.}
%
% \cventry{2005}{Asistente}{Primera Jornada de Didáctica de las Ciencias Naturales}
% {organizada por el Centro de Formación e Investigación en Enseñanza de las Ciencias}
% {Facultad de Ciencias Exactas y Naturales, Universidad de Buenos Aires}{}


%\section{Publicaciones}
%\cventry{2010}{A. E. Garriz, A. Sztrajman, D. Mitnik}
%{Running into troubles with the time-dependent propagation of a wavepacket}{}{}
%{Eur. J. Phys.  \textbf{31}, 4, 785--799}
%%publisher:IOP publishing (UK), ISSN 0143-0807, julio de 2010
%
%\cventry{2010}{C. P. Ramos, A. Sztrajman, R. Bianchi, C.A. Danón, C. Saragovi}
%{Mössbauer Spectroscopy analysis on a tempered martensitic 9 \% Cr steel}{}{}
%{Hyperfine Interactions, \textbf{195}, 1, 257--263}
%%publisher: springer netherlands, ISSN 0304-3843 (print) 1572-9540 (online), Oct 2009 (online)
%
%\cventry{2009}{J. Sztrajman, A. Sztrajman}
%{Enseñanza de la capacidad eléctrica por analogía con un cilindro de gas natural comprimido}{}{}
%{Enseñanza de las Ciencias. Número extra. ISSN 0212-4521}
%%publisher: 
%
%\cventry{2008}{R. Bianchi, A. Sztrajman, C. P. Ramos}
%{Análisis Mössbauer de un acero martensítico-ferrítico 9\% Cr con distintos tiempos de revenido}{}{}
%{Libro de Resúmenes del \textit{$2^{do}$ Encuentro de Jóvenes Investigadores en Ciencia y Tecnología de Materiales}, p. 42}
%%publicado: 07/10/2008, ISBN 978-987-24687-0-5
%
%\cventry{2008}{A. Sztrajman, D. Mitnik}
%{Contribución resonante a la excitación de iones por impacto electrónico}{}{}
%{Libro de Resúmenes de la \textit{$1^\text{era}$ reunión conjunta AFA-SUF}, p. 80}
%
%\cventry{2008}{A. Garriz, A. Sztrajman, D. Mitnik}
%{Problemas en la propagación temporal de la ecuación de Schrödinger}{}{}
%{Libro de Resúmenes de la \textit{$1^\text{era}$ reunión conjunta AFA-SUF}, p. 163}
%
%\cventry{2008}{A. Sztrajman}
%{John Rambo y la Conservación del Momento Lineal}{}{}
%{Memorias de las \textit{$3^{eras}$ Jornadas de Educación en Física y Biofísica}}
%%ISBN 978-987-21295-6-9 %falta la pagina
%
%\cventry{2007}{A. Sztrajman, D. Mitnik}
%{Métodos numéricos en física atómica: condiciones de borde transparente}{}{}
%{Libro de Resúmenes de la \textit{$92^{\circ}$ Reunión Nacional de Física}, p. 202.}
%%editado por AFA tucuman

%averiguar como se dice formacion de recursos humanos  te cabe http://en.wikipedia.org/wiki/Human_resources
\section{Human resources}

\cventry{2011}{Co-director Tesis Licenciatura}{Thesis title: Caracterización del sistema de detección para los contadores de muones de AMIGA, como parte del Observatorio Pierre Auger}{}
{Student: Federico Barabás}{Starting: March 2011}

\cventry{2008}{Co-director Laboratorio 7}{This is a subject of the final stage of the physics degree where the students have to work in a Real laboratory to start knowing about experimental physics.}
{Title: Construcción y caracterización de un detector de muones patrón para el proyecto Pierre Auger}{Students: Dafne Goijman, Hernán De María}{Director: Eduardo Colombo}{}

%tecnopolis, con el tiempo que te fumaste esa cosa! y la url,      la verdad que si... me la re fumé....
%tu articulo de diario! o yo lo pondria en una seccion aparte, que se llame "participacion en el jetset" o la participacion gato del jet set... o con el gato...
%extension es lo mismo que divulgacion? si, es lo mismo... todo es outreach, o a lo sumo science communication
%en todo caso va todo junto, 
\section{Outreach activities}

\cventry{07/2011--08/2011}{Participation in ``Tecnópolis'', an outstanding exposition of argentinian science and culture. I participate with presentations and explanations about the detectors of the Pierre Auger Observatory and cosmic ray physics and activities related to it. The presentations were continuos in an 4 hour basis in a three days a week rotation.}{``Tecnopolis is an impressive display of the country's tradition of scientific excellence, in such areas as aerospace, nuclear energy and medicine, among others.'' from http://www.larouchepac.com/node/18852}
{}{all information url: http://tecnopolis.ar/full/}{}

\cventry{2011}{Journal article at the newspaper ``Tiempo Argentino'' }{``Analizan los rayos cósmicos con detectores desarrollados en el país''}
{}{http://tiempo.infonews.com/notas/analizan-los-rayos-cosmicos-con-detectores-desarrollados-pais}{}

\cventry{2011}{Presentation at the Asociación Argentina Amigos de la Astronomía}{``Estudiando los fenómenos más energéticos que llegan a la Tierra del espacio exterior, los rayos cósmicos de ultra alta energía''}
{Oral presentantion for all audience}{http://www.asaramas.com/section.aspx?id=7415}{}

\cventry{2010}{Organizing member of the 95a National Reunion of the Physics Asociation (RNAFA)}{Malargüe, Mendoza, Argentina. This was the first reunion outside of an official center.}{}{http://fisica.cab.cnea.gov.ar/afa2010/}

\cventry{2005}{Participating of the Science communication programm of the Facultad de Ciencias Exactas y Naturales, University of Buenos Aires}{}
{Equivalent to a Teaching assitance degree, Secretaria General and Secretaría de Extensión, Graduados y Bienestar estudiantil}{}{}

\cventry{2005}{Invierno en el planetario}{}
{Colaboration in physics experiments and demonstrations to secondary school students and general audience at the Planetarium of Buenos Aires}
{Organized by the Facultad de Ciencias Exactas y Naturales, University of Buenos Aires}{}

\cventry{2002--2007}{Semana de la Física (Physics week)}
{Colaboration in physics experiments and demonstrations to secondary school students and general audience over one selected week of the year}
{Organized by the physics department of the Facultad de Ciencias Exactas y Naturales, University of Buenos Aires}{}{}%{http://difusion.df.uba.ar/joomla/index.php?option=com_content&task=blogcategory&id=26&Itemid=41}{}%aca tengo un problema que la chagar no compila... el nabo que puso la pagina le dejo este link, sabes como arreglarlo?


%aca vos fuiste codep del claustro de graduados. pone los años. 
\section{University administration}
\cventry{2008}{Member of CoDep (Consejo Departamental) by the graduate senate}{}{An internal organism of the physics department that take desisions on Jury selection for the selection process of teachers and theacher assitance, licences, rules, and internal issues attained to the department.}
{Facultad de Ciencias Exactas y Naturales, Universidad de Buenos Aires}{}

%pone los programas que desarrollaste, con el lenguaje y para que sirve.
%cualquier desarrollo blah blah... varios scripts de ROOT
%\section{Desarrollos en IT}
%\cventry{2011}{Implementación en Javascript/Webgl/HTML}{Aplicación de algoritmos de dibujo e interpolación de curvas en computación
%gráfica}{}{}{}
%
%\cventry{2011}{Página Web}{correspondiente a la asignatura de posgrado ``Métodos numéricos para ecuaciones diferenciales en derivadas
%parciales''}{Facultad de Ciencias Exactas y Naturales}{Universidad de Buenos Aires}
%{\url{http://www.df.uba.ar/users/asztrajman/metnum_2do2011}}
%
%\cventry{2009}{Página Web}{correspondiente a la asignatura ``Laboratorio de Física 1 (ByG)''}
%{Facultad de Ciencias Exactas y Naturales}{Universidad de Buenos Aires}
%{\url{http://www.df.uba.ar/users/capuzzi/f1bygV2009/labo1web/index.html}}
%
%\cventry{2009}{Implementación en Fortran 77}{Método de Runge-Kutta para cálculo de soluciones a la ecuación de 
%Schrödinger con condiciones de borde entrantes y salientes}{}{}{}
%
%\cventry{2008}{Página Web}{correspondiente a la asignatura ``Física 3''}
%{Facultad de Ciencias Exactas y Naturales}{Universidad de Buenos Aires}
%% \verb|http://www.df.uba.ar/users/miraglia|
%{\url{http://www.df.uba.ar/users/msilvia/index.html}}
%
%\cventry{2008}{Implementación en Fortran 77 (paralelizada con MPI)}{Algoritmo Leap-Frog para la evolución
%temporal de la ecuación de Schrodinger en una dimensión}{}{}{}
%
%\cventry{2007}{Implementación en C++}{Método de Split-Operator para la
%evolución temporal de la ecuación de Schrodinger en una dimensión}{}{}{}
%
%\cventry{2007}{Implementación en C++}{Método de Euler para la simulación de una malla elástica bidimensional. 
%Representación de soluciones a través de OpenGL}{}{}{}

\clearpage

\section{Languages}
\cvlanguage{Castellano}{Mother tongue}{}
\cvlanguage{English}{Speaks, reads, write}{}
\cvlanguage{Italian}{Speaks and reads}{}

%explicalo mejor esto. pone que haces desarrollos en ROOT, y en C/C++. Que tenés experiencia en esto y aquello.
%
\section{Software knowledge}

\cvcomputer{SSOO}{Linux, Windows}{}{}
\cvcomputer{Programming}{C/C++, Root (CERN)}{}{}
\cventry{}{Experience in high energy physics software with the package Root from CERN, http://root.cern.ch/}{Programming skills in C/C++}{}{}{}%aca si que no se que poner.... fuck no entiendo bien por que compila mal esto...
\cvcomputer{Programming}{Mathematica, bash scripting, awk}{}{}
\cvcomputer{Cientific}{\LaTeX, Origin}{}{}

\section{Internship}

\cventry{09/2011}{Stay in the Bariloche Auger group to study cosmic rays anisotropy}{}{San Carlos de Bariloche, Rio Negro, Argentina}{Study of cosmic rays anisotropy using different techniques with Dr. E. Roulet and Dr. S. Mollerach}{}{}

\cventry{01/2009--08/2009}{Stay in the Auger group of the Universitá degli studi di Torino to study the infill array of the Pierre Auger Observatory}{}{Study of the infill surface array of the Pierre Auger Observatory, directed by Professor G. Navarra}{}{We start working in the reconstruction of extended air showers with the infill and compared them with results from the main array, evaluating differences between reconstructions in geometry and energy estimator S(600). Then we perform the first studies of anisotropies with the infill using the east west method.}

\cventry{08/2008}{Stay in the Bariloche Auger group to learn about the software CDAS, for reconstruction of cosmic rays using the surface array}{}{}{}{}

\section{Other works}

\cventry{06/1998}{Mechanical maintenance}{Technitian in the Nuclear Power Plant Atucha I, one of the three nuclear plants in Argentina}{}{}{}

\cventry{05/1999--07/1999}{Mechanical maintenance}{Technitian in the Nuclear Power Plant Atucha I, one of the three nuclear plants in Argentina}{}{}{}

\cventry{08/2010--03/2012}{Sub-task leader of the AMIGA enhancement of the Pierre Auger Collaboration, in the area ``Deployment of Muon Counters for the Auger Observatory''.}{}{Head of instalation of muon counters for AMIGA an enhancement of the Pierre Auger Observatory}{Description of the task: the muon counters are new detectors for AMIGA that is an integral part of the Pierre Auger Observatory, for measuring cosmic rays. These detectors have 30 m$^2$ of surface divided in three modules that must be installed 2.5 m underground to prevento further contamination coming mainly from electromagnetic particles. Each detector is installed near a surface station of Auger. At the beggining the plan is to install seven of these muon counters forming an hexagon with a central detector. Nowadays there are three modules of 10 m$^2$ and one of 5 m$^2$ (to test small modules). These also implies site selection for the instalation, cabling of the detectors on the surface and underground, power system instalation (solar pannels plus holding for the batteries of the detectors, etc), access tubes instalation for maintenance, and emplace the detector in the field as long as all things related to the work of it (except for the electronics). To understand more about the AMIGA project please see the research objetives.}{}{}

\end{document}
